\chapter{Súčasná podoba vrstvy Model}\label{architektura_model_vrstvy}

Model vrstva je štvrtou vrstvou MVVM(MV) architektúry. Táto vrstva je pre zvyšné vrstvy zdrojom dát a mechanizmov spracovávajúcich tieto dáta. Delí sa do mnohých oblastí. Každú túto oblasť má na starosti nejaký \textit{manažér}. Manažéri sú pristupovaný z Model view vrstvy a doručujú jej svoje služby, či už informatívne alebo výpočtové. Viac informácií o Model vrstve je možné nájsť v podsekcii \ref{model}.

Implementáciu Model vrstvy z veľkej časti poznačila už spomínaná strata typových informácií pri komunikácii s Model view vrstvou. Väčšina oblastí sa s touto stratou vysporadúva rovnakým spôsobom:
\begin{itemize}
    \item \textbf{Komunikácia smerujúca z vrstvy Model} - Pre všetky dátové štruktúry, ktoré sa využívajú mimo vrstvy Model, sú vytvorení predkovia, ktorí sú buďto negenerickí alebo obsahujú kovariantné generické parametre. Tým pádom vedia byť prenášané negenerickým spôsobom mimo Model vrstvy.
    \item \textbf{Komunikácia smerujúca do vrstvy Model} - Pre dátové štruktúry, ktorých typovú informáciu je potrebné v Model vrstve opäť nadobudnúť bol vytvorený tzv. \textit{Generic visitor pattern}. Mnoho   
\end{itemize} 

\begin{definice}
    \textbf{Generic visitor pattern} je obdoba klasického návrhového vzoru \emph{Visitor pattern}. Pracuje na podobnom princípe, kedy na navštevovanej inštancii je volaná metóda \texttt{Visit} ktorej sa predá inštancia volajúcej triedy. Následne navštívená inštancia odpovie zavolaním metódy \texttt{Accept} na dodanej volajúcej inštancii a predaním samej seba v argumente indikuje, ktorý overload metódy \texttt{Accept} sa má zavolať.

    V prípade generic visitor pattern-u však volajúca trieda neimplementuje overload metódy \texttt{Accept} pre každý možný typ navštevovanej inštancie ale len jednu generickú metódu \texttt{Accept<T>}. V typovom parametri \texttt{T} je následne predaná informácia o type navštívenej inštancie.
\end{definice}

V následujúcich sekciách sa pozrieme detailnejšie na aktuálne existujúce oblasti vrstvy Model a ich vnútorné mechanizmy.

\section{Template-y}

Táto oblasť sa stará o správu atribútových template-ov (len \textit{template-ov} pre jednoduchosť). Viac informácií o funkcii template-ov je možné nájsť v podsekcii \ref{templatey}. Čo do obsahu je to jedna z najmenších oblastí. 

Hlavným uzlom pre komunikáciu z Model view vrstvy je singleton trieda \texttt{TemplateManager}. Ten ponúka kolekciu všetkých template-ov, ktoré je možné v aplikácii použiť.

\bigskip

Template-y sú reprezentované pomocou tried, ktoré implementujú rozhranie \textbf{\texttt{ITemplate<TVertexAttributes, TEdgeAttributes>}}. Tento generický interface núti svoje implementácie, aby dosadením jeho typových parametrov indikovali typy vrcholových a hranových atribútov ktoré reprezentujú. 

Ďalej v aplikácii existuje ešte aj negenerický predok \textbf{\texttt{ITemplate}} spomenutého rozhrania, ktorý slúži na komunikáciu mimo vrstvy Model. Definuje vlastnosti, ktoré sú potrebné pri práci s template-ami vo vonkajšom prostredí. Tento interface by nemal byť nikdy priamo implementovaný.

Template-ové triedy by mali byť implementované ako singleton-y. V aplikácii je totiž vždy využívaná len jedna inštancia každého template-ového typu a to tá zahrnutá v kolekcii template-ov v triede \texttt{TemplateManager}.

Template-y podporujú návrhový vzor \textit{generic visitor pattern}. Ten sa svojou funkcionalitou trocha odlišuje od iných implementácií. Metóda \texttt{Visit} je totiž definovaná v rozhraní \texttt{ITemplate}, ale metóda \texttt{Accept} vracia typový parameter, ktorý je obmedzený na rozhranie \texttt{ITemplate<TVertexAttributes, TEdgeAttributes>}. Tento trik slúži k tomu, aby aj na premennej typu \texttt{ITemplate} bolo možné ihneď získať typy vrcholových a hranových atribútov daného template-u. Tento trik funguje vďaka predpokladu, že všetky template-ové triedy implementujú rozhranie \texttt{ITemplate<TVertexAttributes, TEdgeAttributes>}. 

\section{Mapy}

Táto oblasť sa stará o vytváranie a správu máp. Viac informácií o koncepte Máp je možné nájsť v podsekcii \ref{mapy}.

Hlavným uzlom pre komunikáciu z Model view vrstvy je singleton tried \texttt{TemplateManager}. Ten poknúka kolekciu mapových formátov, ktorá obsahuje reprezentantov všetkých možných máp, ktoré je možné v aplikácii využiť. Každý reprezentant reprezentuje jeden typ mapy, jeden formát. Popri tom ponúka metódy pre vytváranie mapových inštancií a metódy pre identifikáciu mapových formátov. 

\bigskip

Mapy sú reprezentované pomocou tried, ktoré implementujú rozhranie \textbf{\texttt{IMap}}. Toto rozhranie nie je ničím príliš zaujímavé. Definuje iba pár vlastností využívaných mimo vrstvy Model.

Triedy máp následne môžu implementovať niektoré z ďalších definovaných rozhraní, ktoré pridávajú ďalšie kontrakty. Tieto kontrakty sú zamerané na geografické lokalizovanie a rozlohu máp. Sú využívané predovšetkým pri získavaní dodatočných výškových dát vzhľadom k danej mape. Ak si je mapový typ vedomí toho, že na vytvorenie jeho mapovej reprezentácie bude potrebná výpomoc výškových dát, mal by implementovať aspoň jeden z interface-ov, ktorý definuje informácie o geo-lokalite a rozlohe mapy.

Ako už bolo naznačené, pri mapách je potrebné, aby v aplikácii niekto zastupoval ich formáty. K tomuto slúžia triedy implementujúce trojicu rozhraní:
\begin{itemize}
    \item \textbf{\texttt{IMapFormat<out TMap>}} - Je určený pre komunikáciu mimo vrstvy Model. Definuje vlastnosti, ktoré sú potrebné pri práci s mapovým zástupcom vo vonkajšom prostredí a metódu na vytváranie mapy zastupovaného typu. Tento interface by nemal byť priamo implementovaný.
    \item \textbf{\texttt{IMapIdentifier<in TMap>}} - Slúži na identifikáciu odpovedajúceho mapového formátu pre konkrétny typ mapy. Vďaka kontravariantnej povahe jeho typového parametru bude táto identifikácia fungovať správne aj pre potomkov typu \texttt{TMap}. Tento interface by nemal byť priamo implementovaný.
    \item \textbf{\texttt{IMapRepresentative<TMap>}} - Zastupuje jeden konkrétny typ mapy. Je potomkom predošlých dvoch interface-ov - spája ich funkcionality. Tento interface je určený k tomu, aby bol priamo implementovaný mapovými zástupcami.   
\end{itemize}
Na to, aby bolo možné mapový typ využiť v aplikácii, musí byť jeho zástupca zahrnutý v príslušnej kolekcii mapového manažéra. Z tohto dôvodu je na mieste, aby títo zástupcovia boli implementovaný ako singleton triedy.

Mapy podporujú návrhový vzor \textit{Generic visitor pattern}. Ten je definovaný ako pre typ \texttt{IMap}, tak aj pre typ \texttt{IGeoLocatedMap} (rozhranie pridávajúce kontrakt o geografickej lokalite mapy).

\section{Mapové reprezentácie}

Ďalšia z oblastí sa stará o vytváranie a spravovanie mapových reprezentácii/grafov. Čo do obsahu aj komplexnosti sa jedná o jednu z najväčších oblastí. Viac informácií o koncepte mapových je možné nájsť v podsekcii \ref{mapove_reprezentacie}.

Hlavným uzlom pre prácu s touto oblasťou z vonkajšieho prostredia je singleton trieda \texttt{MapRepreManager}. Ten ponúka kolekciu obsahujúcu zástupcov všetkých typov mapových reprezentácií, ktoré je možné v aplikácii využiť. Popri tom obsahuje metódy určené na 
\begin{itemize}
    \item vytváranie mapových reprezentácií,
    \item identifikáciu reprezentácií vytvoriteľných pre konkrétnu kombináciu typov template-u a mapy,
    \item detekciu potreby výškových dát pri konštrukcii mapovej reprezentácie.
\end{itemize}

%Dopisat info o konstruktoch ako Ivertex, IEdge

\bigskip

Štruktúra dát, ktoré súvisia s mapovými reprezentáciami odzrkadluje dvojakosť významov mapovej reprezentácie a grafu ako bolo popísané v podsekcii \ref{mapove_reprezentacie}. Každému typu mapovej reprezentácie je prisúdený konkrétny typ grafu.

Mapové reprezentácie/grafy sú v aplikácii reprezentované triedami, ktoré implementujú rozhranie \textbf{\texttt{IGraph<TVertexAttributes, TEdgeAttributes>}}. Toto rozhranie reprezentuje myšlienku grafu, ktorý je mapovou reprezentáciou. Jeho typové parametre definujú atribúty ktoré sú používané v jeho vrcholoch a hranách. Obsahuje kontrakty, ktoré musia grafy všetkých typov naplňovať. Aktuálne je to len jedna metóda vracajúca graf do jeho základného stavu.

\texttt{IGraph<TVertexAttributes, TEdgeAttributes>} má za predchodcu rozhranie \textbf{\texttt{IMapRepre}} reprezentujúce myšlienku samotnej mapovej reprezentácie. Cez toto rozhranie sú mapové reprezentácie/grafy distribuované vonkajšiemu svetu. K tomuto účelu rozhranie definuje vlastnosti využívané mimo vrstvy Model. Grafová podstata mapových reprezentácií je tým vonkajšiemu svetu ukrytá a len špecifické oblasti vrstvy Model ju znova nadobúdajú a využívajú.

Každá mapová reprezentácia/graf má množinu svojich implementácií. Implementácie sú tvorené pre konkrétne kombinácie template-ov a mapových formátov. Konkrétny typ mapovej reprezentácie môže byť vytvorený len pre takú kombináciu template-u a mapového formátu, pre ktorý je vytvorená príslušná implementácia.

V aplikácii sú naďalej prítomné ďalšie rozhrania, ktoré môžu (a mali by v čo najväčšom rozsahu) jednotlivé grafy implementovať. Tieto rozhrania definujú kontrakty týkajúce sa vlastností a schopností generovaných grafov.

\bigskip

Na to aby mohla byť mapová reprezentácia, graf či implementácia v aplikácii použitá, musí pre ne existovať vhodný zástupca. Tento zástupca je následne poskytnutý na patričnom registračnom mieste.
\begin{itemize}
    \item Zástupcovia typov mapových reprezentácií sú reprezentovaní triedami, ktoré implementujú rozhranie \textbf{\texttt{IMapRepreRepresentative<out TMapRepre>}}. Tieto triedy sú určené (podobne ako typy mapových reprezentácií) pre komunikáciu mimo vrstvy Model. Definujú vlastnosti, ktoré sú využívané vonkajším svetom pre získanie informácií o zastupovanej mapovej reprezentácii. 
    
    Taktiež si držia indikátorovú kolekciu zástupcov všetkých implementácií zastupovanej mapovej reprezentácie. Táto kolekcia sa následne využíva ako pri identifikácii použiteľných kombinácií template-ov a mapových formátov, tak aj pri samotnom vytváraní mapových reprezentácií/grafov.
    
    Nakoľko typ mapovej reprezentácia je vždy asociovaný s nejakým typom grafu, zástupca typu mapovej reprezentácia disponuje aj referenciou na zástupcu typu daného grafu. Tento zástupca sa následne využíva v procese vytvárania mapovej reprezentácie/grafu. Taktiež špecifické oblasti Model vrstvy ho môžu využiť ku otestovaniu vlastností zastupovaného grafu.

    Samotné vytváranie mapových reprezentácií prebieha za pomoci metód implementovaných rozhraním \texttt{IMapRepreRepresentative<out TMapRepre>}. 

    Na to, aby bolo možné typ mapovej reprezentácie/grafu využiť v aplikácii, musí byť jeho zástupca zahrnutý v príslušnej kolekcii manažéra mapových reprezentácií.

    \item Zástupcovia typov grafov sú reprezentovaní triedami, ktoré implementujú rozhranie \textbf{\texttt{IGraphRepresentative}}. Toto rozhranie je definované s tromi typovými parametrami ktoré definujú typ zastupovaného grafu  a typy atribútov, ktoré sú použité v jeho vrcholoch a hranách. Tento interface nedefinuje žiadny kontrakt pre grafových zástupcov. Implementuje iba metódy, ktoré sú využívané v procese vytvárania mapovej reprezentácie/grafu. Je využívaný špecifickými oblasťami vrstvy Model ku otestovaniu vlastností zastupovaného grafu.

    Každý zástupca grafu je asociovaný s konkrétnym zástupcom mapovej reprezentácie. Ten si naň drží referenciu a využíva ho v procese vytvárania mapovej reprezentácie/grafu.

    \item Ku reprezentácii zástupcov jednotlivých implementácií slúžia triedy ktoré dedia od abstraktnej triedy \textbf{\texttt{ElevDataIndepImplementationRep}} alebo od abstraktnej triedy \textbf{\texttt{ElevDataDepImplementationRep}}. Tieto triedy disponujú dvojitou funkcionalitou:
    \begin{itemize}
        \item Obsahujú vlastnosti indikujúce template a mapový formát, na základe ktorých je implementácia mapovej reprezentácie/grafu agregovaná. Tieto vlastnosti sú naplnením kontraktu indikujúceho rozhrania \textbf{\texttt{IImplementationIndicator}}.
        \item Je schopná skonštruovať (alebo nechať skonštruovať) zastúpenú implementáciu. Tieto triedy sa líšia predovšetkým v potrebe dodatočných výškových dát v procese tvorby danej implementácie. Schopnosť skonštruovať danú implementáciu je naplnením kontraktu jedného z rozhraní \textbf{\texttt{IImplementationElevDataIndepConstr}}, respektíve \textbf{\texttt{IImplementationElevDataDepConstr}}. 
    \end{itemize}
    Taktiež disponujú množinou typových parametrov, z ktorých každý má svoj špecifický význam:
    \begin{itemize}
        \item \texttt{TTemplate} - definuje typ template-u, pre ktorý je zastupovaná implementácia vytvorená
        \item \texttt{TGraph} - definuje typ mapovej reprezentácie/grafu, pre ktorý je zastupovaná implementácia vytvorená,
        \item \texttt{TVertexAttributes, TEdgeAttributes} - definujú typy atribútov použitých v implementovanom grafe
        \item \texttt{TMap, TUsableSubMap} - tieto parametre sú jemne zavádzajúce. Prvý z nich hovorí o tom, aký typ mapy je navonok indikovaný pomocou vlastnosti mapového formátu. Druhý hovorí o tom, ktorý typ priraditeľný do typu \texttt{TMap} je v skutočnosti potrebný na vytvorenie mapovej reprezentácie. Malo by byť zaručené okolným prostredím, že pokiaľ je nejaká mapa správneho formátu, tak je určite možné ju použiť pre vytvorenie mapovej reprezentácie. 
    \end{itemize}
    
    Zástupcovia jednotlivých implementácií mapovej reprezentácie/grafu sú zahrnutí v kolekcii zástupcu tejto mapovej reprezentácie.

\end{itemize}

Bolo by vhodné, aby všetci spomenutí zástupcovia boli implementovaní ako singleton triedy. Od každého z nich je totiž za celú dobu behu programu potrebné vytvoriť iba jedinú inštanciu, ktorá je poskytovaná zvyšku aplikácie na patričnom mieste.

Poslednou súčasťou oblasti mapových reprezentácií sú triedy, ktoré reprezentujú vrcholy a hrany používané v grafoch. Jednotlivé typy sa potom líšia dodávanými vlastnosťami.   

\section{Užívateľské modely}

V tejto oblasti sú vytvárané a spravované užívateľské modely. Viac informácií o koncepte užívateľských modelov je možné nájsť v podsekcii \ref{uzivatelske_modely}.

Hlavným komunikačným uzlom s touto oblasťou pre Model view vrstvu je singleton trieda \texttt{UserModelManager}. Ten ponúka kolekciu všetkých typov užívateľských modelov, ktoré je možné v aplikácii využiť. Mimo to implementuje metódy pre:
\begin{itemize}
    \item Serializáciu a deserializáciu užívateľských modelov do/z súborov.
    \item Vytváranie nových inštancií užívateľských modelov.
    \item Všeobecnú identifikáciu typov užívateľských modelov na základe rôznych vstupov.
\end{itemize}

\bigskip

Užívateľské modely sú v aplikácii reprezentované triedami, ktoré implementujú rozhranie \textbf{\texttt{IUserModel<out TTemplate>}}.  Prostredníctvom tohto rozhrania sú užívateľské modely používané mimo vrstvy Model. Z tohto dôvodu definuje valstnosti, ktoré sú využívané prevažne vo vonkajšom svete. K tomu definuje ešte aj kontrakty zaručujúce schopnosť užívateľského modelu serializovať sa. Každý užívateľský model je viazaný na konkrétny template-u typu \texttt{TTemplate}. K tomu si každý užívateľský model nesie aj referenciu na inštanciu template-u tohto typu.

Na to aby bol užívateľský model použiteľný vo vyhľadávacích algoritmoch, nestačí aby implementoval predošlé základné rozhranie. Je potrebné aby implementoval jeho následníka \textbf{\texttt{IComputingUserModel}}<out TTemplate, in TVertexAttributes, in TEdgeAttributes>. Tento interface sám o sebe nedrží žiadny kontrakt. Až jeho následníci definujú funkcionalitu, ktorú daný užívateľský model následne vie ponúknuť napríklad vyhľadávaciemu algoritmu. 

Jedným z týchto následníkov je interface \textbf{\texttt{IWeightComputingUserModel}}<out TTemplate, in TVertexAttributes, in TEdgeAttributes>. Toto rozhranie zaisťuje že užívateľský model bude schopný na základe dodaných vrcholových a hranových atribútov vypočítať váhu odpovedajúcej hrany. Túto funkcionalitu by mali spĺňať všetky užívateľské modely, nakoľko veľká časť vyhľadávacích algoritmov potrebuje poznať váhu jednotlivých hrán pre správny výber postupu.

Oproti výpočtovým rozhraniam tu existuje rozhranie \textbf{\texttt{ISettableUserModel}}. Toto rozhranie musia implementovať všetky typy užívateľských modelov, ktoré sú určené na to, aby v nich užívateľ mohol konfigurovať nastavitelné hodnoty (\textit{Adjustable}) vzhľadom na svoje preferencie. Kontrakt ktorý definuje zaručuje dodanie kolekcie týchto nastaviteľných hodnôt, aby mohli byť dodané užívateľovi a ten s nimi mohol pracovať. Mechanizmus vytvárania a nastavovania užívateľských modelov zatiaľ v aplikácii nie je implementovaný a teda toto rozhranie aktuálne čaká na svoje využitie.  

\bigskip

V aplikácii je potrebné, aby existencia jednotlivých typov užívateľských modelov bola nejakým spôsobom zastúpená. Týmito zástupcami sú triedy implementujúce následujúcu trojicu rozhrani:
\begin{itemize}
    \item \textbf{\texttt{IUserModelType<out TUserModel, out TTemplate>}} - Je určený pre komunikáciu mimo vrstvy Model. Definuje vlastnosti, ktoré sú potrebné pri práci so zástupcom užívateľského modelu vo vonkajšom prostredí. 
    
    Taktiež obsahuje referenciu na template, na ktorý je zastupovaný užívateľský model viazaný a definuje metódy slúžiace na deserializáciu a vytváranie nových užívateľských modelov. Deserializácia zástupcu a serializácia zastupovaného užívateľského modelu sa musia zhodovať. 
    
    Tento interface by nikdy nemal byť priamo implementovaný.
    \item \textbf{\texttt{IUserModelTemplateBond<in TTemplate>}} - Reprezentuje väzbu užívateľského modelu a template-u.  Vďaka kontravariantnej povahe jeho template-ového typového parametru bude táto identifikácia fungovať správne aj pre prípadných potomkov typu \texttt{TTemplate}. Tento interface by nikdy nemal byť priamo implementovaný.
    \item \textbf{\texttt{IUserModelRepresentative<TUserModel,TTemplate>}} - Zastupuje jeden konkrétny typ užívateľského modelu viazaného na jeden konkrétny typ template-u. Je potomkom predošlých dvoch interface-ov - spája ich funkcionality. Tento interface je určený k tomu, aby bol priamo implementovaný zástupcami užívateľských modelov.   
\end{itemize}

Na to, aby bolo možné typ užívateľský modelu využiť v aplikácii, musí byť jeho zástupca zahrnutý v príslušnej kolekcii manažéra užívateľských modelov. Z tohto dôvodu je na mieste, aby títo zástupcovia boli implementovaný ako singleton triedy.

\section{Vyhľadávacie algoritmy}

Táto oblasť zahrnuje mechanizmy spravujúce algoritmy pre vyhľadávanie ciest v mapových reprezentáciách. Viac informácií o vyhľadávacích algoritmoch samotných je možné nájsť v podsekcii \ref{vyhladavacie_algoritmy}. 

Hlavným komunikačným uzlom s touto oblasťou pre Model view vrstvu je singleton trieda \texttt{SearchingAlgorithmMan}. Táto trieda zverejňuje v kolekcii všetky použiteľné vyhľadávacie algoritmy aplikácie. Popri tom obsahuje metódy zabezpečujúce: 
\begin{itemize}
    \item spúšťanie procesu vyhľadávania cesty na dodanom užívateľskom modelu a mapovej reprezentácii
    \item dodanie \textit{executor}-u vyhľadávacieho algoritmu
    \item identifikáciu vyhľadávacích algoritmov spustiteľných pre konkrétne kombinácie mapových reprezentácií a užívateľských modelov 
\end{itemize}

\bigskip

Vyhľadávacie algoritmy sú v aplikácii reprezentované pomocou tried, ktoré implementujú rozhranie \textbf{\texttt{ISearchingAlgorithm}}. Každý algoritmus môže byť implementovaný niekoľkými spôsobmi. Rozhranie preto definuje kolekciu v ktorej by mali byť všetky použiteľné implementácie daného algoritmu zverejnené. Ďalej definuje a zároveň implementuje metódy, ktoré slúžia na:
\begin{itemize}
    \item testovanie, či zástupcovia typov mapovej reprezentácie a užívateľského modelu zastupujú použiteľnú kombináciu pre daný algoritmus. Teda či existuje implementácia algoritmu, pre ktorú sú vlastnosti zastupovaných typov dostatočné na použitie, 
    \item samotné spúšťanie vyhľadávacieho procesu. K tomuto účelu sa vyberie pre vstupné argumenty vhodná implementácia algoritmu,
    \item možnosť získania \textit{executor}-u daného algoritmu. Executor sa vytvorí na základe vhodnej implementácie algoritmu.
\end{itemize}
Inštancia každého použiteľného vyhľadávacieho algoritmu musí byť obsiahnutá v kolekcii vyhľadávacích algoritmov v príslušnom manažérovi. Inak nebude aplikácia daný vyhľadávací algoritmus registrovať. 

Implementácie vyhľadávacích algoritmov sú reprezentované triedami, ktoré implementujú rozhranie \textbf{\texttt{ISearchingAlgorithmImplementation}}. Toto rozhranie definuje podobné myšlienky funkcionalít tým z rozhrania \texttt{ISearchingAlgorithm}: testovanie typov vstupných mapových reprezentácií a užívateľských modelov, spúšťanie vyhľadávacieho procesu a vytváranie svojích executor-ov. V tomto prípade však nie je táto funkcionalita implementovaná rozhraním a je potrebné aby ju implementácie algoritmov doplnili sami. Na to aby implementácia algoritmu mohla byť použitá, musí byť obsiahnutá v kolekcii implementácií odpovedajúceho vyhľadávacieho algoritmu. 

Výsledkom vyhľadávania je inštancia triedy, ktorá implementuje rozhranie \textbf{\texttt{IPath<out TVertexAttributes, out TEdgeAttributes>}}. Toto rozhranie reprezentuje nájdenú cestu algoritmom pričom môže v sebe niesť atribúty typov \texttt{TVertexAttributes} a \texttt{TEdgeAttributes}. Neskôr v Model view vrstve je z tejto cesty vytvorený report, ktorý je následne vyššími vrstvami spracovaná a predvedený užívateľovi. Pre komunikáciu mimo Model vrstvy sa využíva jeho predchodca \texttt{IPath}. Toto rozhranie by nemalo byť nikdy priamo implementované. 

Vyhľadávacie algoritmy taktiež môžu počas svojho behu podávať reporty o stave vyhľadávania prostredníctvom objektov tried, ktoré implementujú rozhranie \textbf{\texttt{ISearchingState<out TVertexAttributes, out TEdgeAttributes>}}. Algoritmus nechá z týchto stavov agregovať report a podá ho ku následnému spracovaniu a predvedeniu užívateľovi.

Obidve vyššie spomenuté rozhrania taktiež podporujú návrhový vzor \textit{generic visitor pattern}. 

Pokiaľ je po algoritmu požadované vytvorenie jeho executor-u, algoritmus zavolá vhodnú svoju implementáciu nech executor vytvorí. Tá ho inicializuje pomocou dodanej mapovej reprezentácie, užívateľského modelu a pridá delegáta na svoju špecifickú metódu zabezpečujúcu beh algoritmu pre executor.

Všetky triedy reprezentujúce vyhľadávacie algoritmy a ich implementácie by mali byť implementované ako singleton triedy. Ich inštancie budú totiž vytvorené v aplikácii len jedny.

\section{Výškové dáta}

Oblasť pre správu a manipuláciu s výškovými dátami. Viac informácií o funkcii výškových dát v aplikácii je možné nájsť v podsekcii \ref{vyskove_data}. 

Hlavným komunikačným uzlom s touto oblasťou z vrstvy Model view je singleton trieda \texttt{ElevDataManager}. Tá ponúka kolekciu všetkých použiteľných zdrojov výškových dát v aplikácii. Popri tom doručuje metódy pre manipuláciu s výškovými dátami (sťahovanie a odstraňovanie) a taktiež metódy, ktoré testujú prítomnosť a vracajú prítomné výškové dáta odpovedajúce rozlohe dodanej mapy.

\bigskip

Výškové dáta sú v aplikácii reprezentované triedami, ktoré implementujú rozhranie \textbf{\texttt{IElevData}}. Toto jednoduché rozhranie definuje metódy, ktoré dokážu ku zadanej geografickej polohe vrátiť jej nadmorskú výšku. Inštancie týchto tried su na mieru vytvárané tak, aby dokázali dodať výškové dáta zodpovedajúce polohe a rozlohe konkrétnej mapy. 

Výškové dáta dodávajú jednotlivé zdroje. Tie sú v aplikácii reprezentované triedami, ktoré implementujú rozhranie \textbf{\texttt{IElevDataSource}}. Zdroje výškových dát sú zložené z viacerých distribúcií, ktoré následne už dodávajú potrebnú funkcionalitu pre prácu s nimi spravovanými, výškovými dátami. Preto samotné rozhranie \texttt{IElevDataSource} definuje iba kolekciu, v ktorej by mali byť uložené všetky distribúcie daného zdroja na to aby mohli byť v aplikácii použité. Popri tom definuje aj pár vlastností, ktoré sú využívané mimo vrstvy Model (napr. meno daného zdroja).

Jednotlivé distribúcie výškových dát sú v aplikácii reprezentované triedami, ktoré implementujú buď rozhranie \textbf{\texttt{IElevDataDistribution}}. Každá z týchto tried je zodpovedné za prácu s konkrétnou distribúciou výškových dát. Zabezpečujú sťahovanie, odstraňovanie a informovanie o ich prítomnosti. Je ponechané na zodpovednosti implementácií, akým spôsobom budú dáta ukladané, načítané do pamäte a spracovávané do inštancií implementácií rozhrania \texttt{IElevData}.

Rozhranie \texttt{IElevDataDistribution} by nemalo byť implementované priamo. Namiesto toho by mali byť implementované rozširujúce rozhrania \textbf{ICredentialsNotRequiringElevDataDistribution} a \textbf{ICredentialsRequiringElevDataDistribution} ktoré pridávajú samotnú metódu umožňujúcu sťahovanie výškových dát. Tieto metódy, resp. rozhrania, sa líšia v potrebe autorizácie pri získavaní výškových dát zo vzdialených zdrojov.  

Manipulovanie s výškovými dátami prebieha po takzvaných \textit{regiónoch}. Každá distribúcia si tvar a veľkosť svojich regiónov určuje sama. Následne tieto regióny sprostredkováva v kolekcii \texttt{AllTopRegions}, ktorá je definovaná rozhraním \texttt{IElevDataDistribution}.

Regióny sú reprezentované triedami, ktoré dedia od abstraktnej triedy \textbf{\texttt{Region}} a jej potomkov \textbf{\texttt{TopRegion}} a \textbf{\texttt{SubRegion}}. Región ako taký definuje svoje meno, svoj tvar za pomoci reťazca geografických súradníc a množinu svojich pod-regiónov. Taktiež definuje identifikátor svojej prítomnosti. Teda toho, čí sú jemu odpovedajúce výškové dáta stiahnuté v počítači a pripravené na použitie.

Tento indikátor by mal byť aktualizovaný vždy keď sa jeho prítomnosť úspešne zmení. Dôležité však je, že táto informácia by mala byť zachovaná naprieč behmi aplikácie. Teda ak sú v jednom behu aplikácie stiahnuté výškové dáta pre konkrétny región, v následujúcom behu by mal región indikovať, že sú jemu zodpovedajúce dáta stále k dispozícii.

Regióny sú naďalej delené na \textit{vrcholové regióny} a \textit{pod-regióny}. Pod-regióny sú vždy viazané na nejaký vyšší región a reprezentujú nejakú jeho časť. Vrcholový región potom jednoducho nie je nikoho pod-regiónom. Spomínaná kolekcia \texttt{AllTopRegions} obsahuje práve vrcholové regióny definované danou distribúciou.    

\section{Grafika}

Táto oblasť sa zaoberá problematikou vytvárania objektov, z ktorých sa skladajú grafické reprezentácie rôznych dátových štruktúr. Agregácia grafických objektov má špecifický asynchrónny princíp. Vytvorené grafické objekty sa postupne plnia do dodaného kolektoru. Tým pádom je možné vytvorené objekty spracovávať a zobraziť ihneď ako sú vytvorené. Aplikácia obsahuje hneď dva hlavné uzly pre komunikáciu s touto oblasťou:
\begin{itemize}
    \item \texttt{GraphicsManager} je hlavným uzlom pre komunikáciu prichádzajúcu z Model view vrstvy. Zabezpečuje agregáciu grafických objektov pre rôzne konštrukty aplikácie ako sú napríklad mapy.
    \item \texttt{GraphicsSubManager} slúži k rovnakému účelu ako \texttt{GraphicsManager}, avšak pre komunikáciu priamo z vrstvy Model. Reprezentuje prívetivejší spôsob komunikácie so zachovaním typových informácií. Konštrukty, pre ktoré zabezpečuje táto trieda agregáciu grafiky sú napríklad nájdené cesty a stavy vyhľadávacích algoritmov. 
\end{itemize}
Obidve tieto triedy sa riadia návrhovým vzorom singleton.

\bigskip

Grafické objekty sú v aplikácii reprezentované triedami, ktoré implementujú rozhranie \textbf{\texttt{IGraphicObject}}.Toto rozhranie neimplementuje takmer žiadnu funkcionalitu okre podpory návrhového vzoru \textit{Generic visitor pattern}.

Aby bolo možné grafiku dodanej mapy, cesty či stavu extrahovať, musí pre ňu existovať trieda implementujúca rozhranie \textbf{\texttt{IGraphicsAggregator}}, teda presnejšie jedného z jeho špecializovaných potomkov. Títo \textit{agregátori} následne dostanú daný konštrukt na spracovanie a kolektor, do ktorého sa majú naklásť vytvorené grafické objekty.

V prípade ciest a stavov vyhľadávacieho algoritmu dostane agregátor aj užívateľský model, ktorý môže využiť na výpočet niektorých hodnôt z vrcholových a hranových atribútov uložených v dodaných cestách/stavoch. Je však potrebné zdôrazniť, že užívateľský model možno nebude schopný tieto služby doručiť. V takom prípad sa musí agregátor zaobísť bez nich. Je možnosť, aby tieto vlastnosti vynucoval napríklad vyhľadávací algoritmus, ktorý pozná potreby pre extrahovanie grafiky ním používaného typu cesty či stavy vyhľadávania.

Bolo by namieste, keby každý užívateľský model dokázal z atribútov vyťažiť aspoň pozície vrcholov mapovej reprezentácie, aby bolo možné nakresliť aspoň základnú reprezentáciu nájdenej cesty či stavu. 

\bigskip

Nakoniec sú v tejto oblasti definované dve rozhrania ktoré slúžia pre reprezentáciu grafického zdroja. \textbf{\texttt{IGraphicsSource}} definuje jedinú vlastnosť a to zdrojovú kolekciu grafických objektov. Táto zdrojová kolekcia je typu \texttt{SourceList} patriaceho do framework-u \textit{Reactive UI}. Vo vyšších vrstvách MVVM(MV) architektúry je následne možné tento zdrojový zoznam sledovať a reagovať na jeho aktualizácie. 

Špecializácia tohto rozhrania \textbf{\texttt{IGroundGraphicsSource}} dodáva ešte nutnosť, aby daný grafický zdroj definoval svoju rozlohu. Triedy implementujúce toto rozhranie sú následne často využívané ako akési základné grafiky, ktorým sa ostatné prispôsobujú rozlohou.

Implementácie týchto rozhraní sí väčšinou vytvárané mimo tejto oblasti a dokonca mimo vrstvy Model. Sú špecializovane vytvorené na konkrétne využitie v aplikačnej logike. 

\section{Reportovanie}

Oblasti spravujúcej grafiku je architektúrou veľmi podobná oblasť vytvárajúca reporty na základe rôznych dátových štruktúr. V aktuálnej podobe aplikácie sú týmito štruktúrami nájdené cesty a stavy vyhľadávacích algoritmov. Táto oblasť veľmi často využíva služby triedy \texttt{GraphicsSubManager} pre získavanie grafiky, ktorá je pridaná do obsahu report-ov.

Táto Oblasť obsahuje opäť dva hlavné komunikačné uzly:
\begin{itemize}
    \item \texttt{ReportManager} je hlavným uzlom pre komunikáciu prichádzajúcu z Model view vrstvy. Zabezpečuje vytváranie report-ov napríklad pre nájdené cesty.
    \item \texttt{ReportSubManager} slúži k rovnakému účelu ako \texttt{ReportManager}, avšak pre komunikáciu priamo z vrstvy Model. Reprezentuje prívetivejší spôsob komunikácie so zachovaním typovej informácie. Zabezpečuje vytváranie reportov napríklad pre nájdené cesty a stavy vyhľadávaní.
\end{itemize}
Obidve tieto triedy sa riadia návrhovým vzorom singleton.

\bigskip

Pre cesty a stavy vyhľadávania sú reporty v aplikácii reprezentované triedami, ktoré implementujú rozhrania \textbf{\texttt{IPathReport}} a \textbf{\texttt{ISearchingReport}}. Tieto rozhrania nedefinujú takmer žiadnu funkcionalitu okrem podpory návrhového vzoru \textit{generic visitor pattern}.

Podobne ako v oblasti spravujúcej grafiku, na to, aby pre konkrétny typ dátovej štruktúry mohol byť report vytvorený, musí preňho existovať vhodná trieda implementujúca rozhranie \textbf{\texttt{IReportAggregator}}, resp. jedného z jeho špecializovaných potomkov. Títo \textit{agregátori} následne dostanú dátovú štruktúru na spracovanie a v niektorých prípadoch aj užívateľský model, ktorý môže využiť na výpočet niektorých hodnôt z vrcholových a hranových atribútov uložených v dodaných dátach. Podobne však ako u grafických agregátorov, nie je zaručené, že užívateľský model bude schopný požadované služby doručiť. 

\section{Parametre}

Posledná, jednoduchšia ale o to dôležitejšia, oblasť je využívaná na spravovanie a ukladanie všemožných parametrov aplikácie. Hlavným uzlom pre komunikáciu s touto oblasťou je singleton trieda \texttt{ParamsManager}. V tejto triede je možné uložiť od každého typu parametrov práve jednu inštanciu. Táto inštancia môže byť počas behu algoritmu rôzne menená. Inštancie parametrov sú uložené v slovníku pod ich vlastným typom.  

Keď je zavolaná metóda \texttt{SaveAllParams} tak sa táto trieda pokúsi všetky uložené parametre v slovníku serializovať pre možnosť ich použitia v budúcich behoch aplikácie.

Na druhej strane, deserializácia parametrov prebieha \uv{lenivým} spôsobom. Keď je požiadané o parametre typu, ktorý sa v slovníku nenachádza, tak sa najprv skúsi zistiť, či preňho neexistuje odpovedajúca serializácia. Ak áno, deserializuje sa inštancia daných parametrov, vloží do slovníku a vráti užívateľovi. Pokiaľ nie, poznačí sa do slovníku neexistencia takéhoto parametru a navráti sa hodnota null indikujúca neexistenciu parametrov daného typu.  

Serializácia a deserializácia parametrov do súbor, v ktorých prežívajú beh aplikácie, prebieha za pomoci singleton triedy \textbf{\texttt{DataSerializer}}. Táto trieda serializuje objekty do súborov pomocou systémovej triedy \textit{\texttt{JsonSerializer}}. Súbory sú pomenované podľa typu daného serializovaného objektu. To zanmená, že v jednu chvíľu pre každý dátový typ dokáže táto trieda serializovať jedinú inštanciu. Pri deserializácii na základe vstupného generického typového parametru nájde súbor s odpovedajúcim menom a pokúsi sa ho deserializovať do inštancie daného typu.


  