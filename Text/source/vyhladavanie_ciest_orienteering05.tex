\chapter{Vyhľadávanie trás v~mapách pre~orientačný beh}

Ako už bolo spomenuté v~sekcii~\ref{Aspekty_hladania}, vyhľadávanie trás v~mapách je výsledkom kombinácie viacerých konceptov. V~tejto kapitole sa bližšie pozrieme na~implementácie týchto konceptov, ktoré~sa spoločne podieľajú na~vyhľadávaní najrýchlejších trás v~mapách z~prostredia orientačného behu.

\section{Orienteering (ISOM 2017-2) template}

Tento template definuje vrcholové a~hranové atribúty, ktorými dokážeme detailne popísať miesto na~mape využívanej v~orientačnom behu. Je potrebné, aby dokázali zachytiť topológiu mapy, prostredie, v~ktorom sa daný vrchol/hrana nachádza a nadmorskú výšku konkrétneho bodu na~mape.  

Čo sa mapových značiek týka, využíva sa sada mapových značiek definovaná normou ISOM 2017-2. \uv{Cieľom Medzinárodnej špecifikácie máp pre~orientačné športy (ISOM) je poskytnúť mapovú špecifikáciu, ktorú možno používať na~celom svete pre~rôzne typy terénu, ktoré~sú vhodné pre~orientačné športy. Táto špecifikácia by sa mala používať v~spojení s~pravidlami Medzinárodnej federácie orientačných športov (IOF) pre~orientačné preteky.}\cite{ISOM20172}

Vrcholové atribúty samotné nesú informáciu o~polohe vrcholov grafu na~mape a~indikujú nadmorskú výšku v~mieste daného vrcholu. V~aktuálnom stave aplikácie nie je možné získať informáciu o~nadmorskej výške vrcholov grafu a~teda je nadmorská výška všetkých vrcholov v grafoch nastavená na~hodnotu 0. To znamená, že pri~výbere najrýchlejších trás v~mape sa nebude brať do~úvahy informácia o~prevýšení, ktoré~musí bežec na~danej ceste prekonať.

Informáciu o~okolitom teréne a~prostredí následne nesú hranové atribúty, ktoré~popisujú okolie jednotlivých hrán grafu. Okolím hrany sa myslí prostredie na~ľavej a~pravej strane konkrétnej hrany (les, vegetácia, lúka, vodné teleso, mokraď, atď) a~prípadná cesta alebo líniová prekážka, ktorá cez danú hranu prechádza.

Okolité prostredie hrany je reprezentované kombináciou indikátorov plošných mapových značiek (rastrov), ktorá zodpovedá korektnej kombinácii taktiež definovanej normou ISOM 2017-2. Pre~ľavé a~pravé prostredie hrany sú v~hranových atribútoch vytvorené dve premenné, v~ktorých je možné uložiť akúkoľvek korektnú kombináciu týchto indikátorov. Korektnosť kombinácie indikátorov si musí užívateľ týchto atribútov zabezpečiť sám.  

Pre indikovanie toho, že hrana zodpovedá ceste alebo toho, že cez hranu prechádza líniová prekážka, v~hranových atribútoch slúži jediná premenná, v~ktorej je možné uložiť informáciu o~kombinácii jednej cesty, jednej človekom vytvorenej líniovej prekážky a~jednej \uv{prírodnej} líniovej prekážky. 

\section{Complete, net intertwining map representation}

Pre grafovú reprezentáciu máp pre~orientačný beh bola vytvorená takzvaná \textit{sieť-prepletajúca} mapová reprezentácia. Tvorba orientovaného ohodnoteného grafu tejto mapovej reprezentácie spočíva vo vytvorení siete z~vrcholov a~hrán vo veľkosti dodanej mapy, a~následného korektného postupného vkladania jednotlivých objektov z~mapovej reprezentácie do~tejto vytvorenej siete. Táto forma tvorby grafu je vhodná pre~mapy, ktoré~vznikli z~mapových formátov, ktoré~reprezentujú mapu pomocou množiny geometrických útvarov, ktoré~popisujú tvary jednotlivých objektov na~mape.\\
Práve takým formátom je napríklad \textit{OMAP}, ktorý je jedným z~formátov využívaných pre~reprezentáciu máp orientačného behu a~je možné ho v~aplikácii spracovať do~podoby z~ktorej sa bude následne dať vytvoriť sieť-prepletajúca mapová reprezentácia. 

V následujúcej pod-sekcii popíšeme proces, ktorým je mapa  získaná zo súboru formátu OMAP postupne spracovávaná do~grafovej podoby.

\subsection{Postup spracovania mapy na~jej grafovú reprezentáciu}

Na začiatku sa vytvorí spomínaná sieť vrcholov, do~ktorej sa budú následne ukladať jednotlivé objekty mapy. Dĺžka hrán v~tejto sieti je daná konfiguračným parametrom \uv{štandardná dĺžka hrany}. Na~Obrázku \ref{obr20:just_net} je možné nahliadnuť vytvorenú sieť nad testovacou mapou.

Následne sa začnú spracovávať jednotlivé mapové objekty a~to v~následujúcom poradí: polygony, cesty a~líniové prekážky.
Objekty sa postupne vkladajú do~vytváraného grafu. Popri tom sa nastavujú atribúty na~jednotlivých hranách grafu tak, aby správne zodpovedali značkám vkladaných objektov.

Pridávanie objektov vyzerá následovne:

\begin{itemize}
    \item V~prvom rade sa vytvorí retiazka vrcholov spojených hranami odpovedajúca celému obvodu polygonu. Táto vytvorená retiazka však môže na~viacerých miestach krížiť samú seba a~teda je potrebné aby sa všetky tieto kríženia odstránili. Kríženia odstránime správnymi výmenami križujúcich sa hrán. Z~tohto procesu nám môže vyjsť viac disjunktných retiazok, ktoré~si následne pozberáme a~budeme ich naďalej spracovávať separátne. Taktiež nesmieme kvôli následnému spracovávaniu zabudnúť skontrolovať, či sú vytvorené retiazky \uv{pravotočivé}, teda či pravá strana každej hrany je vnútorná. Pokiaľ tomu u nejakej vytvorenej retiazky nie je, otočíme list retiazkových vrcholov a~tým dostaneme pravotočivú retiazku. Táto vlastnosť bude dôležitá v následujúcom spracovaní.
    \item V~druhom kroku musíme prerušiť všetky hrany, ktoré~vytvorená retiazka prerušila. Pri~identifikovaní prerušených hrán sa využívajú techniky z~geometrie a~lineárnej algebry. Takýto graf s~prerušenými hranami je možné nahliadnuť v~Obrázku \ref{obr21:cut_edges_by_polygon_chain}. Pre~ďalšie spracovanie sa zapamätajú všetky vrcholy, ktorým bola prerušená nejaká hrana.

\begin{figure}[p]\centering
\frame{\includegraphics[width=250px]{img/mapRepre/justNet0.jpg}}
\caption{Sieť grafu zhotovená nad vstupnou mapou. Červené čiary predstavujú grafové hrany.} 
\label{obr20:just_net}
\end{figure}

\begin{figure}[p]\centering
\frame{\includegraphics[width=250px]{img/mapRepre/cutEdgesByPolygonChain1.jpg}}
\caption{Retiazka vrcholov (objektu zakázaného priestoru) prerušujúca hrany vytváraného grafu. } 
\label{obr21:cut_edges_by_polygon_chain}
\end{figure}

    \item V~ďalšom kroku sa pomocou prehľadávania do~hĺbky identifikujú všetky vrcholy, ktoré~spadajú do~vnútra polygonu. Tento spôsob je založený na~tom, že všetky hrany spojujúce vnútorné vrcholy polygonu s~vrcholmi mimo polygonu boli v~predchádzajúcom kroku odstránené a~teda DFS neutečie mimo polygonu. Zároveň je zaručené, že sa nájdu všetky vnútorné vrcholy polygonu a~to tak, že sa sleduje, či boli nájdené všetky vrcholy, ktorým bola prerušená hrana retiazkou. Pokiaľ DFS nenavštívi všetky tieto vrcholy, spúšťa sa znova inicializovaná jedným z~takýchto vrcholov. Hranám týchto vnútorných vrcholov sa upravia atribúty na~základe mapovej značky ktorej objekt sa aktuálne spracováva ako je možné nahliadnuť v~Obrázku \ref{obr22:inner_edges_attributes_set}.
    \item V~posledných krokoch sa retiazka vrcholov pripojí na~vrcholy, ktorým bola prerušená aspoň jedna hrana a~správne sa na~týchto hranách nastavia ích atribúty. Následne sa taktiež nastavia atribúty na~hranách samotnej retiazky a~retiazka je vložená do~grafu. Toto nastavenie atribútov je korektné, teda vytvára iba normou ISOM 2017-2 povolené kombinácie mapových značiek. Proces nastavovania komentárov je zložitý a~vysvetľovaný detailnejšie priamo v~komentároch programu. Využíva sa pri~ňom pravotočivosť retiazok zabezpečená v~prvom kroku vytvorenia retiazky. Kompletne pridaný objekt do~grafu je možné nahliadnuť v~Obrázku \ref{obr23:connect_polygon_chain_to_vertices_of_cut_edges}
\end{itemize}

Aktuálne vytváraná mapová reprezentácia nedokáže zachytiť rôzne rýchlosti behu v~mapových značkách ako sú \textit{vegetácie s~lepšou rýchlosťou behu v~jednom smere} alebo \textit{vinohrady a~podobné kultúry}. Pre~správne zabezpečenie určenie rýchlostí v~týchto značkách by musela byť v~procese prítomná zložitá procedúra na~vytvorenie správne orientovanej siete vrcholov, v~ktorej by sa táto vlastnosť spomínaných značiek dala vyjadriť. V~tejto chvíli sú rýchlosti behu v~takýchto mapových značkách brané uniformne. 

Keď sú všetky polygonálne objekty pridané, prichádzajú na~rad cesty. Tie sú pridávané veľmi podobným spôsobom ako polygony, avšak je upustené od~kroku, v~ktorom sa hľadajú vnútorné vrcholy polygonu a~nastavujú sa ich hranám atribúty. Tento krok u ciest nie je potrebný, nakoľko pri~pridávaní ciest sa nemení prostredie, v~ktorom sa daná cesta nachádza.\\ 
Po pripojení cesty do~grafu sa nastavia na~hranách retiazky cesty atribúty indikujúce existenciu cesty na~danej hrane. Na~Obrázku \ref{obr24:connect_path_chain_to_vertices_of_cut_edges} je možné nahliadnuť plne pridanú cestu do~grafu

V poslednom rade sa postupne spracujú všetky líniové prekážky, ktoré~sú schopné spomaliť alebo dokonca znemožniť bežcov postup. V~tomto prípade sa nebudú ako v~predošlých prípadoch pridávať retiazky týchto objektov do~grafu. Postup spracovania týchto objektov je následovný:
\begin{itemize}
    \item V~prvom rade sa opäť vytvorí retiazka vrcholov pospájaných hranami pozdĺž celej dĺžky línie objektu.
    \item V~ďalšom kroku sa nájdu všetky hrany, ktoré~táto retiazka kríži, avšak tentokrát sa tieto hrany neodstránia.
    \item Nakoniec sa iba upravia týmto hranám atribúty tak, aby indikovali, že cez ne prechádza líniová prekážka.
\end{itemize}

Výsledok vloženia líniovej prekážky do~grafu je možné nahliadnuť v~Obr. \ref{obr25:added_linear_obstacle}

\begin{figure}[p]\centering
\frame{\includegraphics[width=250px]{img/mapRepre/innerEdgesAttributesSet2.jpg}}
\caption{Nastavenie atribútov hrán vrcholov vnútri polygonu(objektu zakázaného priestoru). Atribúty sú na~hranách vyznačené farebnými škvrnami vo farbách reprezentovaných mapových značiek.} 
\label{obr22:inner_edges_attributes_set}
\end{figure}

\begin{figure}[p]\centering
\frame{\includegraphics[width=250px]{img/mapRepre/connectPolygonChainToVerticesOfCutEdges3.jpg}}
\caption{Pridanie hrán medzi retiazkou a~vrcholmi, ktorým bola preseknutá aspoň jedna hrana a~nastavenie atribútov na~týchto hranách ako aj na~hranách retiazky. Atribúty sú na~hranách vyznačené farebnými škvrnami vo farbách reprezentovaných mapových značiek.} 
\label{obr23:connect_polygon_chain_to_vertices_of_cut_edges}
\end{figure}

\begin{figure}[p]\centering
\frame{\includegraphics[width=250px]{img/mapRepre/connectPathChainToVerticesOfCutEdges5.jpg}}
\caption{Pridaná retiazka cesty do~grafu. Na~hranách pridanej retiazky je možné vidieť malé čierne šmuhy indikujúce atribúty cesty na~daných hranách.} 
\label{obr24:connect_path_chain_to_vertices_of_cut_edges}
\end{figure}

\begin{figure}[p]\centering
\frame{\includegraphics[width=250px]{img/mapRepre/addedLinearObstacle6.jpg}}
\caption{Pridaná líniová prekážka (objektu neprekonateľného zrázu) do~grafu. Je možné si všimnúť na~hranách, ktoré~táto prekážka kríži, malé čierne šmuhy, ktoré~indikujú, že sa na~danej hrane nachádza atribút neprekonateľného zrázu.} 
\label{obr25:added_linear_obstacle}
\end{figure}

\pagebreak

\section{Orienteering (ISOM 2017-2) user model}

Užívateľský model naviazaný na~template \textit{Orienteering (ISOM 2017-2)}. Umožňuje nastaviť veľkosť spomalenia, ktoré~jednotlivé mapové značky pre~bežca vytvárajú. Na~základe tejto parametrizácie následne dokáže vypočítať čas, ktorý bežcovi zaberie prekonanie konkrétnej hrany grafu.

Príklad, ako môže vyzerať parametrizácia tohto užívateľského modelu je možné nahliadnuť v~Prílohe~\ref{uzivatelsky_model}. Táto parametrizácia vznikla v~spolupráci s~aktívnym reprezentantom Slovenskej republiky v~orientačnom behu.

\subsection{Princíp výpočtu váhy prechodu hrany z~poskytnutých atribútov}

Pri výpočte váhy konkrétnej hrany dostane užívateľský model na~vstupe atribúty počiatočného a~koncového vrcholu tejto hrany a~atribúty hrany samotnej.

Výpočet prebieha následovne:
\begin{itemize}
    \item V~prvom rade sa na~základe atribútov hrany a~koeficientov spomalenia spočítajú koeficienty spomalenia ľavého a~pravého prostredia hrany. Koeficient spomalenia je číslo medzi 0 a~1 udávajúce pomer medzi najväčšou možnou rýchlosťou behu a~spomalenou rýchlosťou behu daným symbolom na~mape. Prostredia hrany môžu byť kombináciou viacerých mapových značiek. Užívateľský model preto vynásobí koeficienty spomalenia všetkých značiek, ktoré~sa v~ľavom/pravom prostredí nachádzajú a~vráti tento produkt ako výsledný koeficient. Výsledné koeficienty spomalenia ľavého a~pravého prostredia sa následne porovnajú a~vyberie sa ten väčší z nich. Teda bežec si vyberie pre~beh po~hrane to prostredie, ktoré~je preňho efektívnejšie. 
    \item V~ďalšom kroku sa započíta efekt cesty na~hrane. Pokiaľ je na danej hrane cesta, potom efekt prostredia nehrá pre~beh žiadnu roľu a~riadime sa iba podľa koeficientu spomalenia danej cesty.\\
    Výnimkou v~tomto pravidle sú mapové značky pre \textit{menej výrazné malé chodníky} a~\textit{prieseky}. Pri~týchto symboloch priebežnosť závisí od~prostredia, v~ktorom sa nachádzajú. Preto podobne ako pri~výpočte kombinácií značiek prostredia hrany sa ich koeficient vynásobí s~vypočítaným koeficientom prostredia a~tým sa získa koeficient spomalenia týchto objektov. Podotýkame, že koeficienty týchto objektov môžu (a mali by) byť väčšie ako 1 pre~zabezpečenie vyššieho koeficientu spomalenia po~danom chodníku/prieseku než je koeficient v~samotnom prostredí. 
    \item Pokiaľ je výsledný koeficient menší než 0.15, zoberie sa do~následujúcich krokov táto hodnota namiesto vypočítaného koeficientu. Týmto spôsobom predídeme príliš malým koeficientom spomalenia. Táto technika by sa dala nahradiť vážením koeficientu podľa toho, koľko mapových značiek sa podielalo na~jeho výpočte.
    \item V~treťom kroku dopočítame čas, ktorý je potrebný na~prekonanie danej hrany. 
    \begin{itemize}
        \item Najprv zo vstupných vrcholových atribútov dopočítame dĺžku hrany medzi nimi. 
        \item Následne aplikujeme vypočítaný koeficient z~predošlých krokov na~konštantu predstavujúcu najrýchlejšie tempo, ktoré~bežec dokáže dosiahnuť. Teda túto konštantu predelíme vypočítaným koeficientom, čím dostaneme tempo, ktorým bežec pobeží v~prostredí danom hranovými atribútmi. 
        \item Nakoniec vynásobíme tempo so vzdialenosťou, ktorú musí bežec prechodom cez~hranu prekonať, a~získame čas na~prechod tejto hrany.  
    \end{itemize}
    \item V~poslednom kroku ešte musíme skontrolovať, či cez danú hranu neprechádza nejaká líniová prekážka. Pokiaľ áno, je potrebné ku času pripočítať čas, ktorý zaberie prekonanie tejto prekážky. 
    \item Výsledný čas sa vráti ako váha hrany.
    \end{itemize}

Poznámka:\\
V tejto chvíli užívateľský model pri~výpočtu doby na~prekonanie konkrétnej hrany nezapočítava faktor prevýšenia, ktoré~musí bežec pri~prechode danou hranou prekonať. Tento faktor v~aktuálnej podobe aplikácie nie je potrebné zahrňovať, nakoľko grafová reprezentácia máp nenesie žiadnu informáciu nadmorskej výške jednotlivých vrcholov a~teda nie je možné prekonané prevýšenie dopočítať. Až po~pridaní tejto funkcionality do~grafovej reprezentácie bude potrebné upraviť tento užívateľský model tak, aby dokázal prevýšenie vo svojom výpočte zohľadniť.  

\subsection{Výpočet A* heuristiky}\label{vypocet_a_star_heuristiky}

Užívateľský model je popri výpočtu váh hrán z~ích atribútov zodpovedný taktiež za výpočet heuristickej funkcie pre~A* algoritmus. Dôvodom, prečo práve užívateľský model je zodpovedný za výpočet tejto heuristiky je ten, že heuristika nezávisí iba na~samotnej topológii grafu, ale~taktiež na~rýchlosti, ktorú~dokáže bežec v~závislosti na~parametrizácii užívateľského modelu dosiahnuť na~jednotlivých typoch povrchov / mapových značkách.

Heuristika počítaná užívateľským modelom musí byť \textit{prípustná} a~\textit{monotónna} aby ju bolo možné použiť v~implementovanom A* algoritme. Vysvetlenie týchto dvoch vlastností heuristiky je možné nahliadnuť v~spodnej časti Podsekcie \ref{klasicky_a_star_algoritmus}. 

Pre dosiahnutie prípustnosti heuristickej funkcie potrebujeme, aby funkcia vždy vracala váhu menšiu ako je skutočná cena cesty z~daného vrcholu do~cieľového. Nakoľko váha hrán je počítaná v~čase na~ich prekonanie, najnižšia možná váha cesty z~daného vrcholu do~cieľového je čas, za ktorý bežec prebehne vzdialenosť medzi týmito dvomi vrcholmi po~povrchu, po~ktorom sa mu pobeží čo najlepšie. Nastavíme preto heuristickú funkciu na~hodnotu vypočítanú práve týmto spôsobom. Tým zaručíme, že hodnota heuristiky nikdy nebude vyššia než hodnota skutočnej ceny cesty a~teda zabezpečíme jej prípustnosť.

Vyššie popísaná heuristika bude taktiež monotónna, nakoľko čas, za ktorý sa dostaneme z~vrcholu priamo do~cieľa po~ideálnom povrchu bude určite menší alebo rovný času, ktorý by nám zabralo prejsť do~iného miesta na~mape po~akomkoľvek povrchu a~následne z~daného miesta po~ideálnom povrchu prísť do~cieľa.  

Výpočet heuristickej funkcie užívateľským modelom teda spočíva v:
\begin{itemize}
    \item nájdení povrchu, po~ktorom sa bežcovi beží najlepšie
    \item vypočítaní času, ktorý bežec bude potrebovať na~zdolanie vzdialenosti medzi daným vrcholom a~cieľom tempom, ktoré~dokáže dosiahnuť na~najideálnejšom povrchu 
\end{itemize}

\section{A* searching algorithm}

Algoritmom výberu pre~vyhľadávanie najrýchlejších ciest v~mapách je algoritmus A*. Tento algoritmus je klasickou prvou volbou pri~úlohách vyhľadávania ciest v~ohodnotených grafoch s~topologickými vlastnosťami. V~našom prípade nepracujeme s~grafom, ktorého váhy na~hranách na~100\% odpovedajú vzdialenostiam medzi jednotlivými vrcholmi, ale za pomoci jednoduchých myšlienok popísaných v~Podsekcii \ref{vypocet_a_star_heuristiky} dokážeme tento problém vyriešiť. 

V následujúcich podsekciách popíšeme najprv klasický pohľad na~algoritmus A* a~potom sa pozrieme na~jeho úpravy, vďaka ktorým dokáže pracovať nad naším grafom s~údajmi od~užívateľského modelu.

\subsection{Klasický A* algoritmus}\label{klasicky_a_star_algoritmus}

Články~\cite{IntrodutionToAStar} and \cite{Heuristics} popisujú A* algoritmus následovne:

A* alogritmus je nepriamou kombináciou Dijkstrovho algoritmu pre~nachádzania najkratších trás v~ohodnotených grafoch a~Greedy Best-First-Search algoritmu využívajúceho metódu na~usmernenie svojho prehľadávania.

Tajomstvo jeho úspechu spočíva v~tom, že kombinuje informácie, ktoré~používa Dijkstrov algoritmus (uprednostňuje vrcholy, ktoré~sú blízko počiatočného bodu) a~informácie, ktoré~používa Greedy Best-First-Search (uprednostňuje vrcholy, ktoré~sú blízko cieľa).
V štandardnej terminológii používané pri~hovorení o~A* znamená g(n) presné náklady cesty z~počiatočného bodu do~akéhokoľvek vrcholu n, a~h(n) predstavuje heuristický odhad nákladov z~vrcholu n do~cieľa.  A* vyváži tieto dve hodnoty, keď sa pohybuje z~počiatočného bodu do~cieľa. Každý raz v~hlavnej slučke skúma vrchol n, ktorý má najnižšiu hodnotu f(n) = g(n) + h(n).

\bigskip

Je dôležité vybrať korektnú heuristickú funkciu, aby A* hľadal najkratšie cesty v~ohodnotených grafoch. Pri~výbere heuristiky môžu nastať následujúce prípady:
\begin{itemize}
    \item Pokiaľ je h(n) = 0 pre~všetky n, potom vo vyhľadávaní zohráva roľu iba g(n) a~A* sa zmen9 na~Dijkstrov algoritmus.
    \item Pokiaľ je h(n) nižšie (alebo rovné) ceny presunu z~n do~cieľa, potom A* zaručene nájde najkratšiu cestu v~grafe. Čím je h(n) nižšie, tým viac vrcholov grafu A* prehľadá, čo spôsobí jeho spomalenie.
    \item Pokiaľ je h(n) presne rovné cene presunu z~n do~cieľa, A* bude sledovať práve tú najlepšiu cestu a~neprehľadá žiadny vrchol navyše, čo ho robí veľmi efektívnym. Avšak nie vo všetkých prípadoch je možné takúto presnosť heuristiky dosiahnuť.
    \item Pokiaľ je h(n) niekedy vyššie ako je cena presunu z~n do~cieľa, A* nezaručuje nájdenie najkratšej cesty v~grafe ale dokáže bežať rýchlejšie.
    \item Pokiaľ je však h(n) príliš vysoké v~porovnaní s~g(n), potom začne zohrávať úlohu iba váha h(n) a~A* sa stáva Greedy Best-First-Search algoritmom.
\end{itemize}

Teda pokiaľ vyžadujeme nachádzanie najkratších trás v~ohodnotenom grafe, potrebujeme, aby heuristická funkcia bola z~hora ohraničená skutočnou cenou cesty z~n do~cieľa. O~takto obmedzenej heuristike následne hovoríme, že je \textit{prípustná}.

Pokiaľ však využívame \textit{grafové prehľadávanie}, pre~zabezpečenie nájdenia najkratšej cesty v~grafe je potrebné aby heuristika bola taktiež \textit{monotónna}. Jedná sa o~istú formu trojuholníkovej nerovnosti kedy hodnota heuristiky vo vrchole n musí byť menšia alebo rovná súčtu ceny prechodu z~n do~jeho následníka a~hodnoty heuristiky v~tomto následníkovi. Táto trojuholníková nerovnosť musí platiť vo všetkých vrcholoch a~ich následníkoch. Dá sa dokázať, že každá monotónna heuristika je taktiež prípustná.

\bigskip

V Programe~\ref{AStar} je ku nahliadnutiu pseudokód možnej implementácie algoritmu A*. Tento pseudokód bol prevzatý z~Článku~\cite{AStarWiki}.

\begin{listing}
\begin{lstlisting}

function A_Star(start, goal, h)
  openSet := {start}

  cameFrom := an empty map

  gScore := map with default value of Infinity
  gScore[start] := 0

  fScore := map with default value of Infinity
  fScore[start] := h(start)

  while openSet is not empty
    current := pop vertex from openSet with lowest fScore value
    if current = goal
      return reconstruct_path(cameFrom, current)

    openSet.Remove(current)
    for each neighbor of current
      tentative_gScore := gScore[current] + d(current, neighbor)
      if tentative_gScore < gScore[neighbor]
        cameFrom[neighbor] := current
        gScore[neighbor] := tentative_gScore
        fScore[neighbor] := tentative_gScore + h(neighbor)
        if neighbor not in openSet
          openSet.add(neighbor)
  return failure

function reconstruct_path(cameFrom, current)
    total_path := {current}
    while current in cameFrom.Keys:
        current := cameFrom[current]
        total_path.prepend(current)
    return total_path

\end{lstlisting}
\caption{A* algoritmus}
\label{AStar}
\end{listing}

\pagebreak

\subsection{Špecifické vlastnosti algoritmu A* v~aplikácii.}

Implementácia A* v~aplikácii nesie následujúce špecifiká:

\begin{itemize}
    \item Algoritmus na~vstupe namiesto heuristickej funkcie dostane inštanciu užívateľského modelu, ktorý bude hodnotu heuristickej funkcie pre~algoritmus počítať. Zároveň tento užívateľský model je zodpovedný za počítanie váh pre~hrany grafu, nakoľko grafová reprezentácia, nad~ktorou~algoritmus pracuje, nesie iba informácie o~vrcholových a~hranových atribútoch, je potrebné, aby užívateľský model tieto atribúty spracoval do~váh, s~ktorými následne už A* dokáže pracovať.   
    \item A* využívaný v~aplikácii je implementovaný ako graph search. Z~tohto dôvodu je potrebné, aby využívaná heuristika počítaná užívateľským modelom bola okrem \textit{prípustnosti} taktiež \textit{monotónna}.
    \item Naša implementácia A* si neudržiava kolekciu predchodcov navštívených vrcholov a~teda z~dôvodu možnosti rekonštrukcie nájdenej cesty vyžaduje po~vrcholoch vstupného grafu, aby boli schopné si zapamätať svojho predchodcu. Keď je A* narazí na~cieľový vrchol, začne sa dotazovať postupne vrcholov na~ich predchodcov dokým nenarazí na~počiatočný vrchol. Po~prevrátení postupnosti predchodcov vrcholov získa retiazku reprezentujúcu nájdenú cestu z~počiatku do~cieľa. 
\end{itemize}
