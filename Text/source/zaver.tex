\chapter*{Závěr}
\addcontentsline{toc}{chapter}{Závěr}

Cielom tejto bakalárskej práce bolo vytvoriť aplikáciu, v~ktorej bude možné vyhľadávať cesty v~otvorenom teréne na~základe dodaného mapového súboru. Vyhľadávanie ciest malo byť konkrétne implementované pre~mapy z~prostredia športového odvetvia \textit{orientačný beh}. 

Práca mala zahrňovať návrh architektúry aplikácie a~jej samotnú implementáciu, špecifickú implementáciu na~spracovávanie mapových súborov z~oblasti orientačného behu, vytvorenie vyhľadávacieho algoritmu, ktorý~by dokázal vyhľadávať na~vytvorenej mapovej reprezentácii a~možnosť užívateľa zahrnúť svoje preferencie do~procesu vyhľadávania. 

Tieto úlohy boli úspešne naplnené. 
\begin{itemize}
    \item Bol vytvorený kvalitný návrh aplikácie na~základe MVVM(MV) návrhového vzoru. Na~jeho základe bol následne implementovaný mechanizmus zabezpečujúci logiku za~vyhľadávaním ciest v~mapách a~taktiež mechanizmus hlavného okna, v~ktorom je možné spravovať nastavenia určené pre~celú aplikáciu. Následne bolo vytvorených 9 oblastí slúžiacich ku získavaniu a~spracovávaniu dátových štruktúr. Celý program je navrhnutý takým spôsobom, aby~bol čo najľahšie rozšíriteľný, či~už v~oblasti aplikačnej logiky alebo~typov dátových štruktúr, s~ktorými dokáže aplikácia pracovať. Vďaka MVVM(MV) architektúre je aplikáciu jednoduché udržovať a~modifikovať.
    \item Pre vyhĽadávanie ciest v mapách pre orientačný beh bol vytvorený
    \begin{itemize}
        \item parser pre mapy formátu OMAP a grafické znázornenie týchto máp,
        \item template zahrňujúci mapové značky dôležité pri vyhľadávaní najrýchlejších ciest v mapách pre orientačný beh,
        \item užívateľský model, ktorý je možné parametrizovať na základe preferencií užívateľa
        \item grafová reprezentácia mapy pre orientačný beh, na ktorej sa bude vyhľadávať najrýchlejšia cesta. V tejto chvíli bohužiaľ bez spracovania a aplikovania výškových dát. 
        \item vyhľadávací algoritmus A*, ktorý je používaný pre vyhľadávanie trás v grafových reprezentáciách máp
    \end{itemize}
\end{itemize}

Vyhľadávanie samotné dodáva uspokojujúce výsledky. Hlavným problémom, ktorý nebol doposiaľ vyriešený je zakomponovanie informácií o nadmorskej výške do grafovej reprezentácie mapy a spracovanie tejto informácie užívateľským modelom. Preto je vyhľadávanie trás v tejto chvíli možné iba na relatívne plochých mapách, kde dokáže vracať zmysluplné výsledky.\\ 
O výsledkoch vyhľadávania by bolo vhodné spraviť exaktnejší výskum, založený na názoroch skúsených orientačných bežcov, avšak z časových dôvodov nebolo možné takýto výskum uskutočniť. Samotné uskutočnenie by taktiež bolo celkom zložité, nakoľko kvalita nájdených trás v mnohých ohľadoch závisí na subjektívnom pohľade jednotlivých bežcov. Z tohto dôvodu sú v aplikácii prítomné užívateľské modely, aby si každý užívateľ mohol vyhľadávanie nastaviť podľa vlastných preferencií.
