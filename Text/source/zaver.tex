\chapter*{Závěr}
\addcontentsline{toc}{chapter}{Závěr}

Cielom tejto bakalárskej práce bolo vytvoriť aplikáciu, v ktorej bude možné vyhľadávať cesty v otvorenom teréne na základe dodaného mapového súboru. Vyhľadávanie ciest malo byť konkrétne implementované pre mapy z prostredia športového odvetvia \textit{orientačný beh}. 

Práca mala zahrňovať návrh architektúry aplikácie a jej samotnú implementáciu, špecifickú implementáciu na spracovávanie mapových súborov z oblasti orientačného behu, vytvorenie vyhľadávacieho algoritmu, ktorý by dokázal vyhľadávať na vytvorenej mapovej reprezentácii a možnosť užívateľa zahrnúť svoje preferencie do procesu vyhľadávania. 

Tieto úlohy boli čiastočne splnené. Bol vytvorený kvalitný návrh aplikácie na základe MVVM(MV) návrhového vzoru. Na jeho základe bol následne implementovaný mechanizmus zabezpečujúci logiku za vyhľadávaním ciest v mapách a taktiež mechanizmus hlavného okna, v ktorom je možné spravovať nastavenia určené pre celú aplikáciu. Následne bolo vytvorených 9 oblastí slúžiacich ku získavaniu a spracovávaniu dátových štruktúr. Celý program je navrhnutý takým spôsobom, aby bol čo najľahšie rozšíriteľný, či už v oblasti aplikačnej logiky alebo typov dátových štruktúr, s ktorými dokáže aplikácia pracovať. 

Na druhej strane však nezvýšil čas na vytvorenie konkrétnej implementácie spracovávania máp z oblasti orientačného behu, vytvorenie vyhľadávacieho algoritmu a ani mechanizmu ktorým by užívateľ mohol zahrnúť svoje preferencie do procesu vyhľadávania. Konečným výsledkom je teda kvalitne vytvorená kostra ktorej chýba funkčné vnútro.  
