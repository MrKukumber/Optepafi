\chapter{Architektúra ako celok}

O architektúre aplikácie sa dá premýšlať ako o \textit{mriežke}. Je rozdelená na horizontálne (MVVM návrhový vzor) a vertikálne (\textit{session}-y + hlavné okno) vrstvy. V následujúcich sekciách popíšeme, ako jednotlivé vrstvy vyzerajú, aké su ich úlohy a ako medzi sebou komunikujú. Na konci kapitoly je následne k nahliadnutiu diagram \ref{obr02:priklad_struktury} znázorňujúci príklad možnej architektúry aplikácie.   

\section{MVVM(MV) návrhový vzor}\label{MVVMNavrhovyVzor}

Ako už bolo spomenuté v podsekcii \ref{ArchitekturaMVVM}, v aplikácii je využívaný návrhový vzor MVVM s drobnou obmenou. Táto obmena sa týka rozdelenia originálnej vrstvy view model na dve časti: view model a model view. Toto rozdelenie zaručí ešte o niečo lepšiu separáciu kódu a odľahčí tým úlohy view model vrstvy. 

MVVM(MV) architektúra teda rozdeluje aplikáciu na 4 vrstvy: View, ViewModel, ModelView a Model. Popis jednotlivých vrstiev je k nahliadnutiu v následujúcich podsekciách. 

\subsection{View}

View je vrstva, ktorá popisuje a implementuje grafickú stránku aplikácie a určuje akým spôsobom sa dáta dodané vrstvou View model zobrazia užívateľovi. Viewy sú viazané na zodpovedajúce view modely za pomoci reaktívneho programovania. Spracovávajú akcie užívateľa a iniciujú reakcie zvyšných vrstiev architektúry prostredníctvom naviazaného view modelu. Následne zabezpečujú grafické znázornenie ním dodaných výsledkov reakcie.

View vrstva je kompletne implementovaná za pomoci Avalonia UI framework-u. V náväznosti na tento fakt sú v aplikácii použité tri hlavné typy view-ov:
\begin{itemize}
    \item \textbf{Window} - reprezentuje špecifické okno aplikácie, top-level kontajner, ktorý drží v sebe nejaký obsah. Samo o sebe veľmi nedefinuje vzhľad aplikácie. Slúži predovšetkým ako rám, v ktorom sa striedajú jednotlivé View-y. Každé okno má naviazaný svoj vlastný view model, ktorý drží informáciu o tom, aký View je v danej chvíli obsahom okna. Naviazaný view model taktiež obsahuje vlastnosti ktoré priamo súvisia s vlastnosťami daného okna.
    \item \textbf{View} - sú to hlavné zložky, ktoré nesú grafiku toho, čo sa aktuálne zobrazuje v konkrétnom okne. Reprezentujú jeden obraz ktorý je užívateľovi vykresľovaný v konkrétnom okne. Každý view je naviazaný na špecifický view model. Viaže sa na jeho vlastnosti a vykresluje dáta ktoré tieto vlastnosti obsahujú. 
    
    Býva zvykom že daný View sa zobrazuje práve v jednom konkrétnom okne. V takom prípade si na jeho view model drží referenciu view model okna. Za pomoci tejto referencie potom dokáže okenný view model oznámiť oknu, že sa v ňom má daný view zobraziť. 
    \item \textbf{DataTemplate} - definuje grafickú reprezentáciu dát dodávaných view modelom naviazanému view-u. Pre každý dátový typ, ktorý chce byť správne vykreslený pre užívateľa, musí existovať špecifický dátový template, pomocou ktorého sa daný údaj vykreslí. Dátové template-y sú špecifické tým, že sú raz definované pre celú aplikáciu, aby sa zachovala konzistencia vykreslovania jednotlivých dátových typov.
    
    Je potrebné podotknúť, že na to aby údaj vygenerovaný vo vrstve Modelov bolo možné vykresliť, musí preňho najprv existovať tzv. \textit{data view model} do ktorého sú jeho informácie zabalené a až v takejto podobe predávané nejakému view modelu, ktorý ich následne pomocou svojich vlastností odovzdá view-u na vykreslenie. Pre každý dátový view model sa následne hľadá príslušný DataTemplate, pomocu ktorého v ňom držané informácie zobrazia užívateľovi. 
\end{itemize}   

View vrstva je implementovaná pomocou dvoch jazykov a to C\# a XAML. Pomocu XAML definujeme všetky polohy a tvary grafických objektov a bind-ujeme vlastnosti týchto objektov na vlastnosti z vrstvy View model. Pre každý view máme obecne jeden špecifický XAML súbor ktorým ho implementujeme. Ku väčšine XAML súborov je priradený C\# zdrojový súbor v ktorom sa implementuje takzvaný \textit{code-behind}. V tomto zdrojovom súbore môžeme doplniť všetku funkcionalitu View-u, ktorú nebolo možné vyjadriť jazykom XAML.    

\subsection{View model}\label{ViewModel}

Druhou v poradí je vrstva View model. Táto vrstva je zodpovedná za logiku spracovania akcií užívateľa oznámených reaktívnym spôsobom View vrstvou a disponuje vedomím toho, aké akcie, v akom poradí sa majú vykonať pre zabezpečenie patričnej reakcie na daný impulz. K tomuto účelu využíva služby nižších vrstiev prostredníctvom volania na vrstve Model view. Tá na základe svojej vnútornej logiky vráti odpoveď s požadovanými údajmi. View model následne spracuje dodané dáta a predostrie ich view-u aby ich mohla ukázať užívateľovi.

View model zároveň, na základe aplikačnej logiky, koriguje a obmedzuje akcie užívateľa a tým zabraňuje vzniku nekonzistentných stavov aplikácie. Taktiež v niektorých prípadoch iniciuje interakcie s inými view modelmi za účelom dodania ich doplňujúcich služieb a aplikačnej logiky do jeho vlastného procesu.


Podobne ako vo View vrstve sa view modely delia natri základné typy:
\begin{itemize}
    \item \textbf{Session view model} + \textit{Main window view model} - odpovedajú jednotlivým \textit{session}-om, ktoré sú základným kameňom vertikálnej štruktúry aplikácie. Viac informácií o session-och je možné nájsť v sekcii \ref{Sessions}. Výnimkou je práve \textit{Main window view model}, ktorý je naviazaný na hlavné okno aplikácie a zabezpečuje preňho aplikačnú logiku. Klasické session view modely sú taktiež naviazané zvyčajne na jedno okno z view vrstvy pre ktoré zabezpečujú aplikačnú logiku.
    
    Každý session view model obsahuje kolekciu príslušných view model-ov ktoré spoločne implementujú mechanizmus daného session-u. Je zvykom, že v jednej chvíli je aktívny iba jeden view model. O aktívnom view modelu informuje session view model naviazané okno, ktoré potom v sebe zobrazuje odpovedajúci view aktívneho view modelu. Informovanie o aktívnom view modelu je aj hlavnou pracovnou náplňou session view modelu.

    K ďalším jeho povinnostiam patrí spracovávanie užívateľom vyvolaných akcií, ktoré sa týkajú samotného viazaného okna. Príkladom takej akcie môže byť požiadavka o zatvorenie dotyčného okna.
    \item \textbf{View model} - reprezentujú zložky, ktorých funkcionalita sa najväčšmi ponáša na obecnú, vyššie popísanú funkcionalitu vrstvy View model. Každý view model vo väčšine prípadov spadá pod réžiu konkrétneho session-u (alebo hlavného okna), pre ktorý implementuje určitú časť jeho mechanizmu. Klasicky sú view modely naviazané na príslušné view-y z View vrstvy. S tými následne reaktívne komunikujú a reagujú na ich podnety. View modely sú navzájom nezávislé. To znamená, že medzi nimi neprebieha takmer žiadna komunikácia ani presun dát. Túto funkciu na seba berie Model view vrstva.

    Väčšina view modelov by mala byť zahrnutá v zodpovedajúcom session view modele (poprípade Main window view modele). Ten potom zabezpečuje správu toho, ktorý view model je v danej chvíli aktívny. Výnimkou sú view modely, ktoré sú výhradne používane pre interakcie z iných view modelov, ktorým týmto spôsobom doručujú svoje služby. 
    
    Tieto view modely sú väčšinou vytvorené na mieste interakcie a po jej dokončení zanikajú. Pri ich vytvorení im sú zvyčajne predané nejaké vstupné parametre a keď je ich práca dokončená, tak vracajú jej výsledok.
    \item \textbf{Data view model} - slúžia ako kontajnery pre informácie abstrahované z dát vygenerovaných v Model vrstve. Dáta sú klasicky konvertované do ich zodpovedajúcich view modelov v Model view vrstve a už takto zabalené informácie sú predávané do View model vrstvy kde sú spracované a pomocou vlastností odovzdané vrstve View na vykreslenie. Data view modely môžu dodané informácie drobne upraviť takým spôsobom, aby ich bolo jednoduchšie vo View vrstve zobraziť.
    
    Z toho vyplýva, že na to aby nejaký údaj vygenerovaný v Model vrstve mohol byť prezentovaný užívateľovi, musí preňho existovať zodpovedajúci dátový view model. Zároveň na to, aby informácie obsiahnuté v dátovom view modele mohli byť vykreslené pre užívateľa, musí vo View vrstve existovať zodpovedajúci \textit{dátový template}, ktorý sa postará o ich správne grafické znázornenie.

    Niektoré dátové view modely nielen že obsahujú informácie príslušných dát ale obsahujú aj dáta samotné. Takéto dátové view modely označujeme pomocou slova \textit{wrapping}. (tvoria akýsi obal okolo dátových inštancií). Táto funkcionalita je dôležitá hlavne v prípadoch, kedy dátový view model slúži taktiež pre spätnú komunikáciu s Model view vrstvou. V takých prípadoch musí byť možné identifikovať, ktorú dátovú inštanciu \uv{obaluje}. Wrapping dátové view modely sú stotožnené s ich dátovým objektom a tiež sa pomocou neho identifikujú.  
\end{itemize}


View model je prvá z vrstiev, ktorá je kompletne písaná v jazyku C\#. Komunikácia medzi View model a View vrstvami funguje čisto na báze reaktívneho programovania za pomoci konštruktov z \textit{Reactive UI} framework-u.  

\subsection{Model view}

Treťou v poradí je, do klasickej MVVM architektúry pridaná, Model view vrstva. Táto vrstva je zodpovedná za \uv{vnútornú} logiku aplikácie. Priamo komunikuje s Model vrstvou a využíva jej zdroje pre zabezpečenie svojich služieb pre View model vrstvu. Dá sa povedať, že nedisponuje vlastným \uv{vedomím}. Medzi jej hlavné úlohy patria:  
\begin{itemize}
    \item prijímanie a spracovávanie požiadaviek od View modelu a dodávanie očakávaných výsledkov.  
    \item zabezpečovať vnútro-session-ovú komunikáciu. Model view-y v rámci jedného session-u si na seba držia referencie a predávajú si medzi sebou držané dáta. Táto komunikácia by mala byť vyšším vrstvam skrytá a na povrch by mal byť vidieť iba interface, pomocou ktorého prebieha komunikácia s View modelom.
    \item konverzia dát, získaných od Model vrstvy, do odpovedajúcich Data view modelov pri ich posielaní do vyšších vrstiev.  
\end{itemize}
Na druhú stranu medzi jej povinnosti nepatrí kontrola konzistentnosti jej vlastného stavu. O konzistenciu stavu aplikácie sa má starať View model.

V aplikácii používame model view-y dvoch typov:
\begin{itemize}
    \item \textbf{Session model view} + \textit{Main window model view}  - odpovedajú jednotlivým session-om. (Viac informácií o session-och je možné nájsť v sekcii \ref{Sessions}). 
    
    Ich hlavnou úlohou je vytvoriť a distribuovať model view-y odpovedajúce danému session-u. Pri ich inicializácii vytvorí medzi nimi väzby, ktoré sú následne počas behu aplikácie využívané na spomínanú vnútro-session-ovú komunikáciu. Taktiež zabezpečujú spracovávanie požiadaviek pre odpovedajúce session view modely. Tieto požiadavky sa typicky týkajú akcií, ktoré súvisia so session-om ako takým (nie s nejakou jeho časťou).  
    
    Výnimkou je práve \textit{main window model view}, ktorý je viazaný na view model hlavného okna a spracováva jeho požiadavky. V ostatných aspektoch je ale identický s klasickým session model view-om.
    
    Každému session model view-u odpovedá jeden konkrétny session view model. Ten pri svojej inicializácii predá model view-y, dodané v session model view-e, odpovedajúcim view modelom.   
    \item \textbf{Model view} - typ, ktorý nesie vyššie popísanú funkcionalitu Model view vrstvy. Zvyčajne pre každý view model existuje dedikovaný model view, ktorý sa stará o zabezpečenie view modelom požadovaných služieb. Nie je to však pravidlo, view model môže obsahovať odkazy na viacero model view-ov, ktorých služby následne využíva alebo sú predané vytvoreným, v interakciách využívaným, view modelom.

    Je zvykom, že každý model view spadá pod nejaký konkrétny session alebo hlavné okno. V takom prípade je daný model view vytváraný a distribuovaný odpovedajúcim session model view-om/main window model view-om.  
\end{itemize}

\subsection{Model}

Poslednou \uv{horizontálnou} vrstvou je Model. Model sa svojou štruktúrou diametrálne odlišuje od predchádzajúcich vrstiev. Je tvorený jednotlivými oblasťami, ktoré spravujú dedikovaní \textit{manažéri}. Manažéri sú pristupovaný z jednotlivých model view-ov a doručujú im svoje služby, či už informačné alebo výpočtové. Predstavujú interface-y ponúkajúce prívetivejší spôsob práce s vnútornými mechanizmami modelov. 

Model je predstaviteľom jedinej perzistentnej \uv{horizontálnej} vrstvy. Manažéri sú väčšinou singleton triedy, ktoré ponúkajú služby všetkým session-om počas celej doby behu aplikácie. Vďaka tomuto spôsobu obsluhovania je Model jediná vrstva, ktorej konštrukty niesu viazané na žiadnu vertikálnu vrstvu (session/hlavné okno). Z tohto návrhu Model vrstvy vyplýva ešte jedna dôležitá vlastnosť modelov a to, že musia byť schopné svoje služby dodávať paralelne pre viacero session-ov. 

Vrstva modelov je jednoducho rozšíriteľná o nových manažérov. Vďaka \textit{singleton} štruktúre sú dosiahnuteľný v podstate z akéhokoľvek miesta v programe a teda nemusia by5 nikde zahrnutí. Manažéri by zasa nemali mať problém prijímať nové, primerane vytvorené implementácie mechanizmov z ich oblastí. Napríklad by nemalo byť zložité dodať nový vyhľadávací algoritmus odpovedajúcemu manažérovi, ktorý ho následne bude ponúkať zvyšku aplikácie. %TODO mozno referencia na miesto, kde sa blizsie hovori o praci manazerov a jak si drzia kolekcie reprezentantov jednotlivych typov pouzitelnych v aplikacii.    

Špecifickým znakom komunikácie medzi Model a Model view vrstvami je, že pri nej dochádza ku strate typovej informácie dodávaných dát. Táto vlastnosť je motivovaná jednoduchým faktom, ktorým je udržanie vrstvy Model view jednoduchou. V Model vrstve sa totiž vo veľkom využívajú generiká pre jednoduché prenášanie typovej informácie v ich mechanizmoch. 

Využívanie generík v Model view vrstve by však prinieslo značné komplikácie v jej implementácii a tomu odpovedajúce zneprehladnenie kódu. Už len Model samotný trpí jemnou, generikami spôsobenou neprehľadnosťou. Z tohto dôvodu bolo rozhodnuté zabezpečiť jednoduchosť vrstvy Model view za cenu straty typovej informácie dát tečúcich z modelov do model view-ov. 

Pri opačnom smere komunikácie je manažérmi typová informácia dodaných parametrov opäť testovaná/získaná (väčšinou za pomoci tzv. \textit{generic visitor pattern}. Viac informácii o tejto modifikácii klasického \textit{visitor pattern} návrhového vzoru nájdete v % kde bude viac informa8cii...referencia na ne).  

V modelu existujú popri manažéroch ešte aj tzv. \textit{sub-manažéri}. Tieto entity sú ale určené pre využitie priamo z modelov. Podporujú generickú komunikáciu, na ktorej báze modely fungujú, bez straty typovej informácie a teda sú príjemnejšie pre modelovú komunikáciu než klasickí manažéri.     

\section{Session-y + \textit{hlavné okno}}\label{Sessions}

\subsection{Session-y}

Aplikácia, či už z vizuálneho, logického či implementačného hladiska, je rozdelená do tzv. \textit{session}-ov. Session-y sú najväčšie stavebné jednotky z ktorých každá predstavuje jedinečný mechanizmus dodávaný aplikáciou. Existencia ich inštancií je pominuteľná - vznikajú a zanikajú na popud užívateľa. Session-ov (aj rovnakého druhu) môže byť v aplikácii spustených viacero naraz. Každý session je klasicky tvorený odpovedajúcimi objektmi z prvých troch vrstiev horizontálnej štruktúry. 

Následne môžu session-y využívať dodatočné objekty z týchto vrstiev, ktoré patria hlavnému oknu alebo inému typu session-u. V takom prípade by malo byť ale poriadne rozmyslené, či takéto \uv{postranné} využitie dáva zmysel a či sa ním neporušujú zasady používania daného objektu. 

Poprípade je možné využívať špecifické model viewy a view modely prostredníctvom interakcií. V takom prípade by ale dané objekty mali byť na daný účel prispôsobené (info v podsekcii \ref{ViewModel}, bod \textbf{View model}, 3. odsek).

Session-y sú klasicky rozdelené na logické časti, ktoré spolupracujú na dodaní požadovaného mechanizmu. Môžu napríklad reprezentovať fázy jeho procesu. Tieto časti sú väčšinou tvorené špecifickými objektmi naprieč prvými tromi vrstvami horizontálnej štruktúry. 

Session-ové inštancie by medzi sebou nemali navzájom komunikovať ani zdielať svoje dáta. Mohlo by to viesť ku problémom s paralelizáciou služieb vykonávaných Model vrstvou.

Session-y ako také reprezentujú cestu ku všeobecnej rozšíriteľnosti aplikácie o nové mechanizmy. Ak by vznikla potreba, aby aplikácia obsahovala nejaký nový mechanizmus, stačí preňho vytvoriť odpovedajúci session a upraviť hlavné okno tak, aby ho vedelo ponúknuť užívateľovi ako jednu z možností.  

\subsection{Hlavné okno}\label{Hlavne_okno_obecne}

Session-y sú vytvárané a spravované \textit{hlavným oknom}. Je to jediná perzistentná \uv{vertikálna} vrstva aplikácie. Beh aplikácie začína s otvorením tohto okna a končí jeho zatvorením. Hlavné okno môže byť používané počas celého behu aplikácie. Pokiaľ je vydaný pokyn na uzavretie hlavného okna, pričom sú stále živé nejaké session-y, užívateľ môže byť na tento fakt upozornený. Ak mu to ale neprekáža, zatvorením hlavného okna sa zavrú aj všetky ostatné a aplikácia sa ukončí. Logika session-ov by mala túto skutočnosť brať do úvahy.

Horizontálna štruktúra je veľmi podobná tej session-ovej. Taktiež využíva MVVM(MV) architektúru. Jej časti sú ale z podstaty hlavného okna vytvárané len raz pri štarte aplikácie a zanikajú pri jej ukončení. \textit{Session view model} a \textit{Session model view} sú nahradené za funkčne identické \textit{Main window view model} a \textit{Main window model view}. Podobne ako pri session-och, aj hlavné okno je rozdelené do niekoľkých častí. Tie si medzi sebou rozdeľujú jeho funkcionalitu. Niektoré z jeho častí môžu byť sprístupnené rôznym session-om, aby si z nich mohli vytiahnuť potrebné parametre platiace pre celú aplikáciu.     

Ako bolo spomenuté na začiatku tejto podsekcie, hlavné okno vytvára, eviduje a spravuje inštancie všetkých session-ov. Definuje maximálny počet otvorených okien, spravuje session-y pri ukončovaní aplikácie, poskytuje dodávateľa hlavných parametrov, inicializuje pri vytváraní session-u jeho view modely a model viewy.  

\begin{figure}[p]\centering
\includegraphics[]{img/priklad_struktury}
\caption{Príklad možnej \uv{mriežkovej} štruktúry aplikácie.} 
\label{obr02:priklad_struktury}
\end{figure}
