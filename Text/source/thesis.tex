%%% Hlavní soubor. Zde se definují základní parametry a odkazuje se na ostatní části. %%%

% Meta-data o práci (je nutno upravit)
\input metadata.tex

% Vygenerujeme metadata ve formátu XMP pro použití balíčkem pdfx
\input xmp.tex

%% Verze pro jednostranný tisk:
% Okraje: levý 40mm, pravý 25mm, horní a dolní 25mm
% (ale pozor, LaTeX si sám přidává 1in)
\documentclass[12pt,a4paper]{report}
\setlength\textwidth{145mm}
\setlength\textheight{247mm}
\setlength\oddsidemargin{15mm}
\setlength\evensidemargin{15mm}
\setlength\topmargin{0mm}
\setlength\headsep{0mm}
\setlength\headheight{0mm}
% \openright zařídí, aby následující text začínal na pravé straně knihy
\let\openright=\clearpage

%% Pokud tiskneme oboustranně:
% \documentclass[12pt,a4paper,twoside,openright]{report}
% \setlength\textwidth{145mm}
% \setlength\textheight{247mm}
% \setlength\oddsidemargin{14.2mm}
% \setlength\evensidemargin{0mm}
% \setlength\topmargin{0mm}
% \setlength\headsep{0mm}
% \setlength\headheight{0mm}
% \let\openright=\cleardoublepage

%% Pokud práci odevzdáváme pouze elektronicky, vypadají lépe symetrické okraje
% \documentclass[12pt,a4paper]{report}
% \setlength\textwidth{145mm}
% \setlength\textheight{247mm}
% \setlength\oddsidemargin{10mm}
% \setlength\evensidemargin{10mm}
% \setlength\topmargin{0mm}
% \setlength\headsep{0mm}
% \setlength\headheight{0mm}
% \let\openright=\clearpage

%% Vytváříme PDF/A-2u
\usepackage[a-2u]{pdfx}

%% Přepneme na českou sazbu a fonty Latin Modern
\usepackage[slovak]{babel}
\usepackage{lmodern}

% Pokud nepouživáme LuaTeX, je potřeba ještě nastavit kódování znaků
\usepackage{iftex}
\ifpdftex
\usepackage[utf8]{inputenc}
\usepackage[T1]{fontenc}
\usepackage{textcomp}
\fi

%%% Další užitečné balíčky (jsou součástí běžných distribucí LaTeXu)
\usepackage{amsmath}        % rozšíření pro sazbu matematiky
\usepackage{amsfonts}       % matematické fonty
\usepackage{amsthm}         % sazba vět, definic apod.
\usepackage{bm}             % tučné symboly (příkaz \bm)
\usepackage{booktabs}       % lepší vodorovné linky v tabulkách
\usepackage{caption}        % umožní definovat vlastní popisky plovoucích objektů
\usepackage{csquotes}       % uvozovky závislé na jazyku
\usepackage{dcolumn}        % vylepšené zarovnání sloupců tabulek
\usepackage{floatrow}       % umožní definovat vlastní typy plovoucích objektů
\usepackage{graphicx}       % vkládání obrázků
\usepackage{icomma}         % inteligetní čárka v matematickém módu
\usepackage{indentfirst}    % zavede odsazení 1. odstavce kapitoly
\usepackage[nopatch=item]{microtype}  % mikrotypografická rozšíření
\usepackage{paralist}       % lepší enumerate a itemize
\usepackage[nottoc]{tocbibind} % zajistí přidání seznamu literatury,
                            % obrázků a tabulek do obsahu
\usepackage{xcolor}         % barevná sazba

% Balíček hyperref, kterým jdou vyrábět klikací odkazy v PDF,
% ale hlavně ho používáme k uložení metadat do PDF (včetně obsahu).
% Většinu nastavítek přednastaví balíček pdfx.
\hypersetup{unicode}
\hypersetup{breaklinks=true}

% Balíčky pro sazbu informatických prací
\usepackage{algpseudocode}  % součást balíčku algorithmicx
\usepackage[Algoritmus]{algorithm}
\usepackage{fancyvrb}       % vylepšené prostředí verbatim
\usepackage{listings}       % zvýrazňování syntaxe zdrojových textů

% Cleveref může zjednodušit odkazování, ale jeho užitečnost pro češtinu
% je minimalní, protože nezvládá skloňování.
% \usepackage{cleveref}

% Formátování bibliografie (odkazů na literaturu)
% Detailní nastavení můžete upravit v souboru macros.tex.

% POZOR Zvyklosti různých oborů a kateder se liší. Konzultujte se svým
% vedoucím, jaký formát citací je pro vaši práci vhodný!
%
% Základní formát podle normy ISO 690 s číslovanými odkazy
\usepackage[natbib,style=iso-numeric,sorting=none]{biblatex}
% ISO 690 s alfanumerickými odkazy (zkratky jmen autorů)
%\usepackage[natbib,style=iso-alphabetic]{biblatex}
% ISO 690 s citacemi tvaru Autor (rok)
%\usepackage[natbib,style=iso-authoryear]{biblatex}
%
% V některých oborech je běžnější obyčejný formát s číslovanými odkazy
% (sorting=none říká, že se bibliografie má řadit podle pořadí citací):
%\usepackage[natbib,style=numeric,sorting=none]{biblatex}
% Číslované odkazy, navíc se [1,2,3,4,5] komprimuje na [1-5]
%\usepackage[natbib,style=numeric-comp,sorting=none]{biblatex}
% Obyčejný formát s alfanumerickými odkazy:
%\usepackage[natbib,style=alphabetic]{biblatex}

% Z tohoto souboru se načítají položky bibliografie
\addbibresource{literatura.bib}

% Definice různých užitečných maker (viz popis uvnitř souboru)
\input macros.tex

%%% Titulní strana a různé povinné informační strany
\begin{document}
\include{title}

%%% Strana s automaticky generovaným obsahem práce

\tableofcontents

%%% Jednotlivé kapitoly práce jsou pro přehlednost uloženy v samostatných souborech
\chapter*{Úvod}
\addcontentsline{toc}{chapter}{Úvod}

Veľmi často v živote narážame na situácie, kedy sa potrebujeme dostať z bodu A do bodu B čo najrýchlejším spôsobom. V mnohých prípadoch máme k dispozícii sieť cestných komunikácii či chodníkov, na základe ktorých sa môžeme rozhodovať, ktorá trasa je pre nás vyhovujúca. Pre takéto prípady nám veľmi dobre poslúžia už existujúce navigačné systémy, ktoré majú podrobne zmapovanú cestné siete a vedia nám na základe parametrov týchto ciest určiť najrýchlejšiu alebo najúspornejšiu trasu.

Avšak môžu nastať aj situácie, kedy je cestná sieť priveľmi riedka, až takmer neprítomná. Väčšinou takéto situácie nastávajú vo voľnej prírode a oblastiach ďaleko od civilizácie. V takých prípadoch nám konvenčné navigácie veľmi dobre neposlúžia. Ak máme k dispozícii kvalitnú mapu, môžeme sa pokúsiť v nej nájsť vyhovujúcu trasu vlastnými silami, ale často existuje príliš mnoho možností, ktorými sa môžeme vydať. Preto by sa niekedy hodilo mať nástroj, ktorý na základe zadaných parametrov nájde na predloženej mape najrýchlejšiu trasu, po ktorej sa človek môže vydať k vytýčenému cieľu.

Príklady využitia takéhoto software-u by sme mohli nájsť v špecifických profesiách ako sú napríklad lesníctvo, záchranné služby či ozbrojené sily. V týchto profesiách je potrebné z času na čas sa dostať na odľahlé miesto, aby bolo možné vykonať ich zámery.

Takýto software by však našiel uplatnenie taktiež v rekreačných aktivitách. Či už turizmus alebo športové aktivity sa často odohrávajú vo voľnej prírode. Z rôznych dôvodov sa potom môže zísť schopnosť nájdenia najrýchlejšej trasy späť do najbližšej obývanej oblasti, či už za alebo bez pomoci ciest. 

Rád by som vyzdvihol špecificky jedno športové odvetvie, ktoré veľmi úzko súvisí s hľadaním ciest v otvorenom teréne a to \textit{orientačný beh}. \uv{Orientačný beh (skratka OB) je športové odvetvie vytrvalostného charakteru, pri ktorom je úlohou prejsť alebo prebehnúť podľa mapy a buzoly trať vyznačenú na mape za čo najkratší čas. V teréne nie je trať vyznačená, sú tam umiestnené iba kontrolné stanovišťa (Kontroly)}\cite{CoJeOrientak}. Tento šport bol hlavnou motiváciou pre vytvorenie aplikácie, v ktorej by si užívateľ mohol nechať vykresliť na mape najrýchlejší postup pre ním zadanú trať. 

\bigskip

V tejto práci sú popísané myšlienky použité pri vytváraní spomínanej aplikácie, jej architektúry a implementácie.

V prvej kapitole je možné nájsť abstrakciu nad samotnou problematikou hľadania najrýchlejšej trasy v otvorenom teréne podľa konkrétnej mapy. Táto abstrakcia je poňatá z pohľadu orientačného behu. Ďalej je v nej možné nájsť informácie o využitých prostriedkoch pri implementácii aplikácie. Týmito prostriedkami sú napríklad využívaný jazyk a knižnice, ale aj použitý návrhový vzor architektúry aplikácie. 

V následujúcej kapitole je predostretý bližší pohľad na spomínanú architektúru aplikácie, jej \uv{horizontálne} a \uv{vertikálne} členenie na vrstvy. Funkcie a správanie jednotlivých vrstiev sú v tejto kapitole dopodrobna vysvetlené.

V tretej kapitole sa pozrieme na aktuálne existujúce vertikálne vrstvy aplikácie. Ich funkciu a architektúru popíšeme po jednotlivých častiach. Popíšeme špecifické vlastnosti a spôsoby použitia týchto častí. 

Vo štvrtej a zároveň poslednej kapitole bude možný k nahliadnutiu popis horizontálnej vrstvy zvanej \textit{Model}. Táto vrstva sa delí na mnoho oblastí, z ktorých každá spravuje určitý typ dát a mechanizmov využívaných vo zvyšných vrstvách aplikácie. Štruktúra implementácií jednotlivých oblastí bude v tejto kapitole detailne rozobraná.


% \include{kap01}
% \include{kap02}
% \include{kap03}
% \include{kap04}

\chapter{Všeobecné informácie o~aplikácii}

\section{Aspekty hľadania najrýchlejšej trasy (v OB)}\label{Aspekty_hladania}

Mapy pre~orientačný beh sú veľmi detailné, a~bežec má preto mnoho informácií na~to, aby~si mohol zvoliť ideálnu trasu. Za~predpokladu, že~bežec nerobí chyby a~beží presne podľa svojho zámeru, sú pre~výber najrýchlejšieho postupu dôležité dva druhy objektov: líniové (cesty, potoky, prieseky, vrstevnice, ...) a~plošné (lúky, kroviská, vodné objekty, močiare, ...). Tie určujú typ terénu nachádzajúci sa v~danej časti mapy a~tým pádom aj~veľkosť odporu, ktorý~je kladený rýchlosti behu pretekára. Táto veličina sa dá použiť ako hlavný parameter, ktorý~bude určovať preferencie výberu postupu.

Proces hľadania cesty sa skladá z~dvoch hlavných častí: 
\begin{itemize}
    \item \textbf{vytvorenie mapovej reprezentácie} - Na~začiatku je potrené na~základe mapového súboru vygenerovať mapovú reprezentáciu, na~ktorej bude možné cestu vyhľadávať. Mapovou reprezentáciou by mal byť objekt, ktorý~dobre vystihne topografiu mapy a~umožní hľadanie najideálnejšej trasy. V~našej aplikácii budú týmito objektmi \textit{orientované ohodnotené grafy}.  
    \item \textbf{aplikácia vyhľadávacieho algoritmu} - Mapová reprezentácia s vybraným užívateľským modelom sú predané vyhľadávaciemu algoritmu. Ten za~použitia interface-u mapovej reprezentácie a výpočtov užívateľského modelu vyhľadá v~mapovej reprezentácii najkratšiu cestu na zadanej trati. V~našom prípade najkratšia cesta znamená tá najrýchlejšia.
\end{itemize}

V~nasledujúcich podsekciách spomenieme koncepty, ktoré~budú v~procese hľadania najrýchlejších ciest vystupovať. Medzi jednotlivými konceptmi sú tvorené rôzne závislosti. Grafické znázornenie týchto závislostí je dostupné k nahliadnutiu na~konci sekcie v~Obrázku~\ref{obr01:konceptove_zavislosti}.   

\subsection{Mapy}\label{mapy}

Na začiatok je potrebné spomenúť koncept mapy. V~procese hľadania ciest bude na~viacerých miestach potrebné agregovať dáta z~užívateľom vybraného mapového súboru. Aby~sme nemuseli neustále čítať priamo zo~súboru, budeme si udržovať jeho obsah v~pamäti prívetivejším spôsobom. 

Touto formou bude práve \textit{mapa}. Mapa si bude udržovať všetky objekty definované v~mapovom súbore a~keď bude potrebné agregovať z~daného súboru nejakú informáciu (mapovú reprezentáciu, mapovú grafiku, ...), použije sa namiesto súboru zodpovedajúci mapový objekt.

Je vhodné zmieniť, že~koncept mapy je odlišný od konceptu \textit{mapovej reprezentácie}. Mapa, na~rozdiel od jej reprezentácie, neobsahuje žiadne zložité prepojenia medzi objektami ktoré~v~sebe drží. Jej vytvorenie by malo byť rýchle, s~lineárnou časovou zložitosťou v~závislosti od~veľkosti mapového súboru.

\subsection{Mapové reprezentácie}\label{mapove_reprezentacie}

Mapové reprezentácie sú jednou z~dôležitých zložiek procesu hľadania ciest. Sú to jednotky, na~ktorých sa vyhľadávanie uskutočňuje. Mapové objekty sú oproti \textit{mapám} zložitejšie objekty, ktoré~už v~sebe zahŕňajú plno závislostí a~prepojení. Ich generovanie môže zabrať oveľa viac času, ako generovanie mapových objektov. 

Ako už bolo spomenuté vyššie, v~aplikácii budú mapovými reprezentáciami \textit{orientované, ohodnotené grafy} (ďalej iba grafy). Napriek tomu, že~mapová reprezentácia a~graf budú v~aplikácii reprezentovať rovnaký objekt, ich významy sú odlišné:

\begin{itemize}
    \item \textbf{Mapová reprezentácia} hovorí o~tom, ako daný objekt pracuje interne. Popisuje jeho vlastnosti, mechanizmy a~spôsoby akými generuje výsledný \textit{graf}, v~ktorom sa následne hľadá cesta. Objekt, ktorý~je agregovaný z~mapy, je v~prvom rade mapovou reprezentáciou a~až v~druhom rade grafom.
    
    Príkladom myšlienky, ktorú môže mapová reprezentácia vyjadrovať je \uv{samo-zahusťujúci} sa graf. Takáto mapová reprezentácia počas behu algoritmu zahusťuje predom pripravený graf o~ďalšie vrcholy na~miestach, v~ktorých sa prehľadávanie aktuálne uskutočňuje. Redukuje sa tým veľkosť predom vygenerovaného grafu a~zvyšuje jeho presnosť na~miestach, kde je to potrebné.   
    \item\textbf{Graf} na~druhej strane hovorí o~vlastnostiach objektu, ktoré~sú viditeľné navonok. Bude informovať o~\uv{grafových} službách objektu, ktoré~dokáže poskytnúť. Tieto služby môžu byť obmedzené práve vnútornou štruktúrou ktorá je definovaná \textit{mapovou reprezentáciou}. Graf sa berie ako druhotný produkt spracovania mapy.   

    Príkladom vlastnosti, ktorú môže graf prezentovať vonkajšiemu svetu, je možnosť poskytnutia kompletne vygenerovaného grafu, pri~ktorom sa vonkajší užívateľ nemusí obávať, že~by sa počas práce s~ním graf nejakým spôsobom modifikoval. Túto vlastnosť napríklad nevie zaručiť graf ktorý~zodpovedá vyššie zmienenej mapovej reprezentácii, ktorá stav~grafu počas práce neustále modifikuje, zahusťuje ho.
\end{itemize}

Pojmy \textit{mapová reprezentácia} a~\textit{graf} reprezentujú dve odlišné myšlienky a~v~texte budú na~ich základe aj~využívané. Je však potrebné mať na~pamäti, že~v~aplikácii sú tieto pojmy jednou a~tou istou dátovou štruktúrou. Z~hľadiska vytvárania a~používania sa teda mapová reprezentácia a~graf stávajú zameniteľnými.

\subsection{Užívateľské modely}\label{uzivatelske_modely}

Je potrebné si uvedomiť, že~rôzni bežci majú rôzne schopnosti a~preto aj~ich preferencie na~výber trasy nemusia byť rovnaké. Niekto sa dokáže rýchlejšie predierať cez husté pasáže, inému ide rýchlejšie beh cez močiar a~ďalšiemu vyhovujú dlhšie postupy po~cestách. 

Preto by bolo vhodné, aby~užívateľ aplikácie mal možnosť aplikovať svoje preferencie do~procesu vyhľadávania. K~tomuto účelu boli vytvorené tzv. \uv{užívateľské modely}. Užívateľ si pomocou nich môže vytvoriť vlastný \uv{profil}, na~základe ktorého bude vyhľadávanie cesty prispôsobené.

Užívateľské modely vďaka svojej informovanosti nadobúdajú zodpovednosť za~výpočty hodnôt, ktoré~sú závislé na~preferenciách užívateľa. Stávajú sa teda dôležitou zložkou procesu vyhľadávania ciest, nakoľko vyhľadávacie algoritmy sú závislé do~nimi vykonávaných výpočtov.

\subsection{Výškové dáta}\label{vyskove_data}

Jedným z~dôležitých a~neopomenuteľných faktorov~voľby najrýchlejšieho postupu je prevýšenie, ktoré~je potrebné pri~jeho prevedení zdolať. Mnoho typov~máp popisuje reliéf krajiny pomocou takzvaných \textit{vrstevníc}. Vrstevnica je krivka na~mape, ktorá spája body rovnakej nadmorskej výšky. Pre~človeka je vrstevnicová abstrakcia výšky terénu veľmi ľahko pochopiteľná a~spracovateľná.

Pre strojové spracovanie mapy však vrstevnice predstavujú veľmi neprirodzený spôsob reprezentácie nadmorskej výšky. Vypracovanie reliéfneho obrazu za~pomoci vrstevníc samotných je veľmi ťažká úloha. V~niektorých prípadoch (mapy pre~OB napríklad) dokonca jednotlivé vrstevnice nie sú v~mapovom súbore reprezentované jedným objektom, nakoľko sú kvôli dobrej čitateľnosti máp na~viacerých miestach prerušované.
%TODO: mozno najst nejaku tu pracu co sa tym zaoberala, zistit od risa, odkazat na~nu

Z~vyššie uvedených dôvodov je v~aplikácii zahrnutý systém, ktorý~sprostredkováva užívateľom možnosť stiahnutia a~spravovania digitálnych výškových dát, ktoré~následne môžu byť použité ako pomocný zdroj v~procese tvorby mapových reprezentácií. Pre~konštrukciu mapových reprezentácií pre~niektoré mapové formáty bude nutná prítomnosť zodpovedajúcich výškových dát a~pri ich absencii jednoducho nebude možné mapové reprezentácie vytvoriť.

Výškové dáta môžu byť sťahované z~viacerých zdrojov. Každý zdroj môže definovať viacero dátových distribúcií, ktoré~dokáže ponúknuť. Tieto distribúcie sa väčšinou líšia kvalitou a~dostupnosťou sprostredkovaných dát (v~niektorých prípadoch je k~stiahnutiu výškových dát potrebná užívateľova autorizácia).

\subsection{Atribútové template-y}\label{templatey}

Ďalšia vec, nad ktorou je potrebné sa zamyslieť je, akým spôsobom sa bude v~grafoch mapových reprezentácií uchovávať informácia o~mapových vlastnostiach a~atribútoch, ktoré~konkrétne vrcholy a~hrany grafu reprezentujú. Zároveň je potrebné aby~dotyčné vlastnosti dokázal príslušný užívateľský model spracovať a~dopočítať z~nich hodnoty, potrebné pre~beh vyhľadávacích algoritmov. Používané atribúty, ktoré~agregujeme z~máp do~mapových reprezentácií preto musia byť jednotné v~celom procese hľadania cesty, od vytvárania mapovej reprezentácie, po~spustenie vyhľadávacieho algoritmu.

V~aplikácii nám definíciu a~jednotnosť atribútov budú zabezpečovať tzv. \textit{template}-y. Každý template bude definovať jedinečnú sadu vrcholových a~hranových atribútov. Príkladom pre~takúto kolekciu pre~potreby OB môže byť napríklad:
\begin{itemize}
    \item vrcholové atribúty - pozícia, výška, indikátory reprezentovaných terénnych objektov (či~sa daný vrchol nachádza na~ceste, v~húštine, na~lúke, v~dobre priebežnom lese,...),
    \item hranové atribúty - či~daná hrana reprezentuje úsek nejakej cesty, hranu nejakého objektu (lúky, húštiny, močiaru,...), sklon terénu
\end{itemize}

Na template-och ako takých bude teda závisieť: 
\begin{itemize}
    \item \textbf{výber užívateľských modelov} - model musí vedieť spracovať atribúty definované daným template-om a~vrátiť od neho požadované výsledky.
    \item \textbf{formát vybraného mapového súboru} - musí existovať konvertor mapy špecifického formátu na~zodpovedajúcu mapovú reprezentáciu, v~ktorej vrcholoch a~hranách sú obsiahnuté atribúty definované daným template-om. Túto závislosť môžeme brať aj~opačným smerom - pre~vybraný mapový formát môžeme zvoliť len použiteľný template.
\end{itemize}

\subsection{Vyhľadávacie algoritmy}\label{vyhladavacie_algoritmy}

Nakoniec nemôžeme opomenúť koncept samotných vyhľadávacích algoritmov, poslednú neodmysliteľnú súčasť procesu hľadania ciest. Vyhľadávací algoritmus, podobne ako \textit{mapová reprezentácia}, reprezentuje koncept vnútorného mechanizmu ktorým je najkratšia cesta v~grafe hľadaná. Príkladom takéhoto algoritmu môže byť napríklad algoritmus \textit{A*}. 

Vstupom do~každého algoritmu sú:
\begin{itemize}
    \item \textbf{graf}, na~ktorom je hľadanie najkratšej cesty vykonané a~
    \item \textbf{užívateľský model}, ktorý~algoritmus nutne potrebuje k~výpočtom váh grafových hrán a~iných hodnôt potrebných k~jeho správnemu chodu.  
\end{itemize}

Každý algoritmus následne ponúka množinu jeho implementácií. Každá implementácia definuje množinu vlastností, ktoré~vložený graf a~užívateľský model musia spĺňať. Napríklad implementácie A* algoritmu budú požadovať, aby~užívateľský model bol schopný popri výpočte váh pre~grafové hrany, dodať ešte aj~výpočet heuristiky využívanej týmto algoritmom.

Stav používaného grafu sa môže počas behu algoritmu meniť. Teda je potrebné ho vždy na~konci vyhľadávania opäť vrátiť do~pôvodného stavu a~zaručiť, že~graf je v~jednu chvíľu súčasťou len jedného vyhľadávacieho procesu.

Vyhľadávanie samotné je sprostredkované dvomi spôsobmi. Buď sa požiada algoritmus priamo o~nájdenie cesty na~danej trati alebo~sa využije takzvaný \textit{executor} algoritmu. 

Executor nám sprostredkuje vybraný algoritmus, ktorý~bude pracovať s~dodanou mapovou reprezentáciou a~užívateľským modelom. Executor si pri~inicializácii pre~seba uzamkne dodanú mapovú reprezentáciu a~až do~jeho uvolnenia ju neodblokuje. Do~tejto doby môže prijímať rôzne vstupné trate a~vracať pre~nenájdené cesty. Hlavná výhoda executor-ov je práve ich schopnosť uzamknutia mapovej reprezentácie na~dlhšiu dobu. Vďaka tomu ju dokážu využívať pre~viacero separátnych vyhľadávaní, bez potreby jej neustáleho uvádzania do~konzistentného stavu.  

\begin{figure}[h]\centering
\includegraphics[]{img/konceptove_zavislosti}
\caption{Diagram závislostí jednotlivých konceptov} 
\label{obr01:konceptove_zavislosti}
\end{figure}

\pagebreak

\section{Zvolené implementačné prostriedky}

\subsection{Použitý programovací jazyk}

Pre implementáciu aplikácie bol vybraný jazyk C\#. Jazyk bol vybraný pre~jeho jednoduchosť a~bezpečnosť použitia. Alternatívnou volbou by mohol byť jazyk Java, ktorý~má blízko práve k~jazyku C\#. Voľba však padla na~C\# pre~široký výber možných framework-ov pre~implementáciu GUI. 

Ďalšími možnosťami by mohol byť typicky vysoko-úrovňový jazyk Python alebo~nízko úrovňový jazyk C++. Nakoľko v~aplikácii bude prebiehať mnoho výpočtov, Python by nebol vhodnou voľbou pre~nedostatočnú rýchlosť ním vytvoreného strojového kódu. Na~druhú stranu, C++ by bol dobrým kandidátom z~hľadiska výpočtovej sily. V~tomto prípade však narážame na~programátorskú neprívetivosť tohto jazyka. Pri~takomto väčšom projekte sme považovali za~potrebné mať istotu, že~nás použitý jazyk v~programovaní podrží. 

Vhodnou úpravou implementácie by bolo implementovať výpočtovo náročné procesy v~jazyku typu C++ a~následne túto implementáciu volať externe z~C\# jadra.     

\subsection{Užívateľské rozhranie}

Pre implementáciu užívateľského rozhrania sme sa rozhodli pre~\textit{Avalonia UI} framework. Užívateľské rozhranie je veľmi jednoduché. Bolo vytvorené tak, aby~zabezpečilo všetku funkcionalitu.

V~C\# existuje viacero možných knižníc, z~ktorých bolo možné si vybrať:
\begin{itemize}
    \item GUI knižnice, ktoré~sú súčasťou samotného .NET framework-u:
    \begin{itemize}
        \item \textbf{Windows Forms} je klasická GUI knižnica. Je jednoducho použiteľná a~vďaka jej dlhoročnej podpore aj~robustná a~spoľahlivá. Jej vek je však aj~jej nevýhodou, nakoľko vzhľad aplikácií vytvorených za~pomoci Windows Forms pôsobí pomerne zastaralo. Ďalšou nevýhodou je rastrová povaha jeho renderovcieho engine-u. Táto vlastnosť sa nehodí pre~aplikáciu, ktorej jedným z~hlavných účelov je vykresľovanie mapových, vektorových objektov.     
        \item \textbf{Windows Presentation Foundation (WPF)} je modernejší nástupca Windows Forms. Aplikácie tvorené touto knižnicou majú modernejší vzhľad a~sú generované renderovacím enginom založeným na~vektorovej grafike. K~programovaniu sa tu využíva popri C\# aj~XAML, v~ktorom sa definuje layout užívateľského rozhrania. Avšak spoločne s~Windows Forms je ich nevýhodou platformová závislosť na~operačnom systéme Windows.
        \item \textbf{.NET MAUI} je nástupcom \textit{Xamarin.Forms}. Je to open-source-ový cross-platformný framework s~množinou UI nástrojových balíčkov pre~jednotlivé platformy. Podobne ako vo~WPF sa pre~vytváranie UI využíva kombinácia C\# a~XAML. 
    \end{itemize}
    \item Alternatívou k~vstavaným .NET GUI knižniciam je práve nezávislá, open-source knižnica \textbf{Avalonia UI}. Čo sa vlastností je veľmi dobre porovnateľná s~\textit{.NET MAUI}. Hlavný rozdiel medzi týmito dvomi knižnicami je v~spôsobe, akým vykresľujú užívateľské rozhranie. Avalonia zapája kresliaci engine poháňaný knižnicou pre~2D grafiku \textit{Skia}. Na~druhú stranu MAUI využíva natívne nástrojové balíčky pre~každú platformu zvlášť. Ďalším rozdielom je, že~Avalonia, na~rozdiel od MAUI, podporuje aj~niektoré distribúcie Linuxových systémov.
\end{itemize}

Rozhodnutie nakoniec padlo na~využitie framework-u Avalonia UI. Nakoľko je veľmi podobný natívnemu .NET MAUI, rozhodla podpora pre~Linuxové systémy a~aj kvalitne spravená dokumentácia, z~ktorej sa dalo ľahko  vyčítať, ako sa s~knižnicou má pracovať. 

Informované rozhodnutie pre~výber GUI knižnice bolo učinené na~základe zdrojov \cite{WpfGuide,WhatIsMAUI,AvaloniaMauiComparison}. Zároveň väčšinu informácií, ktoré~sme o~Avalonia UI počas tvorby programu čerpali, pochádzali z~jej dokumentácie \cite{AvaloniaDokumentacia}.

\subsection{Architektúra Model-View-ViewModel (MVVM)}\label{ArchitekturaMVVM}

Jeden z~ďalších dôležitých aspektov, ktorý~hral úlohu vo~výbere knižnice pre~užívateľské rozhranie, bola podpora \textit{MVVM} návrhového vzoru. Dokumentácia Avalonia UI v preklade popisuje architektúru MVVM následovne: \uv{Model-View-View Model (MVVM) návrhový vzor je bežný spôsob štruktúrovania aplikácie užívateľského rozhrania. Používa systém viazania dát, ktorý~pomáha presúvať dáta medzi jeho časťami view-om a~view model-om. To znamená, že~dosahuje oddelenie aplikačnej logiky (view model) od~zobrazenia užívateľského rozhrania (view). Oddelenie medzi aplikačnou logikou a~obchodnými službami (model) sa bežne dosahuje pomocou systému Dependency Injection (DI).}\cite{MVVMDefByAvalonia}.

% The Model-View-View Model (MVVM) pattern is a~common way of structuring a~UI application. It uses a~data binding system that helps move data between its view and view model parts. This means it achieves separation of application logic (view model) from the display of the UI (view). Separation between the application logic and the business services (model) is commonly achieved by a~Dependency Injection (DI) system. 

MVVM architektúra je vhodná pre~našu aplikáciu, nakoľko pre~jej rozsah a~komplexnosť by klasická \textit{event-driven code-behind} architektúra nemusela postačovať. Týmto spôsobom zaručíme lepšiu separáciu a~dostatočnú vnútornú nezávislosť kódu.    

V~našej aplikácii bude tento návrhový vzor uplatnený s~jemnou obmenou. Vrstva View model, ktorá v~pôvodnom návrhovom vzore zastávala úlohu logiky aplikácie, bude rozdelená na~dve časti: View model a~novovzniknutý \textit{Model view}. Viac informácií o~tejto úprave je možné nájsť v~Sekcii~\ref{MVVMNavrhovyVzor}. 

\subsection{Reaktívne programovanie}

Ďalším dôležitým aspektom aplikácie je Avaloniou a~architektúrou MVVM iniciované využitie \textit{reaktívneho programovania.} Avalonia pre~aplikáciu tejto paradigmy využíva framework \textbf{Reactive UI}. 

Tento framework v preklade definuje reaktívne programovanie ako \uv{Reaktívne programovanie je programovanie s~asynchrónnymi dátovými tokmi.} a~dodáva \uv{Event bus-y alebo~typické udalosti kliknutia sú v~podstate asynchrónny tok udalostí, ktorý~je možné sledovať a~vykonávať na~ňom vedľajšie úkony. Reaktívne programovanie je táto myšlienka na~steroidoch. Je možné vytvárať dátové toky z~čohokoľvek, nielen z~udalostí kliknutia a~pohybu myši. Tieto toky sú lacné a~všadeprítomné, a~čokoľvek môže byť tokom: premenné, vstupy od používateľa, vlastnosti, cache, dátové štruktúry, atď.}\cite{ReactiveProgrammingByReactiveUI}.

% Reactive programming is programming with asynchronous data streams.
% Event buses or your typical click events are really an asynchronous event stream, on which you can observe and do~some side effects. Reactive programming is that idea on steroids. You are able to create data streams of anything, not just from click and hover events. Streams are cheap and ubiquitous and anything can be a~stream: variables, user inputs, properties, caches, data structures, etc.

V~aplikácii sa bude tento framework využívať prevažne vo~vrstvách View a~View model, medzi ktorými prebieha väčšina reaktívnej komunikácie. 
\chapter{Architektúra ako celok}

O architektúre aplikácie sa dá premýšlať ako o \textit{mriežke}. Je rozdelená na horizontálne (MVVM návrhový vzor) a vertikálne (\textit{session}-y + hlavné okno) vrstvy. V následujúcich sekciách popíšeme, ako jednotlivé vrstvy vyzerajú, aké su ich úlohy a ako medzi sebou komunikujú. Na konci kapitoly je následne k nahliadnutiu diagram \ref{obr02:priklad_struktury} znázorňujúci príklad možnej architektúry aplikácie.   

\section{MVVM(MV) návrhový vzor}\label{MVVMNavrhovyVzor}

Ako už bolo spomenuté v podsekcii \ref{ArchitekturaMVVM}, v aplikácii je využívaný návrhový vzor MVVM s drobnou obmenou. Táto obmena sa týka rozdelenia originálnej vrstvy view model na dve časti: view model a model view. Toto rozdelenie zaručí ešte o niečo lepšiu separáciu kódu a odľahčí tým úlohy view model vrstvy. 

MVVM(MV) architektúra teda rozdeluje aplikáciu na 4 vrstvy: View, ViewModel, ModelView a Model. Popis jednotlivých vrstiev je k nahliadnutiu v následujúcich podsekciách. 

\subsection{View}

View je vrstva, ktorá popisuje a implementuje grafickú stránku aplikácie a určuje akým spôsobom sa dáta dodané vrstvou View model zobrazia užívateľovi. Viewy sú viazané na zodpovedajúce view modely za pomoci reaktívneho programovania. Spracovávajú akcie užívateľa a iniciujú reakcie zvyšných vrstiev architektúry prostredníctvom naviazaného view modelu. Následne zabezpečujú grafické znázornenie ním dodaných výsledkov reakcie.

View vrstva je kompletne implementovaná za pomoci Avalonia UI framework-u. V náväznosti na tento fakt sú v aplikácii použité tri hlavné typy view-ov:
\begin{itemize}
    \item \textbf{Window} - reprezentuje špecifické okno aplikácie, top-level kontajner, ktorý drží v sebe nejaký obsah. Samo o sebe veľmi nedefinuje vzhľad aplikácie. Slúži predovšetkým ako rám, v ktorom sa striedajú jednotlivé View-y. Každé okno má naviazaný svoj vlastný view model, ktorý drží informáciu o tom, aký View je v danej chvíli obsahom okna. Naviazaný view model taktiež obsahuje vlastnosti ktoré priamo súvisia s vlastnosťami daného okna.
    \item \textbf{View} - sú to hlavné zložky, ktoré nesú grafiku toho, čo sa aktuálne zobrazuje v konkrétnom okne. Reprezentujú jeden obraz ktorý je užívateľovi vykresľovaný v konkrétnom okne. Každý view je naviazaný na špecifický view model. Viaže sa na jeho vlastnosti a vykresluje dáta ktoré tieto vlastnosti obsahujú. 
    
    Býva zvykom že daný View sa zobrazuje práve v jednom konkrétnom okne. V takom prípade si na jeho view model drží referenciu view model okna. Za pomoci tejto referencie potom dokáže okenný view model oznámiť oknu, že sa v ňom má daný view zobraziť. 
    \item \textbf{DataTemplate} - definuje grafickú reprezentáciu dát dodávaných view modelom naviazanému view-u. Pre každý dátový typ, ktorý chce byť správne vykreslený pre užívateľa, musí existovať špecifický dátový template, pomocou ktorého sa daný údaj vykreslí. Dátové template-y sú špecifické tým, že sú raz definované pre celú aplikáciu, aby sa zachovala konzistencia vykreslovania jednotlivých dátových typov.
    
    Je potrebné podotknúť, že na to aby údaj vygenerovaný vo vrstve Modelov bolo možné vykresliť, musí preňho najprv existovať tzv. \textit{data view model} do ktorého sú jeho informácie zabalené a až v takejto podobe predávané nejakému view modelu, ktorý ich následne pomocou svojich vlastností odovzdá view-u na vykreslenie. Pre každý dátový view model sa následne hľadá príslušný DataTemplate, pomocu ktorého v ňom držané informácie zobrazia užívateľovi. 
\end{itemize}   

View vrstva je implementovaná pomocou dvoch jazykov a to C\# a XAML. Pomocu XAML definujeme všetky polohy a tvary grafických objektov a bind-ujeme vlastnosti týchto objektov na vlastnosti z vrstvy View model. Pre každý view máme obecne jeden špecifický XAML súbor ktorým ho implementujeme. Ku väčšine XAML súborov je priradený C\# zdrojový súbor v ktorom sa implementuje takzvaný \textit{code-behind}. V tomto zdrojovom súbore môžeme doplniť všetku funkcionalitu View-u, ktorú nebolo možné vyjadriť jazykom XAML.    

\subsection{View model}\label{ViewModel}

Druhou v poradí je vrstva View model. Táto vrstva je zodpovedná za logiku spracovania akcií užívateľa oznámených reaktívnym spôsobom View vrstvou a disponuje vedomím toho, aké akcie, v akom poradí sa majú vykonať pre zabezpečenie patričnej reakcie na daný impulz. K tomuto účelu využíva služby nižších vrstiev prostredníctvom volania na vrstve Model view. Tá na základe svojej vnútornej logiky vráti odpoveď s požadovanými údajmi. View model následne spracuje dodané dáta a predostrie ich view-u aby ich mohla ukázať užívateľovi.

View model zároveň, na základe aplikačnej logiky, koriguje a obmedzuje akcie užívateľa a tým zabraňuje vzniku nekonzistentných stavov aplikácie. Taktiež v niektorých prípadoch iniciuje interakcie s inými view modelmi za účelom dodania ich doplňujúcich služieb a aplikačnej logiky do jeho vlastného procesu.


Podobne ako vo View vrstve sa view modely delia natri základné typy:
\begin{itemize}
    \item \textbf{Session view model} + \textit{Main window view model} - odpovedajú jednotlivým \textit{session}-om, ktoré sú základným kameňom vertikálnej štruktúry aplikácie. Viac informácií o session-och je možné nájsť v sekcii \ref{Sessions}. Výnimkou je práve \textit{Main window view model}, ktorý je naviazaný na hlavné okno aplikácie a zabezpečuje preňho aplikačnú logiku. Klasické session view modely sú taktiež naviazané zvyčajne na jedno okno z view vrstvy pre ktoré zabezpečujú aplikačnú logiku.
    
    Každý session view model obsahuje kolekciu príslušných view model-ov ktoré spoločne implementujú mechanizmus daného session-u. Je zvykom, že v jednej chvíli je aktívny iba jeden view model. O aktívnom view modelu informuje session view model naviazané okno, ktoré potom v sebe zobrazuje odpovedajúci view aktívneho view modelu. Informovanie o aktívnom view modelu je aj hlavnou pracovnou náplňou session view modelu.

    K ďalším jeho povinnostiam patrí spracovávanie užívateľom vyvolaných akcií, ktoré sa týkajú samotného viazaného okna. Príkladom takej akcie môže byť požiadavka o zatvorenie dotyčného okna.
    \item \textbf{View model} - reprezentujú zložky, ktorých funkcionalita sa najväčšmi ponáša na obecnú, vyššie popísanú funkcionalitu vrstvy View model. Každý view model vo väčšine prípadov spadá pod réžiu konkrétneho session-u (alebo hlavného okna), pre ktorý implementuje určitú časť jeho mechanizmu. Klasicky sú view modely naviazané na príslušné view-y z View vrstvy. S tými následne reaktívne komunikujú a reagujú na ich podnety. View modely sú navzájom nezávislé. To znamená, že medzi nimi neprebieha takmer žiadna komunikácia ani presun dát. Túto funkciu na seba berie Model view vrstva.

    Väčšina view modelov by mala byť zahrnutá v zodpovedajúcom session view modele (poprípade Main window view modele). Ten potom zabezpečuje správu toho, ktorý view model je v danej chvíli aktívny. Výnimkou sú view modely, ktoré sú výhradne používane pre interakcie z iných view modelov, ktorým týmto spôsobom doručujú svoje služby. 
    
    Tieto view modely sú väčšinou vytvorené na mieste interakcie a po jej dokončení zanikajú. Pri ich vytvorení im sú zvyčajne predané nejaké vstupné parametre a keď je ich práca dokončená, tak vracajú jej výsledok.
    \item \textbf{Data view model} - slúžia ako kontajnery pre informácie abstrahované z dát vygenerovaných v Model vrstve. Dáta sú klasicky konvertované do ich zodpovedajúcich view modelov v Model view vrstve a už takto zabalené informácie sú predávané do View model vrstvy kde sú spracované a pomocou vlastností odovzdané vrstve View na vykreslenie. Data view modely môžu dodané informácie drobne upraviť takým spôsobom, aby ich bolo jednoduchšie vo View vrstve zobraziť.
    
    Z toho vyplýva, že na to aby nejaký údaj vygenerovaný v Model vrstve mohol byť prezentovaný užívateľovi, musí preňho existovať zodpovedajúci dátový view model. Zároveň na to, aby informácie obsiahnuté v dátovom view modele mohli byť vykreslené pre užívateľa, musí vo View vrstve existovať zodpovedajúci \textit{dátový template}, ktorý sa postará o ich správne grafické znázornenie.

    Niektoré dátové view modely nielen že obsahujú informácie príslušných dát ale obsahujú aj dáta samotné. Takéto dátové view modely označujeme pomocou slova \textit{wrapping}. (tvoria akýsi obal okolo dátových inštancií). Táto funkcionalita je dôležitá hlavne v prípadoch, kedy dátový view model slúži taktiež pre spätnú komunikáciu s Model view vrstvou. V takých prípadoch musí byť možné identifikovať, ktorú dátovú inštanciu \uv{obaluje}. Wrapping dátové view modely sú stotožnené s ich dátovým objektom a tiež sa pomocou neho identifikujú.  
\end{itemize}


View model je prvá z vrstiev, ktorá je kompletne písaná v jazyku C\#. Komunikácia medzi View model a View vrstvami funguje čisto na báze reaktívneho programovania za pomoci konštruktov z \textit{Reactive UI} framework-u.  

\subsection{Model view}

Treťou v poradí je, do klasickej MVVM architektúry pridaná, Model view vrstva. Táto vrstva je zodpovedná za \uv{vnútornú} logiku aplikácie. Priamo komunikuje s Model vrstvou a využíva jej zdroje pre zabezpečenie svojich služieb pre View model vrstvu. Dá sa povedať, že nedisponuje vlastným \uv{vedomím}. Medzi jej hlavné úlohy patria:  
\begin{itemize}
    \item prijímanie a spracovávanie požiadaviek od View modelu a dodávanie očakávaných výsledkov.  
    \item zabezpečovať vnútro-session-ovú komunikáciu. Model view-y v rámci jedného session-u si na seba držia referencie a predávajú si medzi sebou držané dáta. Táto komunikácia by mala byť vyšším vrstvam skrytá a na povrch by mal byť vidieť iba interface, pomocou ktorého prebieha komunikácia s View modelom.
    \item konverzia dát, získaných od Model vrstvy, do odpovedajúcich Data view modelov pri ich posielaní do vyšších vrstiev.  
\end{itemize}
Na druhú stranu medzi jej povinnosti nepatrí kontrola konzistentnosti jej vlastného stavu. O konzistenciu stavu aplikácie sa má starať View model.

V aplikácii používame model view-y dvoch typov:
\begin{itemize}
    \item \textbf{Session model view} + \textit{Main window model view}  - odpovedajú jednotlivým session-om. (Viac informácií o session-och je možné nájsť v sekcii \ref{Sessions}). 
    
    Ich hlavnou úlohou je vytvoriť a distribuovať model view-y odpovedajúce danému session-u. Pri ich inicializácii vytvorí medzi nimi väzby, ktoré sú následne počas behu aplikácie využívané na spomínanú vnútro-session-ovú komunikáciu. Taktiež zabezpečujú spracovávanie požiadaviek pre odpovedajúce session view modely. Tieto požiadavky sa typicky týkajú akcií, ktoré súvisia so session-om ako takým (nie s nejakou jeho časťou).  
    
    Výnimkou je práve \textit{main window model view}, ktorý je viazaný na view model hlavného okna a spracováva jeho požiadavky. V ostatných aspektoch je ale identický s klasickým session model view-om.
    
    Každému session model view-u odpovedá jeden konkrétny session view model. Ten pri svojej inicializácii predá model view-y, dodané v session model view-e, odpovedajúcim view modelom.   
    \item \textbf{Model view} - typ, ktorý nesie vyššie popísanú funkcionalitu Model view vrstvy. Zvyčajne pre každý view model existuje dedikovaný model view, ktorý sa stará o zabezpečenie view modelom požadovaných služieb. Nie je to však pravidlo, view model môže obsahovať odkazy na viacero model view-ov, ktorých služby následne využíva alebo sú predané vytvoreným, v interakciách využívaným, view modelom.

    Je zvykom, že každý model view spadá pod nejaký konkrétny session alebo hlavné okno. V takom prípade je daný model view vytváraný a distribuovaný odpovedajúcim session model view-om/main window model view-om.  
\end{itemize}

\subsection{Model}

Poslednou \uv{horizontálnou} vrstvou je Model. Model sa svojou štruktúrou diametrálne odlišuje od predchádzajúcich vrstiev. Je tvorený jednotlivými oblasťami, ktoré spravujú dedikovaní \textit{manažéri}. Manažéri sú pristupovaný z jednotlivých model view-ov a doručujú im svoje služby, či už informačné alebo výpočtové. Predstavujú interface-y ponúkajúce prívetivejší spôsob práce s vnútornými mechanizmami modelov. 

Model je predstaviteľom jedinej perzistentnej \uv{horizontálnej} vrstvy. Manažéri sú väčšinou singleton triedy, ktoré ponúkajú služby všetkým session-om počas celej doby behu aplikácie. Vďaka tomuto spôsobu obsluhovania je Model jediná vrstva, ktorej konštrukty niesu viazané na žiadnu vertikálnu vrstvu (session/hlavné okno). Z tohto návrhu Model vrstvy vyplýva ešte jedna dôležitá vlastnosť modelov a to, že musia byť schopné svoje služby dodávať paralelne pre viacero session-ov. 

Vrstva modelov je jednoducho rozšíriteľná o nových manažérov. Vďaka \textit{singleton} štruktúre sú dosiahnuteľný v podstate z akéhokoľvek miesta v programe a teda nemusia by5 nikde zahrnutí. Manažéri by zasa nemali mať problém prijímať nové, primerane vytvorené implementácie mechanizmov z ich oblastí. Napríklad by nemalo byť zložité dodať nový vyhľadávací algoritmus odpovedajúcemu manažérovi, ktorý ho následne bude ponúkať zvyšku aplikácie. %TODO mozno referencia na miesto, kde sa blizsie hovori o praci manazerov a jak si drzia kolekcie reprezentantov jednotlivych typov pouzitelnych v aplikacii.    

Špecifickým znakom komunikácie medzi Model a Model view vrstvami je, že pri nej dochádza ku strate typovej informácie dodávaných dát. Táto vlastnosť je motivovaná jednoduchým faktom, ktorým je udržanie vrstvy Model view jednoduchou. V Model vrstve sa totiž vo veľkom využívajú generiká pre jednoduché prenášanie typovej informácie v ich mechanizmoch. 

Využívanie generík v Model view vrstve by však prinieslo značné komplikácie v jej implementácii a tomu odpovedajúce zneprehladnenie kódu. Už len Model samotný trpí jemnou, generikami spôsobenou neprehľadnosťou. Z tohto dôvodu bolo rozhodnuté zabezpečiť jednoduchosť vrstvy Model view za cenu straty typovej informácie dát tečúcich z modelov do model view-ov. 

Pri opačnom smere komunikácie je manažérmi typová informácia dodaných parametrov opäť testovaná/získaná (väčšinou za pomoci tzv. \textit{generic visitor pattern}. Viac informácii o tejto modifikácii klasického \textit{visitor pattern} návrhového vzoru nájdete v % kde bude viac informa8cii...referencia na ne).  

V modelu existujú popri manažéroch ešte aj tzv. \textit{sub-manažéri}. Tieto entity sú ale určené pre využitie priamo z modelov. Podporujú generickú komunikáciu, na ktorej báze modely fungujú, bez straty typovej informácie a teda sú príjemnejšie pre modelovú komunikáciu než klasickí manažéri.     

\section{Session-y + \textit{hlavné okno}}\label{Sessions}

\subsection{Session-y}

Aplikácia, či už z vizuálneho, logického či implementačného hladiska, je rozdelená do tzv. \textit{session}-ov. Session-y sú najväčšie stavebné jednotky z ktorých každá predstavuje jedinečný mechanizmus dodávaný aplikáciou. Existencia ich inštancií je pominuteľná - vznikajú a zanikajú na popud užívateľa. Session-ov (aj rovnakého druhu) môže byť v aplikácii spustených viacero naraz. Každý session je klasicky tvorený odpovedajúcimi objektmi z prvých troch vrstiev horizontálnej štruktúry. 

Následne môžu session-y využívať dodatočné objekty z týchto vrstiev, ktoré patria hlavnému oknu alebo inému typu session-u. V takom prípade by malo byť ale poriadne rozmyslené, či takéto \uv{postranné} využitie dáva zmysel a či sa ním neporušujú zasady používania daného objektu. 

Poprípade je možné využívať špecifické model viewy a view modely prostredníctvom interakcií. V takom prípade by ale dané objekty mali byť na daný účel prispôsobené (info v podsekcii \ref{ViewModel}, bod \textbf{View model}, 3. odsek).

Session-y sú klasicky rozdelené na logické časti, ktoré spolupracujú na dodaní požadovaného mechanizmu. Môžu napríklad reprezentovať fázy jeho procesu. Tieto časti sú väčšinou tvorené špecifickými objektmi naprieč prvými tromi vrstvami horizontálnej štruktúry. 

Session-ové inštancie by medzi sebou nemali navzájom komunikovať ani zdielať svoje dáta. Mohlo by to viesť ku problémom s paralelizáciou služieb vykonávaných Model vrstvou.

Session-y ako také reprezentujú cestu ku všeobecnej rozšíriteľnosti aplikácie o nové mechanizmy. Ak by vznikla potreba, aby aplikácia obsahovala nejaký nový mechanizmus, stačí preňho vytvoriť odpovedajúci session a upraviť hlavné okno tak, aby ho vedelo ponúknuť užívateľovi ako jednu z možností.  

\subsection{Hlavné okno}\label{Hlavne_okno_obecne}

Session-y sú vytvárané a spravované \textit{hlavným oknom}. Je to jediná perzistentná \uv{vertikálna} vrstva aplikácie. Beh aplikácie začína s otvorením tohto okna a končí jeho zatvorením. Hlavné okno môže byť používané počas celého behu aplikácie. Pokiaľ je vydaný pokyn na uzavretie hlavného okna, pričom sú stále živé nejaké session-y, užívateľ môže byť na tento fakt upozornený. Ak mu to ale neprekáža, zatvorením hlavného okna sa zavrú aj všetky ostatné a aplikácia sa ukončí. Logika session-ov by mala túto skutočnosť brať do úvahy.

Horizontálna štruktúra je veľmi podobná tej session-ovej. Taktiež využíva MVVM(MV) architektúru. Jej časti sú ale z podstaty hlavného okna vytvárané len raz pri štarte aplikácie a zanikajú pri jej ukončení. \textit{Session view model} a \textit{Session model view} sú nahradené za funkčne identické \textit{Main window view model} a \textit{Main window model view}. Podobne ako pri session-och, aj hlavné okno je rozdelené do niekoľkých častí. Tie si medzi sebou rozdeľujú jeho funkcionalitu. Niektoré z jeho častí môžu byť sprístupnené rôznym session-om, aby si z nich mohli vytiahnuť potrebné parametre platiace pre celú aplikáciu.     

Ako bolo spomenuté na začiatku tejto podsekcie, hlavné okno vytvára, eviduje a spravuje inštancie všetkých session-ov. Definuje maximálny počet otvorených okien, spravuje session-y pri ukončovaní aplikácie, poskytuje dodávateľa hlavných parametrov, inicializuje pri vytváraní session-u jeho view modely a model viewy.  

\begin{figure}[p]\centering
\includegraphics[]{img/priklad_struktury}
\caption{Príklad možnej \uv{mriežkovej} štruktúry aplikácie.} 
\label{obr02:priklad_struktury}
\end{figure}

\chapter{Súčasná podoba vertikálnej štruktúry aplikácie}

V tejto kapitole uvádzame bližší pohľad na súčasnú podobu hlavného okna a session-ov aplikácie. Aktuálne aplikácia disponuje len jedným typom session-u určeným pre samotné vyhľadávanie ciest v mapách (\textit{Path finding session}). V pláne bolo taktiež vytvoriť session určený na vytváranie užívateľských modelov ale z časových dôvodov nakoniec nebol zahrnutý. Pre obecných informácií o vertikálnej štruktúre aplikácie je možné nájsť v sekcii \ref{Sessions}.

\section{Hlavné okno}

O obecných úlohách hlavného okne sme sa zmienili už v podsekcii \ref{Hlavne_okno_obecne}. V tejto sa pozrieme na aktuálne implementovanú architektúru ako aj funkcionalitu hlavného okna.


Hlavné okno je tvorené dvomi časťami: hlavným menu a hlavnými nastaveniami.
Popri tom ešte využíva v hlavných nastaveniach interaktívne služby mechanizmu pre správu výškových dát.

Ako už bolo spomenuté v podsekcii \ref{Hlavne_okno_obecne}, po zatvorení hlavného okna sa ukončuje aj samotná aplikácia. Tým pádom je úlohou hlavného okna zaistiť uloženie všetkých parametrov aplikácie pre využitie v jej budúcich behoch.

\subsection{Hlavné menu}

Hlavné menu je prvá časť, ktorá sa v hlavnom okne užívateľovi po zapnutí aplikácie zobrazí. Obsahuje možnosti vytvorenia inštancií session-ov a možnosť otvorenia hlavných nastavení aplikácie. 

View model hlavného menu si vedie evidenciu všetkých otvorených session-ov. Na popud zatvorenia hlavného okna sa stará o ich správne ukončenie. Taktiež si drží referenciu na sprostredkovateľa hlavných nastavení. Toho následne môže predať session-om, ktorý ho môžu použiť nastavení platných pre celú aplikáciu.

Zvláštnosťou tejto časti je absencia vlastného model view-u. V aktuálnej podobe aplikácie totiž nie je potrebný, nakoľko hlavné menu nepotrebuje komunikovať so žiadnym modelom. Ak by táto potreba v budúcnosti vznikla, nemal by byť problém model view pre hlavné menu doimplementovať. 

Možné vylepšenie tejto časti by mohlo zahrňovať vypísanie všetkých aktuálne otvorených session-ov pre užívateľa. Zaistilo by to preňho jednoduchšiu orientáciu.

\subsection{Hlavné nastavenia}

Sú druhou časťou hlavného okna. Zabezpečujú možnosť pre užívateľa nastaviť parametre, ktoré sú následne aplikované na celú aplikáciu. 

V aktuálnom stave sú to len dve možnosti konfigurácie: možnosť zmeny lokalizácie aplikácie a možnosť nastavenia implicitne používaného zdroja výškových dát (presnejšie konkrétnej distribúcie výškových dát z daného zdroja). 

Lokalizácia je implementovaná na základe lokalizačných .resx zdrojových súborov. Avalonia je schopná takéto zdroje využívať na zmenu pevných nápisov aplikácie.

Zdroj výškových dát je nastavovaný za pomoci interakcie s mechanizmom konfigurácie výškových dát. Po stlačení príslušného tlačidla sa v hlavnom okne zobrazí view zodpovedajúci tomuto mechanizmu. Užívateľ si v ňom môže vybrať, ktorú distribúciu dát chce implicitne v aplikácii využívať. Viac informácií ohľadom konfigurácie výškových dat nájdete v následujúcej podsekcii.

Do budúcna sa počíta s rozšírením hlavných nastavení do takej miery, aby sa z nich dali konfigurovať aj rôzne preferencie v Model vrstve. Pre toto rozšírenie by sa však museli upraviť modeloví manažéri, aby tieto konfigurácie dokázali prijímať.

\subsection{Konfigurácia výškových dát}\label{konfiguracia_vyskovych_dat}

Konfigurácia výškových dát ma špecifický spôsob využitia. Je určená na to byť otváraná pomocou interakcie vytvorenej inou časťou aplikácie. Jej hlavnou funkciou je dodávanie mechanizmu sťahovania a mazania výškových dát, ktoré je možné využiť ako dodatočný informačný zdroj pri vytváraní mapových reprezentácií. 

Výškové dáta sú sťahované zo špecifických zdrojov. Zdroje výškových dát môžu obsahovať viacero dátových distribúcií. Tie sa môžu líšiť v kvalite, presnosti či dostupnosti dodávaných výškových dát. 

Manipulácia s dátami je vykonávaná po oblastiach zvaných \textit{regióny}. Veľkosť a tvar regiónov si každá distribúcia dát definuje sama. 

V niektorých prípadoch zdroj výškových dát môže vyžadovať k ich sprístupneniu autorizáciu užívateľa pomocou mena a hesla. Táto autorizácia môže byť požadovaná iba pri niektorých jeho dátových distribúciach. V takom prípade mechanizmus zabezpečuje možnosť nechať užívateľa dané údaje poskytnúť. Následne sa aplikácia pokúsi na ich základe požadované dáta získať.   

Sťahovanie a mazanie dát prebieha asynchrónne. Užívateľ je informovaný o tom, ktoré regióny su aktuálne stiahnuté, sťahované, neprítomne a mazané. Viacero regiónov môže byť sťahovaných naraz, či už z jednej distribúcie výškových dát alebo z rôznych. Správne asynchrónne fungovanie manipulácie s dátami zaručujú implementácie zdrojov výškových dát v Model vrstve.

Špecifickou vlastnosťou konfigurácie výškových dát je možnosť ju používať súčasne z viacerých miest v aplikácii. Je navrhnutá tak, aby zvládala korektne akceptovať pokyny z mnohých interakcií naraz. Môže za to špecificky navrhnutý model view, ktorý drží informácie o tom, v akom štádiu sťahovania sú jednotlivé regióny. Jeho jediná inštancia je následne využívaná vo viacerých interakciách súčasne. Tým pádom sa do nich dostanú všetky potrebné informácie na to, aby v bolo možné správne korigovať manipuláciu s dátami.

Proces používania konfigurácie je nasledovný:
\begin{itemize}
    \item Pri inicializácii interakcie je novo vytvorenému view modelu(následne použitému v interakcii) predaná aktuálne využívaná distribúcia výškových dát.
    \item Tá je v konfigurácii nastavená ako aktuálne konfigurovaná a ukázaná pomocou view-u užívateľovi.
    \item Následne prichádza fáza samotného konfigurovania výškových dát. Užívateľ môže:
    \begin{itemize}
        \item Zmeniť aktuálne konfigurovanú distribúciu výškových dát.
        \item Sťahovať a mazať výškové dáta aktuálne vybranej distribúcie výškových dát. Tieto úkony sú uskutočňované na základe regiónov definovaných konfigurovanou distribúciou.
        \item Zadať autorizačné údaje pre možnosť využitia dát zo špecifických distribúcií, ktoré autorizáciu vyžadujú. 
        \item Ukončiť konfiguračnú interakciu.
    \end{itemize} 
    \item Pri ukončení interakcie sa posledne konfigurovaná distribúcia vráti ako novo zvolená na používanie.
\end{itemize}

\section{Session pre vyhľadávanie ciest v mapách}

Vyhľadávanie ciest v mapách je hlavnou náplňou tejto aplikácie. V tejto sekcii popíšeme typ session-u, ktorý zabezpečuje mechanizmus pre doručenie tejto služby. Mechanizmus hľadania cesty v mapách zahrnuje:
\begin{itemize}
    \item výber vstupných parametrov tak aby ich kombinácia bola validná 
    \item vytvorenie grafickej reprezentácie mapy
    \item vytvorenie mapovej reprezentácie, v ktorej sa bude hladat cesta   
    \item umožnenie užívateľovi zadať trať, na ktorej sa má cesta vyhľadať
    \item spustenie implementácie vybraného vyhľadávacieho algoritmu na zvolenej trati a vykreslenie nájdenej cesty 
\end{itemize} 
V aktuálnej podobe sa mechanizmus vyhľadávania cesty skladá z troch častí: 
\begin{itemize}
    \item Nastavenie parametrov hľadania cesty (template, mapa, užívateľský model a vyhľadávací algoritmus).
    \item Vytvorenie mapovej reprezentácie na základe vybranej mapy a template-u. Prípadne za pomoci prítomných výškových dát. Táto časť je navrhnutá pre použitie v interakcii. Interakciu iniciuje vyššie uvedená časť. 
    \item Spúšťanie samotného vyhľadávania na vytvorenej mapovej reprezentácii za využitia vybraného užívateľského modelu. Zahrňuje prijímanie trate od užívateľa a vykreslovanie nájdenej cesty.
\end{itemize} 

Časť, ktorá sa z časových dôvodov nedostala do mechanizmu hľadania cesty je tzv. \textit{relevance-feedback} mechanizmus. Ten by slúžil pre užívateľa na dodatočné nastavenie hodnôt vybraného užívateľského modelu na základe jeho preferencií vzhľadom ku aktuálne vybranej mape. Z tejto časti zostal v aplikácii len odpovedajúci model view, ktorý v tejto chvíli nenesie žiadnu užitočnú funkcionalitu. Slúži iba pre prenos dát v Model view vrstve. Bol v aplikácii ponechaný z dôvodu plánovaného budúceho rozšírenia aplikácie o relevance-feedback mechanizmus.

V následujúcich podsekciách popíšeme časti, z ktorých je aktuálne  aplikácia tvorená.

\subsection{Nastavenia parametrov pre vyhľadávanie cesty}

Nastavenia parametrov sú prvou časťou, ktorá je užívateľovi po vytvorení session-u predostretá. Ten si za jej pomoci zvolí:

\begin{itemize}
    \item template atribútov, ktoré budú extrahované do mapovej reprezentácie, 
    \item mapový súbor, na základe ktorého sa bude vytvárať mapová reprezentácia. Teda súbor s mapou, na ktorej bude prebiehať vyhľadávanie ciest.
    \item súbor s užívateľským modelom, ktorý bude používaný algoritmom na agregáciu hodnôt z atribútov uložených v mapovej reprezentácii  
    \item vyhľadávací algoritmus, ktorý bude použitý na samotné hľadanie ciest v mapovej reprezentácii
\end{itemize}

Na začiatku sa predvolia parametre na posledne použité v predchádzajúcom cestu-hľadajúcom session-e. Tieto uložené parametre prežívajú aj život aplikácie samotnej a teda môžu sa načítať aj posledne použité parametre z predošlých behov aplikácie.

Vyberanie parametrov sa musí riadiť istými pravidlami. To z dôvodu závislostí jednotlivých typov objektov popísaných v podsekciách sekcie \ref{Aspekty_hladania}. Danými pravidlami sú:
\begin{itemize}
    \item Jednotlivé parametre sa nastavujú postupne. Najprv je nutné, aby bol vybraný mapový súbor a template. Následne môže byť vybraný aj súbor s užívateľským modelom. Keď sú vybraný všetky tri predchádzajúce položky je možné vybrať vyhľadávací algoritmus. 
    \item Pre zvolenú kombináciu template-u a formátu mapového súboru musí existovať mapová reprezentácia, ktorá túto kombináciu dokáže spracovať. Taktiež pre zvolený template-u musí existovať typ užívateľského modelu, ktorý dokáže spracovávať atribúty definované týmto template-om. Nakoniec musí existovať aspoň jedna kombinácia takto definovanej mapovej reprezentácie a užívateľského modelu, ktorú dokáže využiť aspoň jedna implementácie nejakého vyhľadávacieho algoritmu.   
    
    Vždy, keď je jedna z týchto položiek vybraná a výber druhej by spôsobil neplatnú kombináciu, prvá položka sa opäť vynuluje.
    \item Súbor s užívateľským modelom môže byť vybraný len takého typu, ktorý dokáže spracovávať atribúty definované zvoleným template-om a ktorý spolu s ľubovoľnou mapovou reprezentáciou, ktorá dokáže spracovať aktuálne zvolenú kombináciu template-u a mapového formátu, je vhodnou kombináciou pre aspoň jednu implementáciu nejakého vyhľadávacieho algoritmu. 
    \item Vyhľadávací algoritmus následne môže byť zvolený len taký, ktorý podporuje typ vybraného užívateľského modelu spolu s nejakou mapovou reprezentáciou vytvorenou na základe zvoleného template-u a mapy. 
\end{itemize}    

Po výbere mapového súboru sa ihneď z agregovanej mapy vytvori jej grafická reprezentácia a jej ukážka sa zobrazí pre užívateľa.

Po dosadení všetkých parametrov môže užívateľ pokračovať do ďalšej časti mechanizmu, ktorou je vytváranie mapovej reprezentácie. V okamžiku prechodu do tejto časti sa taktiež uložia aktuálne nastavené parametre, aby mohli byť znovu použité ako predvolené v následujúcich behoch tohto session-u. 

Časť vytvárania mapovej reprezentácie je prevedená za pomoci interakcie z aktuálnej časti nastavení. Na základe jej výsledku sa následne buď presunieme v mechanizme ďalej do cestu-vyhľadávacej časti (úspešné vytvorenie) alebo zostaneme v nastaveniach (neúspešné vytvorenie).  

\subsection{Vytváranie mapovej reprezentácie}

Vytváranie mapovej reprezentácie je akási prechodová časť medzi nastaveniami a samotným vyhľadávaním ciest v mape. Je prevedená za pomoci interakcie iniciovanej v nastaveniach po dosadení všetkých potrebných parametrov. Táto interakcia je spracovaná pomocou dialógového okna. Pomocou neho môže užívateľ pozorovať priebeh tvorby mapovej reprezentácie a aktívne sa zapájať pri riešení jej problémov.  

Priebeh tvorby mapovej reprezentácie je nasledovný:
\begin{itemize}
    \item Ihneď po inicializácii tejto časti sa spustí proces kontroly podmienok tvorby mapovej reprezentácie. Tieto podmienky môžu byť akéhokoľvek zamerania. Aktuálne je implementovaný jediný typ podmienky a to na kontrolu prítomnosti potrebných výškových dát pri procese tvorenia mapovej reprezentácie. (Táto kontrola zahrňuje aj schopnosť mapy informovať o svojej pozícii a rozlohe).
    
    Pokiaľ tvorená mapová reprezentácia indikuje potrebu výškových dát, skontroluje sa či tieto dáta sú prítomné vzhľadom na polohu a rozlohu používanej mapy. Pokiaľ dáta prítomné niesu alebo mapa nie je schopná definovať svoju geografickú polohu alebo rozlohu, vytváranie mapovej reprezentácie zlyhá a užívateľ sa môže vrátiť do nastavení. 
    
    Do budúcna je v pláne umožniť užívateľovi z miesta riešenia problému nedostatku výškových dát ich konfiguráciu za pomoci interakcie. Viac informácii o konfigurácii výškových dát v podsekcii \ref{konfiguracia_vyskovych_dat}.
    \item Pokiaľ kontrola podmienok dobehne úspešne, je automaticky spustený proces vytvárania mapovej reprezentácie. Tento proces môže zabrať dlhšiu dobu a teda je užívateľovi umožnené sledovať jeho vývoj a dokonca ho prerušiť. Po prerušení je užívateľ vrátený naspäť do nastavení.
\end{itemize}

\subsection{Vyhľadávanie cesty v mape}

Po úspešnom vytvorení mapovej reprezentácie sa môžeme presunúť k samotnému vyhľadávaniu ciest v mape. Máme už totiž k dispozícii všetky potrebné dátové zdroje k správnemu vykonávaniu tejto činnosti.

Na začiatku je vykreslená grafika mapy, ktorá bola vytvorená v nastaveniach pri výbere mapového súboru.

Kolobeh vyhľadávania je rozdelený do troch fáz:
\begin{itemize}
    \item Prvou fázou je výber trate užívateľom. Trať sa skladá z postupností bodov medzi ktorými je následne vyhľadávaná cesta. Užívateľ môže pridávať a odoberať body na konci trate.
    \item Po zadaní cesty prichádza na rad fáza samotného vyhľadávanie cesty. Algoritmus má možnosť reportovať postup vyhľadávania, či pomocou textovej informácie alebo grafického znázorňovania. Užívateľ mám možnosť proces vyhľadávania zrušiť. V takom prípade sa kolobeh vráti do prvej fázy výberu trate.
    \item Pokiaľ vyhľadávania cesty dobehne úspešne príde narad tretia fáza ktorou je vykreslenie nájdenej cesty. Zároveň sa po strane môžu vypísať informácie o nájdenej trase (napríklad jej dĺžka). V tejto časti by mala byť aj možnosť pre akúsi interakciu s nájdenou cestou. Po dokončení prezerania nájdenej cesty sa opäť vrátime do prvej fáze výberu trate.  
\end{itemize}

Počas ktorejkoľvek fázy je užívateľ schopný približovať, odďalovať a hýbať s vykreslenou grafikou mapy.

% \subsection{Zaujímavé implementačné postupy} % % V tejto podsekcii vymenujeme pár zaujímavých postupov využitých pri implementácii \textit{hlavného okna}.  % \begin{itemize} % \item \textbf{Zatváranie hlavného okna} - zatváranie hlavného okna je špecifické tým, že ukončuje beh celej aplikácie. Preto je v niektorých prípadoch potrebné, aby bolo možné sa opýtať užívateľa, či si je istý svojou požiadavkou na ukončenie aplikácie. Ideálnym spôsobom na zistenie užívateľovho názoru je použitie dialógového okna. Ač sa to môže zdať ako triviálna úloha, s Avalonia framework-om to zas tak jednoduché nebolo.  % % Následujúci kód ukazuje nefunkčný príklad \texttt{MainWindow\_OnClosing} metódy z triedy \texttt{MainWindow} určenej na zachytávanie a spracovanie udalosti zatvárania hlavného okna: % % \begin{lstlisting} % private async void MainWindow_OnClosing % (object? sender, WindowClosingEventArgs e) % { % bool close = % await ViewModel!.OnClosingCommand.Execute(); % if (!close) % { % e.Cancel = true; % } % } % \end{lstlisting} % 
    % Na prvý pohľad by sa mohlo zdať že je všetko v poriadku. Pokiaľ spustený \texttt{ViewModel.OnClosingCommand} vráti indikáciu toho, že sa okno nesmie zavrieť, nastaví sa vlastnosť \texttt{Cancel} na true a tým sa zabráni zatvoreniu hlavného okna.
    % 
    % Problém je však v tom, že spustenie daného príkazu na view modelu prebieha asynchrónne z dôvodu možnosti otvorenia dialógového okna na komunikáciu s užívateľom. Výsledok príkazu je potrebné očakávať za pomoci kľúčového slova \texttt{await}, nakoľko jeho použitie zabezpečí, že UI thread zostane aktívne a reagujúce.
    % 
    % To však na druhú stranu sa stáva problémom, nakoľko UI spracuje argument \texttt{e} prv než naša metóda dokáže správne dosadiť jeho vlastnosť \texttt{Cancel}. Teda okno sa zavrie skorej, než je užívateľovi daná možnosť to zvrátiť.
    % 
    % Preto bol použitý následujúci návrh metódy, ktorý zaručuje správny spôsob zatvárania hlavného okna:
    % \begin{lstlisting}
    % private bool _alreadyAsked = false;
    % private async void MainWindow_OnClosing
        % (object? sender, WindowClosingEventArgs e)
    % {
        % if (_alreadyAsked) return;
        % e.Cancel = true;
        % bool close = 
            % await ViewModel!.OnClosingCommand.Execute();
        % if (close)
        % {
            % _alreadyAsked = true;
            % Close();
        % }
    % }
    % \end{lstlisting}
% 
    % V tomto prípade sme predošlý problém vyriešili malým trikom. Vždy keď zaznamenáme udalosť zatvárania okna iniciovanú užívateľom, nastavíme automaticky príznak \texttt{Cancel} argumentu \texttt{e} na true. Tým zabránime predčasnému zatvoreniu okna. 
    % 
    % Následne podobne ako v predošlom prípade asynchronne zavoláme spustenie \texttt{ViewModle.OnClosingCommand} a počkáme na výsledný indikátor. Ak indikuje pokračovanie v zatváraní aplikácie, príde na radu náš trik.  
% 
    % Nastavíme hodnotu, k tomuto špecifickému účelu vytvoreného, privátneho poľa \texttt{\_alreadyAsked} na true. Toto pole indikuje, že sa hlavné okno už raz užívateľa pýtalo na jeho názor na zatvorenie aplikácie a že jeho odpoveď bola pozitívna. 
    % 
    % Po nastavení poľa je opäť zavolaná metóda Close() na hlavnom okne. Na základe tohto volania sa opäť hlavné okno pokúsi zavrieť. Tým pádom sa znova zavolá metóda \texttt{MainWindow\_OnClosing}. Tentokrát sa však jej beh zastaví hneď na začiatku na dotaze, či pole \texttt{\_alreadyAsked} indikuje už zistený užívateľov súhlas so zavretím okna. Tým pádom sa vlastnosť \texttt{Cancel} nestihne nastaviť na true a teda hlavné okno sa následne zavrie.    
% 
    % Tento princíp je možné využiť taktiež v ostatných oknách aplikácie, ak by mali potrebu rovnakého mechanizmu ich zatvárania. 
    % \item \textbf{Súbežné využívanie inštancie \texttt{ElevDataModelView}-u} - za možnosťou súbežného využívania model view-u konfigurácie výškových dát je malý trik. Na začiatok je potrebné si uvedomiť, ktoré dáta je potrebné držať synchrónne vo všetkých používaných konfiguráciách výškových dát a ktoré na druhú stranu majú byť pre každú konfiguráciu jedinečné:
    % \begin{itemize}
        % \item Synchrónne je potrebné držať informácie o stave prítomnosti jednotlivých regiónov všetkých distribúcií. Keby sa v týchto dátach objavila akákoľvek nesúmernosť, mohlo by to mať fatálne dôsledky na manipuláciu s výškovými dátami.
        % \item Na druhú stranu každá konfigurácia by si mala sama určovať, ktorá distribúcia výškových dát je aktuálne konfigurovaná. Predsa len to je jedným zo zámerov procesu konfigurácie. Nechať užívateľa vybrať ním žiadanú distribúciu na použitie či upravenie.
    % \end{itemize}
% 
    % Teda potrebujeme, aby inštancia \texttt{ElevDataModelView} v sebe držala informáciu o prítomnosti jednotlivých regiónov pre všetky distribúcie. Pre konkrétny región je táto informácia uložená v data view modele, do ktorého je príslušný región zabalený (viac informácií ku data view modelom nájdete v podsekcii \ref{ViewModel}). Samotný región si nesie informáciu len o tom, či sú preňho dáta stiahnuté alebo nie. Neinformuje ale o tom, čí sú dáta aktuálne sťahované alebo odstraňované.
% 
    % Typicky model view-y vytvárajú vždy nový data view model pre každý údaj získaný z Model vrstvy v momente posúvanie jeho informácie do vrstvy View model. Tu však prichádza na rad malý trik, kedy \texttt{ElevDataModelView} vytvorí data view modely pre všetky existujúce regióny počas svojej inicializácie a uloží ich do slovníka \texttt{TopRegionsOfAllDistributions}. Následne je tento slovník ponúkaný všetkým inštanciám triedy \texttt{ElevConfigViewModel} a teda všetky tieto inštancie pracujú s jednými a tými istými objektmi regiónových data view modelov.
    % 
    % Tým pádom všetky inštancie triedy \texttt{ElevConfigViewModel} zdieľajú informácie o stave prítomnosti jednotlivých regiónov a teda nemôže nastať nekonzistencia v akciách manipulujúcich s výškovými dátami. (samozrejme za predpokladu, že užívateľ nie je schopný stlačiť dve tlačidlá v rôznych konfiguračných oknách v ten istý moment. Pri klasickom používaní aplikácie by tento scenár nikdy nemal nastať).
% 
    % Na druhú stranu každá inštancia triedy \texttt{ElevConfigViewModel} si sama drží informáciu o aktuálne konfigurovanej distribúcii výškových dát a teda v každej z týchto inštancií môžem v rovnakom čase konfigurovať inú distribúciu.
% \end{itemize}










% \subsection{Zaujímavé implementačné postupy}
% 
% V tejto podsekcii vymenujeme pár zaujímavých postupov využitých pri implementácii \textit{session-u pre vyhľadávanie ciest v mapách}.  
% 
% \begin{itemize}
    % \item \textbf{Spôsob zabezpečenia vnútornej komunikácie model view-ov} - Jednou z náplní Model view vrstvy je zabezpečovať vnútro-session-ovú komunikáciu. Nikto okrem samotných model view-ov by nemal mať k tejto komunikácii prístup. Okolie by mal mať možnosť vidieť iba metódy, ktorými Model view vrstva sprístupňuje svoje služby. 
    % 
    % Následujúca implementácia zaručuje ukrytie vnútornej komunikácie model view-ov cesty-vyhľadávacieho session-u. Môže slúžiť ako príklad pre implementáciu Model view vrstvy ostatných session-ov. 
% 
    % Model view inštancie pre daný session inicializuje \textit{session model view}. Ten ich následne aj distribuuje do ostatných potrebných častí session-u. 
    % 
    % Chyták však je, že vytvárané inštancie niesu typu, ktorý je navonok prezentovaný. Session model view definuje privátnych potomkov týchto prezentovaných typov, ktorí majú, narozdiel od ich predkov, implementovanú funkcionalitu vzájomnej komunikácie. Taktiež override-ujú všetku funkcionalitu predkov, v ktorej je potrebná práca s dátami prúdiacimi vo vnútornej komunikácii.
% 
    % Tým že sú tieto \uv{intra} triedy definované ako privátne, nie je možné mimo name-space-u session model view-u sledovať ich komunikáciu.
% 
    % \begin{listing}[h]
    % \begin{lstlisting}
    % public class PathFindingSessionModelView : SessionModelView
    % {
        % public PFSettingsMV Settings { get; }
        % public PFMapRepreCreatingMV MapRepreCreating { get; }
        % public PFRelevanceFeedbackMV RelevanceFeedback { get; }
        % public PFPathFindingMV PathFinding { get; }
        % 
        % public PathFindingSessionModelView()
        % {
            % var setI = new PFSettingsIntraMV();
            % var graCreI = new PFMapRepCreIntraMV(setI);
            % var relFeeI = 
                % new PFRelevanceFeedbackIntraMV(setI, graCreI);
            % var patFinI = new PFPathFindingIntraMV(relFeeI);
% 
            % Settings = setI;
            % MapRepreCreating = graCreI;
            % RelevanceFeedback = relFeeI;
            % PathFinding = patFinI;
        % }
        % private class PFSettingsIntraMV : PFSettingsMV        
        % private class PFMapRepreCreatingIntraMV : 
            % PFMapRepreCreatingMV
        % private class PFRelevanceFeedbackIntraMV : 
            % PFRelevanceFeedbackMV
        % private class PFPathFindingIntraMV : PFPathFindingMV
    % }
% 
    % public abstract class PFSettingsMV : ModelViewBase { }        
    % public abstract class PFMapRepreCreatingMV : 
        % ModelViewBase { }        
    % public abstract class PFRelevanceFeedbackMV : 
        % ModelViewBase { }
    % public abstract class PFPathFindingMV : ModelViewBase { }      
    % \end{lstlisting}
    % \caption{Príklad návrhu session model view-u, ktorý skrýva vnútornú komunikáciu (upravený \texttt{PathFindingSessionModelView})}
    % \end{listing}
% 
% \end{itemize}
\chapter{Architektúra vrstvy Model}\label{architektura_model_vrstvy}

Model vrstva je štvrtou vrstvou MVVM(MV) architektúry. Táto vrstva je pre~zvyšné vrstvy zdrojom dát a~mechanizmov, ktoré~tieto dáta spracovávajú. Delí sa do~mnohých oblastí. Pre~komunikáciu s~týmito oblasťami sú vytvorení takzvaní \textit{manažéri}. Manažéri sú pristupovaný z~Model view vrstvy a~doručujú jej svoje služby, či~už informatívne alebo~výpočtové. Viac informácií o~vrstve Model je možné nájsť v~Podsekcii~\ref{model}.

Implementáciu Model vrstvy z~veľkej časti poznačila už spomínaná strata typovej informácie pri~komunikácii s~Model view vrstvou. Väčšina oblastí sa s~touto stratou vysporadúva podobným spôsobom:
\begin{itemize}
    \item \textbf{Komunikácia smerujúca z~vrstvy Model} - Pre~všetky dátové štruktúry, ktoré~sa využívajú mimo vrstvy Model, sú vytvorení predkovia, ktorí sú buď negenerickí alebo~obsahujú kovariantné generické parametre. Tým pádom vedia byť prenášaní negenerickým spôsobom mimo vrstvy Model.
    \item \textbf{Komunikácia smerujúca do~vrstvy Model} - Pre~dátové štruktúry, ktorých typovú informáciu je potrebné v~Model vrstve opäť nadobudnúť, bol vytvorený tzv. \textit{Generic visitor pattern}. V~prípadoch, kedy nie je potrebné poznať ich presný typ, sa využíva klasické typové testovanie cez operátor \textit{is}.
\end{itemize} 

\textbf{Generic visitor pattern} je obdoba klasického návrhového vzoru \emph{Visitor pattern}. Pracuje na~podobnom princípe, kedy je na~navštevovanej inštancii volaná metóda \texttt{Visit} ktorej sa predá inštancia volajúcej triedy. Následne navštívená inštancia odpovie zavolaním metódy \texttt{Accept} na~dodanej volajúcej inštancii a~predaním samej seba v~argumente indikuje, ktorý~overload metódy \texttt{Accept} sa má zavolať.

V~prípade generic visitor pattern-u však volajúca trieda neimplementuje overload metódy \texttt{Accept} pre~každý možný typ navštevovanej inštancie ale~len jednu generickú metódu \texttt{Accept<T>}. V~typovom parametri \texttt{T} je následne predaná informácia o~type navštívenej inštancie.

\bigskip

V~následujúcich sekciách sa pozrieme detailnejšie na~aktuálne existujúce oblasti vrstvy Model a~ich vnútorné mechanizmy.

\section{Template-y}

Táto oblasť sa stará o~správu atribútových template-ov (len \textit{template-ov} pre~jednoduchosť). Viac informácií o~myšlienke a~funkcii template-ov je možné nájsť v~Podsekcii~\ref{templatey}. Čo sa obsahu týka, je to jedna z~najmenších oblastí.  

Hlavným uzlom pre~komunikáciu z~Model view vrstvy je singleton trieda \texttt{TemplateManager}. Ten ponúka kolekciu všetkých template-ov, ktoré~je možné v~aplikácii použiť. 

V~následujúcich odsekoch popíšeme objektovú štruktúru template-ov. Jednoduché grafické znázornenie tejto architektúry je dostupné k nahliadnutiu v~Diagrame~\ref{obr05:templatey_architektura}.   

\bigskip

Template-y sú reprezentované pomocou tried, ktoré~implementujú rozhranie \textbf{\texttt{ITemplate<TVertexAttributes, TEdgeAttributes>}}. Tento generický interface núti svoje implementácie, aby~dosadením jeho typových parametrov indikovali typy vrcholových a~hranových atribútov ktoré~reprezentujú. 

Ďalej v~aplikácii existuje ešte aj~negenerický predok \textbf{\texttt{ITemplate}} spomenutého rozhrania, ktorý~slúži na~komunikáciu mimo vrstvy Model. Definuje vlastnosti, ktoré~sú potrebné pri~práci s~template-ami vo~vonkajšom prostredí. Tento interface by nemal byť nikdy priamo implementovaný.

Template-ové triedy by mali byť implementované ako singleton-y. V~aplikácii je totiž vždy využívaná len jedna inštancia každého template-ového typu a~to tá zahrnutá v~kolekcii template-ov v~triede \texttt{TemplateManager}.

Template-y podporujú návrhový vzor \textit{generic visitor pattern}. Ten sa svojou funkcionalitou trocha odlišuje od ostatných implementácií. Metóda \texttt{Visit} je totiž definovaná v~rozhraní \texttt{ITemplate}, ale~metóda \texttt{Accept} vracia typový parameter, ktorý~je obmedzený na~rozhranie \texttt{ITemplate<TVertexAttributes, TEdgeAttributes>}. Tento trik slúži k~tomu, aby~aj na~premennej typu \texttt{ITemplate} bolo možné ihneď získať typy vrcholových a~hranových atribútov daného template-u. Tento trik funguje na~základe predpokladu, že~všetky template-ové triedy implementujú rozhranie \texttt{ITemplate<TVertexAttributes, TEdgeAttributes>}. 

\begin{figure}[h]\centering
\includegraphics[]{img/templatey_architektura}
\caption{Diagram popisujúci objektovú štruktúru template-ov.} 
\label{obr05:templatey_architektura}
\end{figure}

\section{Mapy}

Táto oblasť sa stará o~vytváranie a~správu mapových objektov. Viac informácií o~koncepte Máp je možné nájsť v~Podsekcii~\ref{mapy}.

Hlavným uzlom pre~komunikáciu z~Model view vrstvy je singleton trieda \texttt{TemplateManager}. Tá ponúka kolekciu mapových formátov, ktorá obsahuje zástupcov všetkých mapových typov, ktoré~je možné v~aplikácii využiť. Každý zástupca zastupuje jeden typ mapy, jeden formát. Popri tom ponúka metódy pre~vytváranie mapových inštancií a~metódy pre~identifikáciu mapových formátov. 

V~následujúcich odsekoch popíšeme objektovú štruktúru máp a~ich~zástupcov. Grafické znázornenie tejto architektúry je možné nahliadnuť v~Diagrame~\ref{obr06:mapy_architektura}.   

\bigskip

Mapy sú reprezentované pomocou tried, ktoré~implementujú rozhranie \textbf{\texttt{IMap}}. Toto rozhranie nedefinuje žiadnu zaujímavú funkcionalitu. Obsahuje pár vlastností využívaných mimo vrstvy Model pre~identifikáciu mapy.

Triedy máp následne môžu implementovať niektoré z~ďalších definovaných rozhraní, ktoré~pridávajú svoje kontrakty. Tieto kontrakty sú zamerané na~geografické lokalizovanie a~rozlohu máp. Sú využívané predovšetkým pri~získavaní dodatočných výškových dát zodpovedajúcich konkrétnej mape. Ak si je mapový typ vedomý toho, že~na~vytvorenie jeho mapovej reprezentácie bude potrebná výpomoc výškových dát, mal by implementovať aspoň jeden z~interface-ov, ktorý~definuje informácie o~geo-lokalite a~rozlohe mapy.

Ako už bolo naznačené, pri~mapách je potrebné, aby~v~aplikácii niekto zastupoval ich formáty. K~tomuto slúžia triedy implementujúce trojicu rozhraní:
\begin{itemize}
    \item \textbf{\texttt{IMapFormat<out TMap>}} - Je určený pre~komunikáciu mimo vrstvy Model. Definuje vlastnosti, ktoré~sú potrebné pri~práci s~mapovým zástupcom vo~vonkajšom prostredí a~metódu na~vytváranie mapy zastupovaného typu. Tento interface by nemal byť priamo implementovaný.
    \item \textbf{\texttt{IMapIdentifier<in TMap>}} - Slúži na~identifikáciu zodpovedajúceho mapového formátu pre~konkrétny typ mapy. Vďaka kontravariantnej povahe jeho typového parametru bude táto identifikácia fungovať správne aj~pre~potomkov typu \texttt{TMap}. Tento interface by nemal byť priamo implementovaný.
    \item \textbf{\texttt{IMapRepresentative<TMap>}} - Zastupuje jeden konkrétny typ/formát mapy. Je potomkom predošlých dvoch interface-ov - spája ich funkcionality. Tento interface je určený k~tomu, aby~bol priamo implementovaný mapovými zástupcami.   
\end{itemize}

Na to, aby~bolo možné mapový typ využiť v~aplikácii, musí byť jeho zástupca zahrnutý v~príslušnej kolekcii v~triede \texttt{MapManager}. Z~tohto dôvodu je na~mieste, aby~boli títo zástupcovia implementovaní ako singleton triedy.

Mapy podporujú návrhový vzor \textit{Generic visitor pattern}. Ten je definovaný ako pre~rozhranie \texttt{IMap}, tak aj~pre~rozhranie \texttt{IGeoLocatedMap} (pridáva kontrakt o~geografickej lokalite mapy).

\begin{figure}[h]\centering
\includegraphics[]{img/mapy_architektura}
\caption{Diagram popisujúci objektovú štruktúru máp a ich zástupcov.} 
\label{obr06:mapy_architektura}
\end{figure}

\section{Mapové reprezentácie}

Ďalšia z~oblastí sa stará o~vytváranie a~spravovanie mapových reprezentácii, resp. grafov. Čo do~obsahu aj~komplexnosti sa jedná o~jednu z~najväčších oblastí. Viac informácií o~koncepte mapových reprezentácií je možné nájsť v~Podsekcii~\ref{mapove_reprezentacie}.

Hlavným uzlom pre~prácu s~touto oblasťou z~vonkajšieho prostredia je singleton trieda \texttt{MapRepreManager}. Ten ponúka kolekciu obsahujúcu zástupcov všetkých typov mapových reprezentácií, ktoré~je možné v~aplikácii využiť. Popri tom obsahuje metódy určené na~
\begin{itemize}
    \item vytváranie mapových reprezentácií,
    \item identifikáciu reprezentácií vytvoriteľných pre~konkrétnu kombináciu typov template-u a~mapy,
    \item detekciu potreby výškových dát pri~konštrukcii mapovej reprezentácie.
\end{itemize}

V~nasledujúcich odsekoch popíšeme objektovú štruktúru mapových reprezentácií/grafov, ich implementácií a~zástupcov. Pre~jednoduchšie porozumenie je možné v~Diagrame~\ref{obr07:mapove_reprezentacie_architektura} nahliadnuť grafické znázornenie tejto architektúry. 

\bigskip

Štruktúra dát, ktoré~súvisia s~mapovými reprezentáciami odzrkadľuje dvojakosť významov mapovej reprezentácie a~grafu ako bolo popísané v~Podsekcii~\ref{mapove_reprezentacie}. Každému typu mapovej reprezentácie je prisúdený konkrétny typ grafu.

Mapové reprezentácie/grafy sú v~aplikácii reprezentované triedami, ktoré~implementujú rozhranie \textbf{\texttt{IGraph<TVertexAttributes, TEdgeAttributes>}}. Toto rozhranie reprezentuje myšlienku grafu, ktorý~zastupuje konkrétnu mapovú reprezentáciu. Jeho typové parametre definujú atribúty, ktoré~sú používané v~jeho vrcholoch a~hranách. Obsahuje kontrakty, ktoré~musia grafy všetkých typov napĺňať. Aktuálne je to len jedna metóda vracajúca graf do~jeho základného stavu.

\texttt{IGraph<TVertexAttributes, TEdgeAttributes>} má za~predchodcu rozhranie \textbf{\texttt{IMapRepre}} reprezentujúce myšlienku samotnej mapovej reprezentácie. Cez toto rozhranie sú mapové reprezentácie/grafy distribuované vonkajšiemu svetu. K~tomuto účelu rozhranie definuje vlastnosti využívané mimo vrstvy Model. Grafová podstata mapových reprezentácií je tým vonkajšiemu svetu ukrytá a~len špecifické oblasti Model vrstvy ju znova nadobúdajú a~využívajú.

Každá mapová reprezentácia/graf má množinu svojich implementácií. Implementácie sú tvorené pre~konkrétne kombinácie template-ov a~mapových formátov. Konkrétny typ mapovej reprezentácie môže byť vytvorený len pre~takú kombináciu template-u a~mapového formátu, pre~ktorý je vytvorená príslušná implementácia.

V~aplikácii sú naďalej prítomné ďalšie rozhrania, ktoré~môžu (a mali by v~čo najväčšom rozsahu) jednotlivé grafy implementovať. Tieto rozhrania definujú kontrakty týkajúce sa vlastností a~funkcií generovaných grafov.

\bigskip

Na to aby~mohla byť mapová reprezentácia, graf či~implementácia v~aplikácii použitá, musí pre~ne existovať vhodný zástupca. Tento zástupca je následne poskytnutý na~patričnom mieste.
\begin{itemize}
    \item Zástupcovia typov mapových reprezentácií sú v aplikácii reprezentovaní triedami, ktoré~implementujú rozhranie \textbf{\texttt{IMapRepreRepresentative<out TMapRepre>}}. Tieto triedy sú určené (podobne ako typy mapových reprezentácií) pre~komunikáciu mimo vrstvy Model. Definujú vlastnosti, ktoré~sú využívané vonkajším svetom pre~získanie informácií o~zastupovanej mapovej reprezentácii. 
    
    Taktiež si držia indikátorovú kolekciu zástupcov všetkých implementácií zastupovanej mapovej reprezentácie. Táto kolekcia sa následne využíva pri~identifikácii použiteľných kombinácií template-ov a~mapových formátov a~pri samotnom vytváraní mapových reprezentácií/grafov.
    
    Nakoľko typ mapovej reprezentácia je vždy asociovaný s~nejakým typom grafu, zástupca typu mapovej reprezentácia disponuje aj~referenciou na~zástupcu typu daného grafu. Tento zástupca sa následne využíva v~procese vytvárania mapovej reprezentácie/grafu. Taktiež špecifické oblasti vrstvy Model ho môžu využiť k~otestovaniu vlastností zastupovaného grafu.

    Taktiež toto rozhranie definuje metódy pre~vytváranie zastupovaného typu mapovej reprezentácie. 

    Na~to, aby~bolo možné typ mapovej reprezentácie/grafu využiť v~aplikácii, musí byť jeho zástupca zahrnutý v~príslušnej kolekcii manažéra mapových reprezentácií.

    \item Zástupcovia grafových typov sú reprezentovaní triedami, ktoré~implementujú rozhranie \textbf{\texttt{IGraphRepresentative < out TGraph, TVertexAttributes, TEdgeAttributes>}}. Typové parametre definujú typ zastupovaného grafu a~typy atribútov, ktoré~sú použité v~jeho vrcholoch a~hranách. Tento interface nedefinuje žiadny kontrakt pre~grafových zástupcov. Implementuje iba metódy, ktoré~sú využívané v~procese vytvárania mapovej reprezentácie/grafu. Je využívaný špecifickými oblasťami vrstvy Model k~otestovaniu vlastností zastupovaného grafu.

    Každý zástupca grafu je asociovaný s~konkrétnym zástupcom mapovej reprezentácie. Ten si na~jeho inštanciu drží referenciu a~využíva ho v~procese vytvárania mapovej reprezentácie/grafu.

    \item Ku~reprezentácii zástupcov jednotlivých implementácií slúžia triedy, ktoré~dedia od abstraktnej triedy \textbf{\texttt{ElevDataIndepImplementationRep}} alebo~od abstraktnej triedy \textbf{\texttt{ElevDataDepImplementationRep}}. Tieto rozhrania disponujú dvojitou funkcionalitou:
    \begin{itemize}
        \item Obsahujú vlastnosti indikujúce template a~mapový formát, na~základe ktorých je implementácia mapovej reprezentácie/grafu agregovaná. Tieto vlastnosti sú naplnením kontraktu definovaného ich predchodcom \textbf{\texttt{IImplementationIndicator}}.
        \item Je schopná skonštruovať (alebo~nechať skonštruovať) zastúpenú implementáciu mapovej reprezentácie. Tieto triedy sa líšia predovšetkým v~potrebe dodatočných výškových dát v~procese tvorby danej implementácie. Schopnosť skonštruovať danú implementáciu je naplnením kontraktu jedného z~rozhraní \textbf{\texttt{IImplementationElevDataIndepConstr}}, respektíve \textbf{\texttt{IImplementationElevDataDepConstr}}. 
    \end{itemize}
    Taktiež disponujú množinou typových parametrov, z~ktorých každý má svoj špecifický význam:
    \begin{itemize}
        \item \texttt{TTemplate} - definuje typ template-u, pre~ktorý je zastupovaná implementácia vytvorená,
        \item \texttt{TGraph} - definuje typ mapovej reprezentácie/grafu, pre~ktorý je zastupovaná implementácia vytvorená,
        \item \texttt{TVertexAttributes, TEdgeAttributes} - definujú typy atribútov použitých v~implementovanom grafe,
        \item \texttt{TMap, TUsableSubMap} - tieto parametre sú jemne zavádzajúce. Prvý z~nich hovorí o~tom, aký typ mapy je navonok indikovaný pomocou vlastnosti mapového formátu. Druhý hovorí o~tom, ktorý~typ potomok typu \texttt{TMap} je v~skutočnosti potrebný na~vytvorenie mapovej reprezentácie. Malo by byť zaručené okolitým prostredím, že~pokiaľ je nejaká mapa správneho formátu, tak je určite možné ju použiť pre~vytvorenie mapovej reprezentácie. 
    \end{itemize}
    
    Zástupcovia jednotlivých implementácií mapovej reprezentácie sú zahrnutí v~kolekcii zástupcu tejto mapovej reprezentácie.

\end{itemize}

Bolo by vhodné, aby~všetci spomenutí zástupcovia boli implementovaní ako singleton triedy. Od každého z~nich je totiž za~celú dobu behu programu potrebné vytvoriť iba jedinú inštanciu, ktorá je poskytovaná zvyšku aplikácie na~patričnom mieste.

Poslednou súčasťou oblasti mapových reprezentácií sú triedy, ktoré~reprezentujú vrcholy a~hrany používané v~grafoch. Jednotlivé typy sa potom líšia dodávanými vlastnosťami.   

\begin{figure}[p]\centering
\includegraphics[]{img/mapove_reprezentacie_architektura}
\caption{Diagram popisujúci objektovú štruktúru mapových reprezentácií/grafov, ich implementácií a~zástupcov.} 
\label{obr07:mapove_reprezentacie_architektura}
\end{figure}

\pagebreak

\section{Užívateľské modely}

V~tejto oblasti sú vytvárané a~spravované užívateľské modely. Viac informácií o~koncepte užívateľských modelov je možné nájsť v~Podsekcii~\ref{uzivatelske_modely}.

Hlavným komunikačným uzlom s~touto oblasťou pre~Model view vrstvu je singleton trieda \texttt{UserModelManager}. Ten ponúka kolekciu všetkých typov užívateľských modelov, ktoré~je možné v~aplikácii využiť. Mimo to implementuje metódy pre:
\begin{itemize}
    \item Serializáciu a~deserializáciu užívateľských modelov do/z súborov,
    \item Vytváranie nových inštancií užívateľských modelov,
    \item Všeobecnú identifikáciu typov užívateľských modelov na~základe rôznych typov argumentov.
\end{itemize}

V~následujúcich odsekoch popíšeme objektovú štruktúru užívateľských modelov a~ich~zástupcov. Grafické znázornenie tejto architektúry je dostupné k nahliadnutiu v~Diagrame~\ref{obr08:uzivatelske_modely_architektura}.   

\bigskip

Užívateľské modely sú v~aplikácii reprezentované triedami, ktoré~implementujú rozhranie \textbf{\texttt{IUserModel<out TTemplate>}}.  Prostredníctvom tohto rozhrania sú užívateľské modely využívané mimo vrstvy Model. Z~tohto dôvodu definuje informatívne vlastnosti, ktoré~sú potrebné prevažne vo~vonkajšom svete. Popri tom definuje ešte aj~kontrakty zaručujúce schopnosť serializácie užívateľského modelu. Každý užívateľský model je viazaný na~konkrétny template-u, typu \texttt{TTemplate}. K~tomu si každý užívateľský model nesie aj~referenciu na~inštanciu daného template-u.

Na to aby~bol užívateľský model použiteľný vo~vyhľadávacích algoritmoch, nestačí aby~implementoval predchádzajúce základné rozhranie. Je potrebné aby~implementoval následnícke rozhranie \textbf{\texttt{IComputingUserModel<out TTemplate, in TVertexAttributes, in TEdgeAttributes>}}. Tento interface sám o~sebe nedefinuje žiadny kontrakt. Až jeho následníci definujú výpočtovú funkcionalitu, ktorú implementujúci užívateľský model vie ponúknuť napríklad vyhľadávaciemu algoritmu. 

Jedným z~týchto následníkov je interface \textbf{\texttt{IWeightComputingUserModel<out TTemplate, in TVertexAttributes, in TEdgeAttributes>}}. Toto rozhranie zaisťuje schopnosť užívateľského modelu, na~základe dodaných vrcholových a~hranových atribútov, vypočítať váhu zodpovedajúcej hrany. Túto funkcionalitu by mali spĺňať všetky užívateľské modely, nakoľko veľká časť vyhľadávacích algoritmov potrebuje poznať váhu jednotlivých hrán pre~správny výber postupu.

Oproti výpočtovým rozhraniam existuje v~tejto oblasti aj~špecifické rozhranie \textbf{\texttt{ISettableUserModel}}. Toto rozhranie musia implementovať všetky typy užívateľských modelov, ktoré~sú určené na~to, aby~v~nich užívateľ mohol konfigurovať nastaviteľné hodnoty (tzv.\textit{Adjustables}) vzhľadom na~svoje preferencie. Kontrakt ktorý~definuje zaručuje dodanie kolekcie týchto nastaviteľných hodnôt, aby~mohli byť dodané užívateľovi a~ten s~nimi mohol pracovať. Mechanizmus vytvárania a~nastavovania užívateľských modelov zatiaľ v~aplikácii nie je implementovaný, a~teda toto rozhranie aktuálne čaká na~svoje využitie.  

\bigskip

V~aplikácii je potrebné, aby~existencia jednotlivých typov užívateľských modelov bola nejakým spôsobom zastúpená. Týmito zástupcami sú triedy implementujúce následujúcu trojicu rozhraní:
\begin{itemize}
    \item \textbf{\texttt{IUserModelType<out TUserModel, out TTemplate>}} - Je určený pre~komunikáciu mimo vrstvy Model. Definuje vlastnosti, ktoré~sú potrebné pri~práci so~zástupcom užívateľského modelu vo~vonkajšom prostredí. 
    
    Taktiež obsahuje referenciu na~template, na~ktorý je zastupovaný užívateľský model viazaný a~definuje metódy slúžiace na~deserializáciu a~vytváranie nových užívateľských modelov. Implementácie deserializácie zástupcu a~serializácie zastupovaného užívateľského modelu sa musia zhodovať. 
    
    Tento interface by nikdy nemal byť priamo implementovaný.
    \item \textbf{\texttt{IUserModelTemplateBond<in TTemplate>}} - Reprezentuje väzbu užívateľského modelu a~template-u.  Vďaka kontravariantnej povahe jeho template-ového typového parametru bude táto identifikácia fungovať správne aj~pre~prípadných potomkov typu \texttt{TTemplate}. Tento interface by nikdy nemal byť priamo implementovaný.
    \item \textbf{\texttt{IUserModelRepresentative<TUserModel,TTemplate>}} - Zastupuje jeden konkrétny typ užívateľského modelu viazaného na~jeden konkrétny typ template-u. Je potomkom predošlých dvoch interface-ov - spája ich funkcionality. Tento interface je určený k~tomu, aby~bol priamo implementovaný zástupcami užívateľských modelov.   
\end{itemize}

Na to, aby~bolo možné typ užívateľského modelu využiť v~aplikácii, musí byť jeho zástupca zahrnutý v~príslušnej kolekcii triedy \texttt{UserModelManager}. Z~tohto dôvodu je na~mieste, aby~títo zástupcovia boli implementovaný ako singleton-y.

\begin{figure}[h]\centering
\includegraphics[]{img/uzivatelske_modely_architektura}
\caption{Diagram popisujúci objektovú štruktúru užívateľských modelov a~ich~zástupcov.} 
\label{obr08:uzivatelske_modely_architektura}
\end{figure}

\section{Vyhľadávacie algoritmy}

Táto oblasť zahŕňa mechanizmy spravujúce algoritmy pre~vyhľadávanie ciest v~mapových reprezentáciách. Viac informácií o~vyhľadávacích algoritmoch samotných je možné nájsť v~Podsekcii~\ref{vyhladavacie_algoritmy}. 

Hlavným komunikačným uzlom s~touto oblasťou pre~Model view vrstvu je singleton trieda \texttt{SearchingAlgorithmMan}. Táto trieda zverejňuje v~kolekcii všetky použiteľné vyhľadávacie algoritmy aplikácie. Popri tom obsahuje metódy zabezpečujúce: 
\begin{itemize}
    \item spúšťanie procesu vyhľadávania cesty na~základe dodaného užívateľského modelu a~mapovej reprezentácie
    \item dodávanie \textit{executor}-u vyhľadávacieho algoritmu
    \item identifikáciu vyhľadávacích algoritmov spustiteľných pre~konkrétne kombinácie mapových reprezentácií a~užívateľských modelov 
\end{itemize}

V~následujúcich odsekoch popíšeme objektovú štruktúru vyhľadávacích algoritmov a~ich~implementácií. Jednoduché grafické znázornenie tejto architektúry je dostupné k nahliadnutiu v~Diagrame~\ref{obr09:vyhladavacie_algoritmy_architektura}.   

\bigskip

Vyhľadávacie algoritmy sú v~aplikácii reprezentované pomocou tried, ktoré~implementujú rozhranie \textbf{\texttt{ISearchingAlgorithm}}. Každý algoritmus môže byť implementovaný niekoľkými spôsobmi. Rozhranie preto definuje kolekciu, v~ktorej by mali byť všetky použiteľné implementácie daného algoritmu zverejnené. Ďalej definuje a~zároveň implementuje metódy, ktoré~slúžia na:
\begin{itemize}
    \item testovanie, či~zástupcovia typov mapovej reprezentácie a~užívateľského modelu zastupujú použiteľnú kombináciu pre~daný algoritmus. Teda či~existuje implementácia algoritmu, pre~ktorú sú vlastnosti zastupovaných typov dostatočné na~použitie, 
    \item samotné spúšťanie vyhľadávacieho procesu. K~tomuto účelu sa vyberie pre~vstupné argumenty vhodná implementácia algoritmu,
    \item možnosť získania \textit{executor}-u daného algoritmu. Executor sa vytvorí na~základe vhodnej implementácie algoritmu.
\end{itemize}
Inštancia každého použiteľného vyhľadávacieho algoritmu musí byť obsiahnutá v~kolekcii vyhľadávacích algoritmov v~príslušnom manažérovi. Inak nebude aplikácia daný vyhľadávací algoritmus registrovať. 

Implementácie vyhľadávacích algoritmov sú reprezentované triedami, ktoré~implementujú rozhranie \textbf{\texttt{ISearchingAlgorithmImplementation}}. Toto rozhranie definuje myšlienky funkcionalít podobné tým z~rozhrania \texttt{ISearchingAlgorithm}: testovanie typov vstupných mapových reprezentácií a~užívateľských modelov, spúšťanie vyhľadávacieho procesu a~vytváranie svojich executor-ov. V~tomto prípade však nie je táto funkcionalita implementovaná rozhraním a~je potrebné aby~ju implementácie algoritmov doplnili sami. Na~to aby~implementácia algoritmu mohla byť použitá, musí byť obsiahnutá v~kolekcii implementácií zodpovedajúceho vyhľadávacieho algoritmu. 

Výsledkom vyhľadávania je inštancia triedy, ktorá implementuje rozhranie \textbf{\texttt{IPath<out TVertexAttributes, out TEdgeAttributes>}}. Toto rozhranie reprezentuje nájdenú cestu algoritmom pričom môže v~sebe niesť atribúty typov \texttt{TVertexAttributes} a~\texttt{TEdgeAttributes}. Neskôr v~Model view vrstve je z~tejto cesty vytvorený report, ktorý~je následne vyššími vrstvami spracovaný a~predvedený užívateľovi. Pre~komunikáciu mimo Model vrstvy sa využíva jeho predchodca \texttt{IPath}. Toto rozhranie by nemalo byť nikdy priamo implementované. 

Vyhľadávacie algoritmy taktiež môžu počas svojho behu podávať report-y o~stave vyhľadávania prostredníctvom objektov tried, ktoré~implementujú rozhranie \textbf{\texttt{ISearchingState<out TVertexAttributes, out TEdgeAttributes>}}. Algoritmus nechá z~týchto stavov agregovať report a~podá ho k~následnému spracovaniu a~predvedeniu užívateľovi.

Obidve vyššie spomenuté rozhrania taktiež podporujú návrhový vzor \textit{generic visitor pattern}. 

Pokiaľ je po~algoritme požadované vytvorenie jeho executor-u, algoritmus zavolá vhodnú svoju implementáciu nech executor vytvorí. Tá ho inicializuje pomocou dodanej mapovej reprezentácie, užívateľského modelu a~pridá delegáta na~svoju špecifickú metódu zabezpečujúcu beh algoritmu pre~executor.

Všetky triedy reprezentujúce vyhľadávacie algoritmy a~ich implementácie by mali byť implementované ako singleton triedy. Ich inštancie budú totiž vytvorené v~aplikácii len jeden krát.


\begin{figure}[h]\centering
\includegraphics[]{img/vyhladavacie_algoritmy_architektura}
\caption{Diagram popisujúci objektovú štruktúru vyhľadávacích algoritmov a~ich~implementácií.} 
\label{obr09:vyhladavacie_algoritmy_architektura}
\end{figure}

\pagebreak

\section{Výškové dáta}

Oblasť pre~správu a~manipuláciu s~výškovými dátami. Viac informácií o~funkcii výškových dát v~aplikácii je možné nájsť v~Podsekcii~\ref{vyskove_data}. 

Hlavným komunikačným uzlom s~touto oblasťou z~vrstvy Model view je singleton trieda \texttt{ElevDataManager}. Tá ponúka kolekciu všetkých použiteľných zdrojov výškových dát v~aplikácii. Popri tom doručuje metódy pre~manipuláciu s~výškovými dátami (sťahovanie a~odstraňovanie) a~taktiež metódy, ktoré~testujú prítomnosť a~vracajú prítomné výškové dáta zodpovedajúce rozlohe konkrétnej mapy.

V~následujúcich odsekoch popíšeme objektovú štruktúru výškových dát, ich zdrojov, distribúcií a~regiónov. Pre lepšie porozumenie je v~Diagrame~\ref{obr10:vyskove_data_architektura} dostupný náhľad  grafického znázornenia architektúry týchto dátových štruktúr.   

\bigskip

Výškové dáta sú v~aplikácii reprezentované triedami, ktoré~implementujú rozhranie \textbf{\texttt{IElevData}}. Toto jednoduché rozhranie definuje metódy, ktoré~dokážu k~zadanej geografickej polohe vrátiť jej nadmorskú výšku. Inštancie týchto tried su na~mieru vytvárané tak, aby~dokázali dodať výškové dáta primerané polohe a~rozlohe konkrétnej mapy. 

Výškové dáta sú dodávané jednotlivými zdrojmi. Tie sú v~aplikácii reprezentované triedami, ktoré~implementujú rozhranie \textbf{\texttt{IElevDataSource}}. Zdroje výškových dát sú zložené z~viacerých distribúcií, ktoré~následne už dodávajú potrebnú funkcionalitu pre~prácu s~nimi spravovanými, výškovými dátami. Preto rozhranie \texttt{IElevDataSource} definuje kolekciu, v~ktorej by mali byť uložené všetky distribúcie daného zdroja na~to, aby~mohli byť v~aplikácii použité. Popri tom definuje ďalších pár vlastností, ktoré~sú využívané mimo vrstvy Model (napr. meno daného zdroja).

Jednotlivé distribúcie výškových dát sú v~aplikácii reprezentované triedami, ktoré~implementujú rozhranie \textbf{\texttt{IElevDataDistribution}}. Každá z~týchto tried je zodpovedná za~prácu s~konkrétnou distribúciou výškových dát. Zabezpečujú sťahovanie, odstraňovanie a~informovanie o~ich prítomnosti. Je ponechané na~zodpovednosti implementácií, akým spôsobom budú dáta ukladané, načítané do~pamäte a~spracovávané do~inštancií implementácií rozhrania \texttt{IElevData}.

Rozhranie \texttt{IElevDataDistribution} by nemalo byť implementované priamo. Namiesto toho by mali byť implementované rozširujúce rozhrania \textbf{\texttt{ICredentials-} \texttt{NotRequiringElevDataDistribution}} a~\textbf{\texttt{ ICredentialsRequiringElevData- } \texttt{Distribution}},  ktoré~pridávajú samotnú metódu umožňujúcu sťahovanie výškových dát. Tieto metódy, resp. rozhrania, sa líšia v~potrebe autorizácie pri~získavaní výškových dát zo~vzdialených zdrojov.  

Manipulovanie s~výškovými dátami prebieha po~takzvaných \textit{regiónoch}. Každá distribúcia si tvar a~veľkosť svojich regiónov určuje sama. Následne tieto regióny sprostredkováva v~kolekcii \texttt{AllTopRegions}, ktorá je definovaná rozhraním \texttt{IElevDataDistribution}.

Regióny sú reprezentované triedami, ktoré~dedia od abstraktnej triedy \textbf{\texttt{Region}} a~jej potomkov \textbf{\texttt{TopRegion}} a~\textbf{\texttt{SubRegion}}. Región ako taký definuje svoje meno, svoj tvar za~pomoci reťazca geografických súradníc a~množinu svojich pod-regiónov. Taktiež definuje identifikátor svojej prítomnosti. Teda toho, čí sú jemu zodpovedajúce výškové dáta stiahnuté v~počítači a~pripravené na~použitie alebo~nie.

Tento indikátor by mal byť aktualizovaný vždy keď sa prítomnosť dát regiónu zmení. Dôležité však je, že~táto informácia by mala byť zachovaná naprieč behmi aplikácie. Teda ak sú v~jednom behu aplikácie stiahnuté výškové dáta pre~konkrétny región, v~následujúcom behu by mal región indikovať, že~sú jemu prislúchajúce dáta stále k~dispozícii.

Regióny sú naďalej delené na~\textit{vrcholové regióny} a~\textit{pod-regióny}. Pod-regióny sú vždy viazané na~nejaký vyšší región a~reprezentujú nejakú jeho časť. Vrcholový región potom jednoducho nie je nikoho pod-regiónom. Spomínaná kolekcia \texttt{AllTopRegions} obsahuje práve vrcholové regióny definované danou distribúciou.    


\begin{figure}[h]\centering
\includegraphics[]{img/vyskove_data_architektura}
\caption{Diagram popisujúci objektovú štruktúru zdrojov, distribúcií a~regiónov výškových dát.} 
\label{obr10:vyskove_data_architektura}
\end{figure}

\pagebreak

\section{Grafika}

Táto oblasť sa zaoberá problematikou vytvárania objektov, z~ktorých sa skladajú grafické reprezentácie rôznych dátových štruktúr. Agregácia grafických objektov sa vykonáva špecifickým asynchrónnym postupom. Vytvorené grafické objekty sa plnia do~tzv \textit{kolektoru}. Tým pádom je možné vytvorené objekty paralelne spracovávať a~zobrazovať ihneď po~ich vytvorení. 

Aplikácia obsahuje hneď dva hlavné uzly pre~komunikáciu s~touto oblasťou:
\begin{itemize}
    \item \texttt{GraphicsManager} je hlavným uzlom pre~komunikáciu prichádzajúcu z~Model view vrstvy. Zabezpečuje agregáciu grafických objektov pre~rôzne dátové štruktúry aplikácie ako sú napríklad mapy.
    \item \texttt{GraphicsSubManager} slúži k~rovnakému účelu ako \texttt{GraphicsManager}, avšak pre~komunikáciu priamo z~vrstvy Model. Reprezentuje prívetivejší spôsob komunikácie so~zachovaním typových informácií. Dátové štruktúry, pre~ktoré zabezpečuje táto trieda agregáciu grafiky, sú napríklad nájdené cesty a~stavy vyhľadávacích algoritmov. 
\end{itemize}
Obidve tieto triedy sa riadia návrhovým vzorom singleton.

\bigskip

Grafické objekty sú v~aplikácii reprezentované triedami, ktoré~implementujú rozhranie \textbf{\texttt{IGraphicObject}}.Toto rozhranie neimplementuje takmer žiadnu funkcionalitu okre podpory návrhového vzoru \textit{Generic visitor pattern}.

Aby bolo možné grafiku dodanej mapy, cesty či~stavu extrahovať, musí pre~ňu existovať trieda implementujúca rozhranie \textbf{\texttt{IGraphicsAggregator}}, teda presnejšie jedného z~jeho špecializovaných potomkov. Títo \textit{agregátori} následne dostanú objekt na~spracovanie a~kolektor, do~ktorého sa majú naklásť vytvorené grafické objekty.

V~prípade ciest a~stavov vyhľadávacieho algoritmu dostane agregátor aj~užívateľský model, ktorý~môže využiť na~výpočet niektorých hodnôt z~vrcholových a~hranových atribútov uložených v~dodaných cestách/stavoch. Je však potrebné zdôrazniť, že~užívateľský model možno nebude schopný tieto služby doručiť. V~takom prípad sa musí agregátor zaobísť bez nich. Je možnosť, aby~tieto vlastnosti vynucoval napríklad vyhľadávací algoritmus, ktorý~pozná potreby pre~extrahovanie grafiky ním používaného typu cesty či~stavu vyhľadávania.

Bolo by namieste, keby každý užívateľský model dokázal z~atribútov vyťažiť aspoň pozície vrcholov mapovej reprezentácie, aby~bolo možné nakresliť aspoň základnú reprezentáciu nájdenej cesty či~stavu. 

\bigskip

Nakoniec sú v~tejto oblasti definované dve rozhrania ktoré~slúžia pre~reprezentáciu grafického zdroja. \textbf{\texttt{IGraphicsSource}} definuje jedinú vlastnosť a~to zdrojovú kolekciu grafických objektov. Táto zdrojová kolekcia je typu \texttt{SourceList} patriaceho do~framework-u \textit{Reactive UI}. vo~vyšších vrstvách MVVM(MV) architektúry je následne možné tento zdrojový zoznam sledovať a~reagovať na~jeho aktualizácie. 

Špecializácia predošlého rozhrania \textbf{\texttt{IGroundGraphicsSource}} dodáva ešte nutnosť, aby~daný grafický zdroj definoval aj~svoju rozlohu. Triedy implementujúce toto rozhranie sú často využívané ako akési základné grafiky, ktorých rozlohe sa ostatné grafické zdroje prispôsobujú.

Implementácie týchto rozhraní sú väčšinou vytvárané mimo vrstvy Model. Sú na~mieru vytvorené pre~konkrétnu aplikačnú logiku. 

\section{Reportovanie}

Oblasti spravujúcej grafiku je architektúrou veľmi podobná oblasť vytvárajúca report-y na~základe rôznych typov dátových štruktúr. V~aktuálnej podobe aplikácie sú týmito štruktúrami cesty, nájdené vyhľadávacími algoritmami a~stavy vyhľadávacích algoritmov. Táto oblasť veľmi často využíva služby triedy \texttt{GraphicsSubManager} pre~získavanie grafiky, ktorá je pridaná do~obsahu report-ov.

Oblasť obsahuje opäť dva hlavné komunikačné uzly:
\begin{itemize}
    \item \texttt{ReportManager} je hlavným uzlom pre~komunikáciu prichádzajúcu z~Model view vrstvy. Aktuálne zabezpečuje vytváranie report-ov pre~nájdené cesty vyhľadávacím algoritmom.
    \item \texttt{ReportSubManager} slúži k~rovnakému účelu ako \texttt{ReportManager}, avšak pre~komunikáciu priamo z~vrstvy Model. Reprezentuje prívetivejší spôsob komunikácie so~zachovaním typovej informácie. Aktuálne zabezpečuje vytváranie report-ov pre~nájdené cesty a~stavy vyhľadávaní.
\end{itemize}
Obidve tieto triedy sa riadia návrhovým vzorom singleton.

\bigskip

Pre cesty a~stavy vyhľadávania sú reporty v~aplikácii reprezentované triedami, ktoré~implementujú rozhrania \textbf{\texttt{IPathReport}} a~\textbf{\texttt{ISearchingReport}}. Tieto rozhrania nedefinujú takmer žiadnu funkcionalitu, okrem podpory návrhového vzoru \textit{generic visitor pattern}.

Podobne ako v~oblasti spravujúcej grafiku, na~to, aby~pre~konkrétny typ dátovej štruktúry mohol byť report vytvorený, musí preňho existovať vhodná trieda implementujúca rozhranie \textbf{\texttt{IReportAggregator}}, resp. jedného z~jeho špecializovaných potomkov. Títo \textit{agregátori} následne dostanú dátovú štruktúru na~spracovanie a~v~niektorých prípadoch aj~užívateľský model, ktorý~môžu využiť na~výpočet hodnôt z~vrcholových a~hranových atribútov uložených v~dodaných dátach. Podobne však ako u grafických agregátorov, nie je zaručené, že~užívateľský model bude schopný požadované služby doručiť. 

\section{Parametre}

Posledná, jednoduchšia, ale~o~to dôležitejšia, oblasť je využívaná na~spravovanie a~ukladanie všemožných parametrov aplikácie. Hlavným uzlom pre~komunikáciu s~touto oblasťou je singleton trieda \texttt{ParamsManager}. V~tejto triede je možné uložiť od každého typu parametrov práve jednu inštanciu. Táto inštancia môže byť počas behu algoritmu rôzne upravovaná. Inštancie parametrov sú uložené v~slovníku pod kľúčom reprezentujúcom ich vlastný typ.  

Keď je zavolaná metóda \texttt{SaveAllParams}, tak sa trieda \texttt{ParamsManager} pokúsi všetky uložené parametre v~slovníku serializovať do~súborov pre~možnosť ich použitia v~budúcich behoch aplikácie.

Pri následnom behu aplikácie sú parametre deserializované zo~súborov tzv. \uv{lenivým} spôsobom. Keď je požiadané o~parametre typu, ktorý~v~slovníku nie je zastúpený, trieda sa pokúsi najprv zistiť, či~preňho neexistuje zodpovedajúca serializácia. Ak áno, deserializuje sa inštancia zodpovedajúcich parametrov, vloží sa do~slovníku a~vráti užívateľovi. Pokiaľ nie, poznačí sa do~slovníku neexistencia parametru daného typu a~navráti sa hodnota \texttt{null}.  

Serializácia a~deserializácia parametrov do~súborov, v~ktorých prežívajú beh aplikácie, prebieha za~pomoci singleton triedy \textbf{\texttt{DataSerializer}}. Táto trieda serializuje objekty do~súborov pomocou systémovej triedy \textit{\texttt{JsonSerializer}}. Súbory sú pomenované podľa typu daného serializovaného objektu. To znamená, že~v~jednu chvíli pre~každý dátový typ dokáže táto trieda serializovať jednu jedinú inštanciu. Pri~deserializácii, na~základe vstupného generického typového parametru nájde súbor so~zodpovedajúcim menom a~pokúsi sa ho deserializovať do~inštancie daného typu.


  
\chapter{Vyhľadávanie trás v~mapách pre~orientačný beh}

Ako už bolo spomenuté v~sekcii~\ref{Aspekty_hladania}, vyhľadávanie trás v~mapách je výsledkom kombinácie viacerých konceptov. V~tejto kapitole sa bližšie pozrieme na~implementácie týchto konceptov, ktoré~sa spoločne podieľajú na~vyhľadávaní najrýchlejších trás v~mapách z~prostredia orientačného behu.

\section{Orienteering (ISOM 2017-2) template}

Tento template definuje vrcholové a~hranové atribúty, ktorými dokážeme detailne popísať miesto na~mape využívanej v~orientačnom behu. Je potrebné, aby dokázali zachytiť topológiu mapy, prostredie, v~ktorom sa daný vrchol/hrana nachádza a nadmorskú výšku konkrétneho bodu na~mape.  

Čo sa mapových značiek týka, využíva sa sada mapových značiek definovaná normou ISOM 2017-2. \uv{Cieľom Medzinárodnej špecifikácie máp pre~orientačné športy (ISOM) je poskytnúť mapovú špecifikáciu, ktorú možno používať na~celom svete pre~rôzne typy terénu, ktoré~sú vhodné pre~orientačné športy. Táto špecifikácia by sa mala používať v~spojení s~pravidlami Medzinárodnej federácie orientačných športov (IOF) pre~orientačné preteky.}\cite{ISOM20172}

Vrcholové atribúty samotné nesú informáciu o~polohe vrcholov grafu na~mape a~indikujú nadmorskú výšku v~mieste daného vrcholu. V~aktuálnom stave aplikácie nie je možné získať informáciu o~nadmorskej výške vrcholov grafu a~teda je nadmorská výška všetkých vrcholov v grafoch nastavená na~hodnotu 0. To znamená, že pri~výbere najrýchlejších trás v~mape sa nebude brať do~úvahy informácia o~prevýšení, ktoré~musí bežec na~danej ceste prekonať.

Informáciu o~okolitom teréne a~prostredí následne nesú hranové atribúty, ktoré~popisujú okolie jednotlivých hrán grafu. Okolím hrany sa myslí prostredie na~ľavej a~pravej strane konkrétnej hrany (les, vegetácia, lúka, vodné teleso, mokraď, atď) a~prípadná cesta alebo líniová prekážka, ktorá cez danú hranu prechádza.

Okolité prostredie hrany je reprezentované kombináciou indikátorov plošných mapových značiek (rastrov), ktorá zodpovedá korektnej kombinácii taktiež definovanej normou ISOM 2017-2. Pre~ľavé a~pravé prostredie hrany sú v~hranových atribútoch vytvorené dve premenné, v~ktorých je možné uložiť akúkoľvek korektnú kombináciu týchto indikátorov. Korektnosť kombinácie indikátorov si musí užívateľ týchto atribútov zabezpečiť sám.  

Pre indikovanie toho, že hrana zodpovedá ceste alebo toho, že cez hranu prechádza líniová prekážka, v~hranových atribútoch slúži jediná premenná, v~ktorej je možné uložiť informáciu o~kombinácii jednej cesty, jednej človekom vytvorenej líniovej prekážky a~jednej \uv{prírodnej} líniovej prekážky. 

\section{Complete, net intertwining map representation}

Pre grafovú reprezentáciu máp pre~orientačný beh bola vytvorená takzvaná \textit{sieť-prepletajúca} mapová reprezentácia. Tvorba orientovaného ohodnoteného grafu tejto mapovej reprezentácie spočíva vo vytvorení siete z~vrcholov a~hrán vo veľkosti dodanej mapy, a~následného korektného postupného vkladania jednotlivých objektov z~mapovej reprezentácie do~tejto vytvorenej siete. Táto forma tvorby grafu je vhodná pre~mapy, ktoré~vznikli z~mapových formátov, ktoré~reprezentujú mapu pomocou množiny geometrických útvarov, ktoré~popisujú tvary jednotlivých objektov na~mape.\\
Práve takým formátom je napríklad \textit{OMAP}, ktorý je jedným z~formátov využívaných pre~reprezentáciu máp orientačného behu a~je možné ho v~aplikácii spracovať do~podoby z~ktorej sa bude následne dať vytvoriť sieť-prepletajúca mapová reprezentácia. 

V následujúcej pod-sekcii popíšeme proces, ktorým je mapa  získaná zo súboru formátu OMAP postupne spracovávaná do~grafovej podoby.

\subsection{Postup spracovania mapy na~jej grafovú reprezentáciu}

Na začiatku sa vytvorí spomínaná sieť vrcholov, do~ktorej sa budú následne ukladať jednotlivé objekty mapy. Dĺžka hrán v~tejto sieti je daná konfiguračným parametrom \uv{štandardná dĺžka hrany}. Na~Obrázku \ref{obr20:just_net} je možné nahliadnuť vytvorenú sieť nad testovacou mapou.

Následne sa začnú spracovávať jednotlivé mapové objekty a~to v~následujúcom poradí: polygony, cesty a~líniové prekážky.
Objekty sa postupne vkladajú do~vytváraného grafu. Popri tom sa nastavujú atribúty na~jednotlivých hranách grafu tak, aby správne zodpovedali značkám vkladaných objektov.

Pridávanie objektov vyzerá následovne:

\begin{itemize}
    \item V~prvom rade sa vytvorí retiazka vrcholov spojených hranami odpovedajúca celému obvodu polygonu. Táto vytvorená retiazka však môže na~viacerých miestach krížiť samú seba a~teda je potrebné aby sa všetky tieto kríženia odstránili. Kríženia odstránime správnymi výmenami križujúcich sa hrán. Z~tohto procesu nám môže vyjsť viac disjunktných retiazok, ktoré~si následne pozberáme a~budeme ich naďalej spracovávať separátne. Taktiež nesmieme kvôli následnému spracovávaniu zabudnúť skontrolovať, či sú vytvorené retiazky \uv{pravotočivé}, teda či pravá strana každej hrany je vnútorná. Pokiaľ tomu u nejakej vytvorenej retiazky nie je, otočíme list retiazkových vrcholov a~tým dostaneme pravotočivú retiazku. Táto vlastnosť bude dôležitá v následujúcom spracovaní.
    \item V~druhom kroku musíme prerušiť všetky hrany, ktoré~vytvorená retiazka prerušila. Pri~identifikovaní prerušených hrán sa využívajú techniky z~geometrie a~lineárnej algebry. Takýto graf s~prerušenými hranami je možné nahliadnuť v~Obrázku \ref{obr21:cut_edges_by_polygon_chain}. Pre~ďalšie spracovanie sa zapamätajú všetky vrcholy, ktorým bola prerušená nejaká hrana.

\begin{figure}[p]\centering
\frame{\includegraphics[width=250px]{img/mapRepre/justNet0.jpg}}
\caption{Sieť grafu zhotovená nad vstupnou mapou. Červené čiary predstavujú grafové hrany.} 
\label{obr20:just_net}
\end{figure}

\begin{figure}[p]\centering
\frame{\includegraphics[width=250px]{img/mapRepre/cutEdgesByPolygonChain1.jpg}}
\caption{Retiazka vrcholov (objektu zakázaného priestoru) prerušujúca hrany vytváraného grafu. } 
\label{obr21:cut_edges_by_polygon_chain}
\end{figure}

    \item V~ďalšom kroku sa pomocou prehľadávania do~hĺbky identifikujú všetky vrcholy, ktoré~spadajú do~vnútra polygonu. Tento spôsob je založený na~tom, že všetky hrany spojujúce vnútorné vrcholy polygonu s~vrcholmi mimo polygonu boli v~predchádzajúcom kroku odstránené a~teda DFS neutečie mimo polygonu. Zároveň je zaručené, že sa nájdu všetky vnútorné vrcholy polygonu a~to tak, že sa sleduje, či boli nájdené všetky vrcholy, ktorým bola prerušená hrana retiazkou. Pokiaľ DFS nenavštívi všetky tieto vrcholy, spúšťa sa znova inicializovaná jedným z~takýchto vrcholov. Hranám týchto vnútorných vrcholov sa upravia atribúty na~základe mapovej značky ktorej objekt sa aktuálne spracováva ako je možné nahliadnuť v~Obrázku \ref{obr22:inner_edges_attributes_set}.
    \item V~posledných krokoch sa retiazka vrcholov pripojí na~vrcholy, ktorým bola prerušená aspoň jedna hrana a~správne sa na~týchto hranách nastavia ích atribúty. Následne sa taktiež nastavia atribúty na~hranách samotnej retiazky a~retiazka je vložená do~grafu. Toto nastavenie atribútov je korektné, teda vytvára iba normou ISOM 2017-2 povolené kombinácie mapových značiek. Proces nastavovania komentárov je zložitý a~vysvetľovaný detailnejšie priamo v~komentároch programu. Využíva sa pri~ňom pravotočivosť retiazok zabezpečená v~prvom kroku vytvorenia retiazky. Kompletne pridaný objekt do~grafu je možné nahliadnuť v~Obrázku \ref{obr23:connect_polygon_chain_to_vertices_of_cut_edges}
\end{itemize}

Aktuálne vytváraná mapová reprezentácia nedokáže zachytiť rôzne rýchlosti behu v~mapových značkách ako sú \textit{vegetácie s~lepšou rýchlosťou behu v~jednom smere} alebo \textit{vinohrady a~podobné kultúry}. Pre~správne zabezpečenie určenie rýchlostí v~týchto značkách by musela byť v~procese prítomná zložitá procedúra na~vytvorenie správne orientovanej siete vrcholov, v~ktorej by sa táto vlastnosť spomínaných značiek dala vyjadriť. V~tejto chvíli sú rýchlosti behu v~takýchto mapových značkách brané uniformne. 

Keď sú všetky polygonálne objekty pridané, prichádzajú na~rad cesty. Tie sú pridávané veľmi podobným spôsobom ako polygony, avšak je upustené od~kroku, v~ktorom sa hľadajú vnútorné vrcholy polygonu a~nastavujú sa ich hranám atribúty. Tento krok u ciest nie je potrebný, nakoľko pri~pridávaní ciest sa nemení prostredie, v~ktorom sa daná cesta nachádza.\\ 
Po pripojení cesty do~grafu sa nastavia na~hranách retiazky cesty atribúty indikujúce existenciu cesty na~danej hrane. Na~Obrázku \ref{obr24:connect_path_chain_to_vertices_of_cut_edges} je možné nahliadnuť plne pridanú cestu do~grafu

V poslednom rade sa postupne spracujú všetky líniové prekážky, ktoré~sú schopné spomaliť alebo dokonca znemožniť bežcov postup. V~tomto prípade sa nebudú ako v~predošlých prípadoch pridávať retiazky týchto objektov do~grafu. Postup spracovania týchto objektov je následovný:
\begin{itemize}
    \item V~prvom rade sa opäť vytvorí retiazka vrcholov pospájaných hranami pozdĺž celej dĺžky línie objektu.
    \item V~ďalšom kroku sa nájdu všetky hrany, ktoré~táto retiazka kríži, avšak tentokrát sa tieto hrany neodstránia.
    \item Nakoniec sa iba upravia týmto hranám atribúty tak, aby indikovali, že cez ne prechádza líniová prekážka.
\end{itemize}

Výsledok vloženia líniovej prekážky do~grafu je možné nahliadnuť v~Obr. \ref{obr25:added_linear_obstacle}

\begin{figure}[p]\centering
\frame{\includegraphics[width=250px]{img/mapRepre/innerEdgesAttributesSet2.jpg}}
\caption{Nastavenie atribútov hrán vrcholov vnútri polygonu(objektu zakázaného priestoru). Atribúty sú na~hranách vyznačené farebnými škvrnami vo farbách reprezentovaných mapových značiek.} 
\label{obr22:inner_edges_attributes_set}
\end{figure}

\begin{figure}[p]\centering
\frame{\includegraphics[width=250px]{img/mapRepre/connectPolygonChainToVerticesOfCutEdges3.jpg}}
\caption{Pridanie hrán medzi retiazkou a~vrcholmi, ktorým bola preseknutá aspoň jedna hrana a~nastavenie atribútov na~týchto hranách ako aj na~hranách retiazky. Atribúty sú na~hranách vyznačené farebnými škvrnami vo farbách reprezentovaných mapových značiek.} 
\label{obr23:connect_polygon_chain_to_vertices_of_cut_edges}
\end{figure}

\begin{figure}[p]\centering
\frame{\includegraphics[width=250px]{img/mapRepre/connectPathChainToVerticesOfCutEdges5.jpg}}
\caption{Pridaná retiazka cesty do~grafu. Na~hranách pridanej retiazky je možné vidieť malé čierne šmuhy indikujúce atribúty cesty na~daných hranách.} 
\label{obr24:connect_path_chain_to_vertices_of_cut_edges}
\end{figure}

\begin{figure}[p]\centering
\frame{\includegraphics[width=250px]{img/mapRepre/addedLinearObstacle6.jpg}}
\caption{Pridaná líniová prekážka (objektu neprekonateľného zrázu) do~grafu. Je možné si všimnúť na~hranách, ktoré~táto prekážka kríži, malé čierne šmuhy, ktoré~indikujú, že sa na~danej hrane nachádza atribút neprekonateľného zrázu.} 
\label{obr25:added_linear_obstacle}
\end{figure}

\pagebreak

\section{Orienteering (ISOM 2017-2) user model}

Užívateľský model naviazaný na~template \textit{Orienteering (ISOM 2017-2)}. Umožňuje nastaviť veľkosť spomalenia, ktoré~jednotlivé mapové značky pre~bežca vytvárajú. Na~základe tejto parametrizácie následne dokáže vypočítať čas, ktorý bežcovi zaberie prekonanie konkrétnej hrany grafu.

Príklad, ako môže vyzerať parametrizácia tohto užívateľského modelu je možné nahliadnuť v~Prílohe~\ref{uzivatelsky_model}. Táto parametrizácia vznikla v~spolupráci s~aktívnym reprezentantom Slovenskej republiky v~orientačnom behu.

\subsection{Princíp výpočtu váhy prechodu hrany z~poskytnutých atribútov}

Pri výpočte váhy konkrétnej hrany dostane užívateľský model na~vstupe atribúty počiatočného a~koncového vrcholu tejto hrany a~atribúty hrany samotnej.

Výpočet prebieha následovne:
\begin{itemize}
    \item V~prvom rade sa na~základe atribútov hrany a~koeficientov spomalenia spočítajú koeficienty spomalenia ľavého a~pravého prostredia hrany. Koeficient spomalenia je číslo medzi 0 a~1 udávajúce pomer medzi najväčšou možnou rýchlosťou behu a~spomalenou rýchlosťou behu daným symbolom na~mape. Prostredia hrany môžu byť kombináciou viacerých mapových značiek. Užívateľský model preto vynásobí koeficienty spomalenia všetkých značiek, ktoré~sa v~ľavom/pravom prostredí nachádzajú a~vráti tento produkt ako výsledný koeficient. Výsledné koeficienty spomalenia ľavého a~pravého prostredia sa následne porovnajú a~vyberie sa ten väčší z nich. Teda bežec si vyberie pre~beh po~hrane to prostredie, ktoré~je preňho efektívnejšie. 
    \item V~ďalšom kroku sa započíta efekt cesty na~hrane. Pokiaľ je na danej hrane cesta, potom efekt prostredia nehrá pre~beh žiadnu roľu a~riadime sa iba podľa koeficientu spomalenia danej cesty.\\
    Výnimkou v~tomto pravidle sú mapové značky pre \textit{menej výrazné malé chodníky} a~\textit{prieseky}. Pri~týchto symboloch priebežnosť závisí od~prostredia, v~ktorom sa nachádzajú. Preto podobne ako pri~výpočte kombinácií značiek prostredia hrany sa ich koeficient vynásobí s~vypočítaným koeficientom prostredia a~tým sa získa koeficient spomalenia týchto objektov. Podotýkame, že koeficienty týchto objektov môžu (a mali by) byť väčšie ako 1 pre~zabezpečenie vyššieho koeficientu spomalenia po~danom chodníku/prieseku než je koeficient v~samotnom prostredí. 
    \item Pokiaľ je výsledný koeficient menší než 0.15, zoberie sa do~následujúcich krokov táto hodnota namiesto vypočítaného koeficientu. Týmto spôsobom predídeme príliš malým koeficientom spomalenia. Táto technika by sa dala nahradiť vážením koeficientu podľa toho, koľko mapových značiek sa podielalo na~jeho výpočte.
    \item V~treťom kroku dopočítame čas, ktorý je potrebný na~prekonanie danej hrany. 
    \begin{itemize}
        \item Najprv zo vstupných vrcholových atribútov dopočítame dĺžku hrany medzi nimi. 
        \item Následne aplikujeme vypočítaný koeficient z~predošlých krokov na~konštantu predstavujúcu najrýchlejšie tempo, ktoré~bežec dokáže dosiahnuť. Teda túto konštantu predelíme vypočítaným koeficientom, čím dostaneme tempo, ktorým bežec pobeží v~prostredí danom hranovými atribútmi. 
        \item Nakoniec vynásobíme tempo so vzdialenosťou, ktorú musí bežec prechodom cez~hranu prekonať, a~získame čas na~prechod tejto hrany.  
    \end{itemize}
    \item V~poslednom kroku ešte musíme skontrolovať, či cez danú hranu neprechádza nejaká líniová prekážka. Pokiaľ áno, je potrebné ku času pripočítať čas, ktorý zaberie prekonanie tejto prekážky. 
    \item Výsledný čas sa vráti ako váha hrany.
    \end{itemize}

Poznámka:\\
V tejto chvíli užívateľský model pri~výpočtu doby na~prekonanie konkrétnej hrany nezapočítava faktor prevýšenia, ktoré~musí bežec pri~prechode danou hranou prekonať. Tento faktor v~aktuálnej podobe aplikácie nie je potrebné zahrňovať, nakoľko grafová reprezentácia máp nenesie žiadnu informáciu nadmorskej výške jednotlivých vrcholov a~teda nie je možné prekonané prevýšenie dopočítať. Až po~pridaní tejto funkcionality do~grafovej reprezentácie bude potrebné upraviť tento užívateľský model tak, aby dokázal prevýšenie vo svojom výpočte zohľadniť.  

\subsection{Výpočet A* heuristiky}\label{vypocet_a_star_heuristiky}

Užívateľský model je popri výpočtu váh hrán z~ích atribútov zodpovedný taktiež za výpočet heuristickej funkcie pre~A* algoritmus. Dôvodom, prečo práve užívateľský model je zodpovedný za výpočet tejto heuristiky je ten, že heuristika nezávisí iba na~samotnej topológii grafu, ale~taktiež na~rýchlosti, ktorú~dokáže bežec v~závislosti na~parametrizácii užívateľského modelu dosiahnuť na~jednotlivých typoch povrchov / mapových značkách.

Heuristika počítaná užívateľským modelom musí byť \textit{prípustná} a~\textit{monotónna} aby ju bolo možné použiť v~implementovanom A* algoritme. Vysvetlenie týchto dvoch vlastností heuristiky je možné nahliadnuť v~spodnej časti Podsekcie \ref{klasicky_a_star_algoritmus}. 

Pre dosiahnutie prípustnosti heuristickej funkcie potrebujeme, aby funkcia vždy vracala váhu menšiu ako je skutočná cena cesty z~daného vrcholu do~cieľového. Nakoľko váha hrán je počítaná v~čase na~ich prekonanie, najnižšia možná váha cesty z~daného vrcholu do~cieľového je čas, za ktorý bežec prebehne vzdialenosť medzi týmito dvomi vrcholmi po~povrchu, po~ktorom sa mu pobeží čo najlepšie. Nastavíme preto heuristickú funkciu na~hodnotu vypočítanú práve týmto spôsobom. Tým zaručíme, že hodnota heuristiky nikdy nebude vyššia než hodnota skutočnej ceny cesty a~teda zabezpečíme jej prípustnosť.

Vyššie popísaná heuristika bude taktiež monotónna, nakoľko čas, za ktorý sa dostaneme z~vrcholu priamo do~cieľa po~ideálnom povrchu bude určite menší alebo rovný času, ktorý by nám zabralo prejsť do~iného miesta na~mape po~akomkoľvek povrchu a~následne z~daného miesta po~ideálnom povrchu prísť do~cieľa.  

Výpočet heuristickej funkcie užívateľským modelom teda spočíva v:
\begin{itemize}
    \item nájdení povrchu, po~ktorom sa bežcovi beží najlepšie
    \item vypočítaní času, ktorý bežec bude potrebovať na~zdolanie vzdialenosti medzi daným vrcholom a~cieľom tempom, ktoré~dokáže dosiahnuť na~najideálnejšom povrchu 
\end{itemize}

\section{A* searching algorithm}

Algoritmom výberu pre~vyhľadávanie najrýchlejších ciest v~mapách je algoritmus A*. Tento algoritmus je klasickou prvou volbou pri~úlohách vyhľadávania ciest v~ohodnotených grafoch s~topologickými vlastnosťami. V~našom prípade nepracujeme s~grafom, ktorého váhy na~hranách na~100\% odpovedajú vzdialenostiam medzi jednotlivými vrcholmi, ale za pomoci jednoduchých myšlienok popísaných v~Podsekcii \ref{vypocet_a_star_heuristiky} dokážeme tento problém vyriešiť. 

V následujúcich podsekciách popíšeme najprv klasický pohľad na~algoritmus A* a~potom sa pozrieme na~jeho úpravy, vďaka ktorým dokáže pracovať nad naším grafom s~údajmi od~užívateľského modelu.

\subsection{Klasický A* algoritmus}\label{klasicky_a_star_algoritmus}

Články~\cite{IntrodutionToAStar} and \cite{Heuristics} popisujú A* algoritmus následovne:

A* alogritmus je nepriamou kombináciou Dijkstrovho algoritmu pre~nachádzania najkratších trás v~ohodnotených grafoch a~Greedy Best-First-Search algoritmu využívajúceho metódu na~usmernenie svojho prehľadávania.

Tajomstvo jeho úspechu spočíva v~tom, že kombinuje informácie, ktoré~používa Dijkstrov algoritmus (uprednostňuje vrcholy, ktoré~sú blízko počiatočného bodu) a~informácie, ktoré~používa Greedy Best-First-Search (uprednostňuje vrcholy, ktoré~sú blízko cieľa).
V štandardnej terminológii používané pri~hovorení o~A* znamená g(n) presné náklady cesty z~počiatočného bodu do~akéhokoľvek vrcholu n, a~h(n) predstavuje heuristický odhad nákladov z~vrcholu n do~cieľa.  A* vyváži tieto dve hodnoty, keď sa pohybuje z~počiatočného bodu do~cieľa. Každý raz v~hlavnej slučke skúma vrchol n, ktorý má najnižšiu hodnotu f(n) = g(n) + h(n).

\bigskip

Je dôležité vybrať korektnú heuristickú funkciu, aby A* hľadal najkratšie cesty v~ohodnotených grafoch. Pri~výbere heuristiky môžu nastať následujúce prípady:
\begin{itemize}
    \item Pokiaľ je h(n) = 0 pre~všetky n, potom vo vyhľadávaní zohráva roľu iba g(n) a~A* sa zmen9 na~Dijkstrov algoritmus.
    \item Pokiaľ je h(n) nižšie (alebo rovné) ceny presunu z~n do~cieľa, potom A* zaručene nájde najkratšiu cestu v~grafe. Čím je h(n) nižšie, tým viac vrcholov grafu A* prehľadá, čo spôsobí jeho spomalenie.
    \item Pokiaľ je h(n) presne rovné cene presunu z~n do~cieľa, A* bude sledovať práve tú najlepšiu cestu a~neprehľadá žiadny vrchol navyše, čo ho robí veľmi efektívnym. Avšak nie vo všetkých prípadoch je možné takúto presnosť heuristiky dosiahnuť.
    \item Pokiaľ je h(n) niekedy vyššie ako je cena presunu z~n do~cieľa, A* nezaručuje nájdenie najkratšej cesty v~grafe ale dokáže bežať rýchlejšie.
    \item Pokiaľ je však h(n) príliš vysoké v~porovnaní s~g(n), potom začne zohrávať úlohu iba váha h(n) a~A* sa stáva Greedy Best-First-Search algoritmom.
\end{itemize}

Teda pokiaľ vyžadujeme nachádzanie najkratších trás v~ohodnotenom grafe, potrebujeme, aby heuristická funkcia bola z~hora ohraničená skutočnou cenou cesty z~n do~cieľa. O~takto obmedzenej heuristike následne hovoríme, že je \textit{prípustná}.

Pokiaľ však využívame \textit{grafové prehľadávanie}, pre~zabezpečenie nájdenia najkratšej cesty v~grafe je potrebné aby heuristika bola taktiež \textit{monotónna}. Jedná sa o~istú formu trojuholníkovej nerovnosti kedy hodnota heuristiky vo vrchole n musí byť menšia alebo rovná súčtu ceny prechodu z~n do~jeho následníka a~hodnoty heuristiky v~tomto následníkovi. Táto trojuholníková nerovnosť musí platiť vo všetkých vrcholoch a~ich následníkoch. Dá sa dokázať, že každá monotónna heuristika je taktiež prípustná.

\bigskip

V Programe~\ref{AStar} je ku nahliadnutiu pseudokód možnej implementácie algoritmu A*. Tento pseudokód bol prevzatý z~Článku~\cite{AStarWiki}.

\begin{listing}
\begin{lstlisting}

function A_Star(start, goal, h)
  openSet := {start}

  cameFrom := an empty map

  gScore := map with default value of Infinity
  gScore[start] := 0

  fScore := map with default value of Infinity
  fScore[start] := h(start)

  while openSet is not empty
    current := pop vertex from openSet with lowest fScore value
    if current = goal
      return reconstruct_path(cameFrom, current)

    openSet.Remove(current)
    for each neighbor of current
      tentative_gScore := gScore[current] + d(current, neighbor)
      if tentative_gScore < gScore[neighbor]
        cameFrom[neighbor] := current
        gScore[neighbor] := tentative_gScore
        fScore[neighbor] := tentative_gScore + h(neighbor)
        if neighbor not in openSet
          openSet.add(neighbor)
  return failure

function reconstruct_path(cameFrom, current)
    total_path := {current}
    while current in cameFrom.Keys:
        current := cameFrom[current]
        total_path.prepend(current)
    return total_path

\end{lstlisting}
\caption{A* algoritmus}
\label{AStar}
\end{listing}

\pagebreak

\subsection{Špecifické vlastnosti algoritmu A* v~aplikácii.}

Implementácia A* v~aplikácii nesie následujúce špecifiká:

\begin{itemize}
    \item Algoritmus na~vstupe namiesto heuristickej funkcie dostane inštanciu užívateľského modelu, ktorý bude hodnotu heuristickej funkcie pre~algoritmus počítať. Zároveň tento užívateľský model je zodpovedný za počítanie váh pre~hrany grafu, nakoľko grafová reprezentácia, nad~ktorou~algoritmus pracuje, nesie iba informácie o~vrcholových a~hranových atribútoch, je potrebné, aby užívateľský model tieto atribúty spracoval do~váh, s~ktorými následne už A* dokáže pracovať.   
    \item A* využívaný v~aplikácii je implementovaný ako graph search. Z~tohto dôvodu je potrebné, aby využívaná heuristika počítaná užívateľským modelom bola okrem \textit{prípustnosti} taktiež \textit{monotónna}.
    \item Naša implementácia A* si neudržiava kolekciu predchodcov navštívených vrcholov a~teda z~dôvodu možnosti rekonštrukcie nájdenej cesty vyžaduje po~vrcholoch vstupného grafu, aby boli schopné si zapamätať svojho predchodcu. Keď je A* narazí na~cieľový vrchol, začne sa dotazovať postupne vrcholov na~ich predchodcov dokým nenarazí na~počiatočný vrchol. Po~prevrátení postupnosti predchodcov vrcholov získa retiazku reprezentujúcu nájdenú cestu z~počiatku do~cieľa. 
\end{itemize}



\chapter*{Závěr}
\addcontentsline{toc}{chapter}{Závěr}

Cielom tejto bakalárskej práce bolo vytvoriť aplikáciu, v~ktorej bude možné vyhľadávať cesty v~otvorenom teréne na~základe dodaného mapového súboru. Vyhľadávanie ciest malo byť konkrétne implementované pre~mapy z~prostredia športového odvetvia \textit{orientačný beh}. 

Práca mala zahrňovať návrh architektúry aplikácie a~jej samotnú implementáciu, špecifickú implementáciu na~spracovávanie mapových súborov z~oblasti orientačného behu, vytvorenie vyhľadávacieho algoritmu, ktorý~by dokázal vyhľadávať na~vytvorenej mapovej reprezentácii a~možnosť užívateľa zahrnúť svoje preferencie do~procesu vyhľadávania. 

Tieto úlohy boli úspešne naplnené. 
\begin{itemize}
    \item Bol vytvorený kvalitný návrh aplikácie na~základe MVVM(MV) návrhového vzoru. Na~jeho základe bol následne implementovaný mechanizmus zabezpečujúci logiku za~vyhľadávaním ciest v~mapách a~taktiež mechanizmus hlavného okna, v~ktorom je možné spravovať nastavenia určené pre~celú aplikáciu. Následne bolo vytvorených 9 oblastí slúžiacich ku získavaniu a~spracovávaniu dátových štruktúr. Celý program je navrhnutý takým spôsobom, aby~bol čo najľahšie rozšíriteľný, či~už v~oblasti aplikačnej logiky alebo~typov dátových štruktúr, s~ktorými dokáže aplikácia pracovať. Vďaka MVVM(MV) architektúre je aplikáciu jednoduché udržovať a~modifikovať.
    \item Pre vyhĽadávanie ciest v mapách pre orientačný beh bol vytvorený
    \begin{itemize}
        \item parser pre mapy formátu OMAP a grafické znázornenie týchto máp,
        \item template zahrňujúci mapové značky dôležité pri vyhľadávaní najrýchlejších ciest v mapách pre orientačný beh,
        \item užívateľský model, ktorý je možné parametrizovať na základe preferencií užívateľa
        \item grafová reprezentácia mapy pre orientačný beh, na ktorej sa bude vyhľadávať najrýchlejšia cesta. V tejto chvíli bohužiaľ bez spracovania a aplikovania výškových dát. 
        \item vyhľadávací algoritmus A*, ktorý je používaný pre vyhľadávanie trás v grafových reprezentáciách máp
    \end{itemize}
\end{itemize}

Vyhľadávanie samotné dodáva uspokojujúce výsledky. Hlavným problémom, ktorý nebol doposiaľ vyriešený je zakomponovanie informácií o nadmorskej výške do grafovej reprezentácie mapy a spracovanie tejto informácie užívateľským modelom. Preto je vyhľadávanie trás v tejto chvíli možné iba na relatívne plochých mapách, kde dokáže vracať zmysluplné výsledky.\\ 
O výsledkoch vyhľadávania by bolo vhodné spraviť exaktnejší výskum, založený na názoroch skúsených orientačných bežcov, avšak z časových dôvodov nebolo možné takýto výskum uskutočniť. Samotné uskutočnenie by taktiež bolo celkom zložité, nakoľko kvalita nájdených trás v mnohých ohľadoch závisí na subjektívnom pohľade jednotlivých bežcov. Z tohto dôvodu sú v aplikácii prítomné užívateľské modely, aby si každý užívateľ mohol vyhľadávanie nastaviť podľa vlastných preferencií.


%%% Seznam použité literatury
\include{literatura}

%%% Obrázky v práci
%%% (pokud jich je malé množství, obvykle není třeba seznam uvádět)
% \listoffigures

%%% Tabulky v práci (opět nemusí být nutné uvádět)
%%% U matematických prací může být lepší přemístit seznam tabulek na začátek práce.
% \listoftables

%%% Použité zkratky v práci (opět nemusí být nutné uvádět)
%%% U matematických prací může být lepší přemístit seznam zkratek na začátek práce.

% \chapwithtoc{Seznam použitých zkratek}
% \begin{itemize}
    % \item OB - orientačný beh
    % \item GUI - grafické užívateľské rozhranie (graphical user interface)
    % \item XAML - Extension application markup language
    % \item MVVM - Model-View-ViewModel
% \end{itemize}



%%% Součástí doktorských prací musí být seznam vlastních publikací

% \ifx\ThesisType\TypePhD
% \chapwithtoc{Seznam publikací}
% \fi

%%% Přílohy k práci, existují-li. Každá příloha musí být alespoň jednou
%%% odkazována z vlastního textu práce. Přílohy se číslují.
%%%
%%% Do tištěné verze se spíše hodí přílohy, které lze číst a prohlížet (dodatečné
%%% tabulky a grafy, různé textové doplňky, ukázky výstupů z počítačových programů,
%%% apod.). Do elektronické verze se hodí přílohy, které budou spíše používány
%%% v elektronické podobě než čteny (zdrojové kódy programů, datové soubory,
%%% interaktivní grafy apod.). Elektronické přílohy se nahrávají do SISu.
%%% Povolené formáty souborů specifikuje opatření rektora č. 72/2017.
%%% Výjimky schvaluje fakultní koordinátor pro zavěrečné práce.
\appendix
\chapter{Přílohy}

\section{Užívateľská dokumentácia}\label{uzivatelska_dokumentacia}

\subsection{Popis}

Aplikácia s~grafickým užívateľským rozhraním slúžiaca na~hľadanie ciest v~otvorenom teréne na~základe dodaného mapového súboru.

Skladá sa z~dvoch častí: hlavného okna a session-u pre vyhľadávanie ciest v~mape. Hlavné okno zabezpečuje:
\begin{itemize}
    \item spúšťanie jednotlivých session-ov
    \item nastavenie jazyka aplikácie
    \item nastavenie konfiguráci mapových reprezentácií, vyhľadávacích algoritmov a užívateľských modelov
    \item nastavenie implicitnej distribúcie výškových dát a~konfigurácia výškových dát
    \item ukončenie aplikácie zatvorením okna
\end{itemize}
Session vyhľadávajúci cesty v~mapách zabezpečuje:
\begin{itemize}
    \item výber parametrov, na základe ktorých sa bude vyhľadávanie uskutočňovať
    \item vytvorenie mapovej reprezentácie, v~ktorej sa bude hľadať cesta
    \item interaktívne zadávanie hľadanej trasy v~mape a~vykresľovanie nájdených ciest užívateľovi
\end{itemize}

\subsection{Zachádzanie s aplikáciou}

\subsubsection{\underline{Spustenie aplikácie}}

Aplikácia sa spúšťa pomocou spustitelného súboru \textit{Optepafi.exe}, vytvoreného kompiláciou zdrojového kódu projektu \textit{Optepafi.csproj} v prostredí operačného systému Windows. Po spustení sa na~obrazovke objaví hlavé okno s~hlavným menu.

\subsubsection{\underline{Ukončenie behu aplikácie}}

Aplikácia sa ukončuje pomocou zatvorenia hlavného okna. To sa v~prípade otvorených session-ov môže užívateľa dotázať pomocou dialógového okna, či naozaj chce aplikáciu opustiť.

\pagebreak

\subsubsection{\underline{Hlané okno}}

\subsubsection{Hlané menu}

\begin{figure}[h]\centering
\frame{\includegraphics[width=250px]{usr_img/hlavne_menu}}
\end{figure}

V hlavnom menu je možné vidieť dve tlačidlá:
\begin{itemize}
    \item tlačidlo v~pravom dolnom rohu slúži pre~prechod do~hlavných nastavení aplikácie
    \item tlačidlo uprostred okna slúži pre~vytváranie inštancií session-u pre~vyhľadávanie ciest v~mapách
\end{itemize}

\subsubsection{Hlané nastavenia}

\begin{figure}[h]\centering
\frame{\includegraphics[width=250px]{usr_img/hlavne_nastavenia}}
\end{figure}

V hlavných nastaveniach môžeme:
\begin{itemize}
    \item vybrať jazyk, ktorým chceme aby~na~nás aplikácia rozprávala. Je možné zvoliť buď slovenskú alebo anglickú lokalizáciu aplikácie.
    \item prejsť stlačením príslušného tlačidla do~konfigurácie výškových dát. Pod týmto tlačidlom môžeme vidieť aktuálne vybranú, implicitne využívanú distribúciu výškových dát.
    \item nastaviť konkrétnemu typu mapovej reprezentácie, vyhľadávacieho algoritmu alebo užívateľského modelu parametre, podľa ktorých sa bude pri vyhľadávaní ciest riadiť. V tejto chvíli sú nastaviteľné parametre dostupné iba pre mapovú reprezentáciu \textit{Complete, net intertwining map representation}. 
    
    Pre tú je možné nastaviť štandardnú dĺžku hrán vo vytvorenom grafe a pomer najkratších povolených hrán ku štandardnej dĺžke hrany v retiazkach vrcholov hraníc mapových objektov vpletených do siete grafu. Tvar siete grafu je možné v tejto chvíli nastaviť ib na \textit{Triangular}. 
    \item stlačením tlačidla v~pravom dolnom rohu sa opäť vrátiť do~hlavného menu aplikácie.
\end{itemize}

\subsubsection{Konfigurácia výškových dát}

\begin{figure}[h]\centering
\frame{\includegraphics[width=250px]{usr_img/konfig_vyskovych_dat}}
\end{figure}

V~konfigurácii výškových dát môžeme sťahovať a~odstraňovať stiahnuté výškové dáta z~rôznych zdrojov. Dáta sú sťahované po~regiónoch. Posledne konfigurovaná distribúcia sa pri~návrate do~hlavných nastavení nastaví ako~implicitne používaná distribúcia v~aplikácii.

Aplikácia farebne indikuje, ktoré regióny sú stiahnuté (zelená), ktoré niesu (biela), ktoré sú aktuálne sťahované (oranžová) a~ktoré odstraňované (červená).

Konfigurácia výškových dát umožňuje:
\begin{itemize}
    \item Pomocou vrchného boxu vybrať dátovú distribúciu, s~ktorej~dátami chce užívateľ manipulovať (sťahovať alebo odstraňovať ich).
    \item Po vybraní distribúcie sa na~obrazovke zobrazia regióny a~ich pod-regióny
    \item Následne po vybraní regiónu ho môže užívateľ na~základe jeho aktuálneho stavu stiahnuť, odstrániť alebo~prerušiť jeho sťahovanie
    \item Sťahovanie regiónov sa riadi následujúcimi pravidlami:
    \begin{itemize}
        \item Keď sa sťahuje konkrétny región, sťahujú sa aj všetky jeho pod-regióny
        \item Pokiaľ je región odstraňovaný, odstraňujú sa aj všetky jeho pod-regióny. Zároveň ak niektorí z~jeho nad-regiónov boli doteraz stiahnutí, zmení sa ich stav na~nestiahnuté (nejaká ich časť bola odstránená).
        \item Kedykoľvek je s regiónom manipulované, či už priamo alebo nepriamo, nie je možné na~ňom v~danej chvíli robiť žiadne operácie. Pokiaľ je región sťahovaný priamo, je možné dané sťahovanie zrušiť.
    \end{itemize}
    \item Pokiaľ je ku~stiahnutiu výškových dát vybranej distribúcie potrebná autorizácia užívateľa, je nutné zadať užívateľské meno a~heslo. Pomocou týchto údajov bude možné následne získať výškové dáta z danej distribúcie.
    \item Po~dokončení konfigurácie je možné sa pomocou stlačenia tlačidla v~pravom dolnom rohu vrátiť naspäť do~hlavných nastavení. Posledne konfigurovaná distribúcia sa nastaví ako aktuálne implicitne používaná.
\end{itemize}
Nakoľko plnohodnotné sťahovanie výškových dát doposiaľ nieje v aplikácii poriadne implementované, je možné si zatiaľ vybrať z~dvoch ukážkových zdrojov výškových dát:
\begin{itemize}
    \item \textit{Simulating elevation data source} - Zdroj simulujúci prácu sťahovania a~odstraňovania výškových dát. Skladá sa z~dvoch dátových distribúcií:
    \begin{itemize}
        \item \textit{ Authorization simulating elevation data distribution with~name 'Name' and~password 'Password'} - Simulujúca distribúcia výškových dát, ktorá vyžaduje autorizáciu užívateľa. Užívateľské meno je slovo \textit{Name} a~heslo je slovo \textit{Password}. Obsahuje tri regióny, ktoré~užívateľ môže \uv{sťahovať a~odstraňovať}.
        \item \textit{No authorization simulating elevation data distribution} - Simulujúca distribúcia výškových dát, ktorá~nevyžaduje autorizáciu užívateľa. Obsahuje tri regióny, ktoré~užívateľ môže \uv{sťahovať a~odstraňovať}.
    \end{itemize}
    \item \textit{Not sufficient elevation data source} - Zdroj výškových dát, ktorý~nikdy nedokáže zabezpečiť požadované výškové dáta. Obsahuje jednu dátovú distribúciu:
    \begin{itemize}
        \item \textit{Not sufficient elevation data distribution} - Distribúcia výškových dát, ktorá~nikdy nie je schopná dodať potrebné výškové dáta. Nedefinuje žiadny vlastný región.
    \end{itemize}
\end{itemize}

\pagebreak

\subsubsection{\underline{Session pre vyhľadávanie ciest}}

Session pre vyhľadávanie ciest je možné vytvoriť v hlavnom menu hlavného okna. Po vytvorení sa zobrazia nastavenia session-ových parametrov. 

\subsubsection{Nastavenia parametrov}

\begin{figure}[h]\centering
\frame{\includegraphics[width=300px]{usr_img/path_finding_settings}}
\end{figure}

V nastaveniach parametrov je možné vybrať template, mapový súbor, typ mapovej reprezentácie, súbor s~užívateľským modelom a~vyhľadávací algoritmus, ktoré budú použité v procesu vyhľadávania ciest. Na začiatku sú tieto zdroje implicitne nastavené na posledne použité v~predchádzajúcej inštancii session-u.

Nastavovanie parametrov sa riadi určitými pravidlami:
\begin{itemize}
    \item najprv sa vyberá template a~mapový súbor, následne súbor s~užívateľským modelom a typ mapovej reprezentácie, a~na~koniec vyhľadávací algoritmus
    \item pre zvolenú kombináciu template-u a~formátu mapového súboru musí platiť:
    \begin{itemize}
        \item musí existovať mapová reprezentácia, ktorá~túto kombináciu dokáže spracovať,
        \item taktiež pre~zvolený template-u musí existovať typ užívateľského modelu, ktorý~dokáže spracovávať atribúty definované týmto template-om,
        \item nakoniec musí existovať aspoň jedna kombinácia takto definovanej mapovej reprezentácie a~užívateľského modelu, ktorú dokáže využiť aspoň jedna implementácie ľubovoľného vyhľadávacieho algoritmu.
    \end{itemize}
    \item keď je vybraný buďto template alebo~mapový súbor a~výber druhej položky by~spôsobil neplatnú kombináciu, prvá položka sa opäť vynuluje.
    \item súbor s~užívateľským modelom môže byť vybraný len takého typu, ktorý dokáže spracovávať atribúty definované zvoleným template-om a~ktorý spolu s~vybranou mapovou reprezentáciou, je vhodnou kombináciou pre aspoň jednu implementáciu nejakého vyhľadávacieho algoritmu.
    \item typ mapovej reprezentácie musí obsahovať implementáciu, ktorá dokáže spracovať dvojicu vybraného template-u a formát vybranej mapy. Zároveň s vybraným užívateľským modelom musia tvoriť kombináciu, ktorú dokáže využiť aspoň jedna implementácia nejakého vyhľadávacieho algoritmu.
    \item vyhľadávací algoritmus následne môže byť zvolený len taký, ktorý~podporuje typ vybraného užívateľského modelu a~mapovej reprezentácie.
\end{itemize}

Na obrazovke je ďalej možné pozorovať názvy vybraných súborov, ako aj grafický náhľad vybranej mapy.

Po vybraní všetkých parametrov môže užívateľ pomocou stlačenia tlačidla v~dolnom rohu pokračovať do~ďalšej časti mechanizmu, ktorou je vytváranie mapovej reprezentácie.

Je možné vybrať si z~následujúcich parametrov:
\begin{itemize}
    \item \textbf{Template} - pre vyhľadávanie trás v mapách pre orientačný beh je potrebné zvoliť template \textit{Orienteering (ISOM 2017-2)} 

    - náslene sú na~výber dva ukážkové template-y, demonštrujúce mechaniky aplikácie:
    \begin{itemize}
        \item \textit{Blank template} - Template, ktorý reprezentuje prázdne vrcholové a~hranové atribúty. Slúži pre~demonštráciu aplikačných mechanizmov.
        \item \textit{Not usable template} - Demonštračný template, ktorý nie je možné využiť pri tvorbe žiadnej mapovej reprezentácie a~ani preňho neexistuje zodpovedajúci typ užívateľského modelu.
    \end{itemize}
    \item \textbf{Mapa} - pre vyhľadávanie trás v mapách pre orientačný beh je potrebné zvoliť mapový súbor formátu \textit{.omap}. Tento mapový súbor by mal mať symbolový set rovnaký ako je základný symbolový novo vytvorenej mapy aplikáciou \textit{OpenOrienteeringMapper}. Pokiaľ sa symbolový set bude líšiť, je možné, že~sa niektoré mapové značky budú nesprávne interpretovať.
    
    - pre účely demonštrácie funkcionalít aplikácie je taktiež možné vybrať aj klasický \textit{textový} súbor. Na obrazovku budú následne vypísané náhodne všetky slová z~vybraného textového súboru.
    \item \textbf{Užívateľský model} - pre vyhľadávanie trás v mapách pre orientačný beh je potrebné vybrať užívateľský model, ktorého suffix tvorí \textit{.ori17UM.json} indikujúci užívateľský model ktorý je viazaný na template \textit{Orienteering (ISOM 2017-2)}. 
    
    - pri použití ukážkového templateu \textit{Blank template} je možné taktiež vybrať užívateľský model typu \textit{Blank user model}. Podobne ako naviazaný template, ani tento užívateľský model nezabezpečuje žiadnu funkcionalitu. Slúži pre demonštráciu aplikačných mechanizmov.
    \item \textbf{Mapová reprezentácia} - pre vyhľadávanie trás v mapách pre orientačný beh je potrebné vybrať mapovú reprezentáciu typu \textit{Complete, net intertwining map representation}.

    - pri použití ukážkového template-u \textit{Blank template} alebo textového mapového formátu je možné využiť aj mapovú reprezentáciu \textit{Blank representation}. Táto mapová reprezentácia slúži pre demonštráciu mechanizmov aplikácie.   
    \item \textbf{Vyhľadávací algoritmus} - vyhľadávanie trás v mapách pre orientačný beh je sprostredkovávané algoritmom \textit{A*}.
    
    - pre~účely demonštrácie funkcionalít aplikácie bol vytvorený taktiež mechanizmus s názvom \textit{Smiley face drawer}. Ten počas svojho behu postupne vykresluje usmievajúce sa tváre medzi jednotlivými bodmi zadanej trate. Na záver sú všetky tváre naraz vykreslené.
\end{itemize}

\subsubsection{Vytváranie mapovej reprezentácie}

\begin{figure}[h]\centering
\frame{\includegraphics[width=150px]{usr_img/map_repre_creating_dialog}}
\end{figure}

S~vytváraním mapovej reprezentácie môže užívateľ interagovať pomocou dialógového okna. Vytváranie mapovej reprezentácie sa skladá z~dvoch častí:
\begin{itemize}
    \item \textit{kontrola podmienok} - v~tejto časti sa kontrolujú podmienky pre~vytvorenie mapovej reprezentácie. Pokiaľ sa naskytne nejaký problém, užívateľ ho môže riešiť za~pomoci interakcie s~dialógovým oknom.
    \item \textit{vytváranie reprezentácie} - ak sú všetky podmienky splnené, prichádza na~rad samotné vytváranie mapovej reprezentácie. Tento proces môže byť zdĺhavý a~preto je užívateľovi umožnené sledovať postup vytvárania, ako~aj možnosť vytváranie prerušiť.
\end{itemize}

Pri akomkoľvek neúspechu vytvárania je užívateľ vrátený do~nastavení parametrov. Pokiaľ vytváranie prebehne úspešne, je užívateľ pustený do~cesty-vyhľadávajúceho okna.

\subsubsection{Vyhľadávanie cesty}

\begin{figure}[h]\centering
\frame{\includegraphics[width=300px]{usr_img/path_finding}}
\end{figure}

Vyhľadávanie cesty je cyklus zložený z~troch fáz:
\begin{enumerate}
    \item \textbf{zadanie trasy} - V~prvom rade je potrebné zadať trasu, pre~ktorú sa má v~mape vyhľadať najrýchlejšia cesta. Užívateľ zadáva body trasy na~mape, poprípade ich môže mazať od~konca tlačidlom umiestnenýmv pravom dolnom rohu. Keď je s~výberom trasy spokojný, stlačením príslušného tlačidla prejde do~fázy vyhľadávania cesty.
    \item \textbf{vyhľadávanie cesty} - Po výbere trasy prichádza na~rad samotné vyhľadávanie cesty. Proces vyhľadávania môže byť dynamicky zobrazovaný užívateľovi. Užívateľ taktiež môže prerušiť proces vyhľadávania príslušným tlačidlom. Pokiaľ je vyhľadávanie prerušené, je cyklus vyhľadávania cesty navrátený do~prvej fáze výberu trasy. Pokiaľ sa vyhľadávanie úspešne dokončí, prejde sa ku~fáze zobrazenia nájdenej cesty.
    \item \textbf{zobrazenie nájdenej cesty} - Po úspešnom dokončení vyhľadávania je užívateľovi zobrazená nájdená cesta a~jej doplňujúce informácie. Po dokončení prezerania výslednej cesty sa užívateľ stlačením príslušného tlačidla môže navrátiť do~prvej fáze vyhľadávacieho cyklu.
\end{enumerate}

Pokiaľ užívateľ hodlá session pre~vyhľadávanie ciest opustiť, môže tak učiniť kedykoľvek pomocou stlačenia tlačidla v~pravom hornom rohu obrazovky. 

\section{Príklad parametrizácie užívateľského modelu}\label{uzivatelsky_model}

\textbf{Plošné značky + cesty a schody:}

\begin{itemize}
    \item koeficient spomalenia: 0 znamená, že plocha/línia je nepriechodná, 1 znamená plná rýchlosť, ktorú bežec dokáže dosiahnuť
    \item ak ide o kombináciu značiek, použije sa násobok koeficientov
    \item značky ciest sa nekombinujú so značkami okolitého prostredia s výnimkou značiek 507 a 508
    \item pri značkách 507 a 508 nie je jedno, v akom prostredí sa tieto značky nachádzajú. Ich koeficienty sú teda nastavené na >1 a sú kombinované s koeficientom okolitého prostredia.
\end{itemize}

\textit{Terénne útvary}\\
113 Rozbitý povrch (A): 0.95\\
114 Veľmi rozbitý povrch (A): 0.80

\textit{Skaly a balvany}\\
206 Obrovský balvan alebo skalná veža (A): 0\\
208 Balvanové pole (A): 0.70\\
209 Husté balvanové pole (A): 0.55\\
210 Kamenistý povrch, pomalý beh (A): 0.65\\
211 Kamenistý povrch, chôdza (A): 0.50\\
212 Kamenistý povrch, boj (A): 0.35\\
213 Piesočný povrch (A): 0.70\\
214 Holá skala (A): 0.95

\textit{Voda a močiare}\\
301 Neprekonateľná vodná plocha (A): 0\\
302 Plytká vodná plocha (A): 0.50\\
307 Neprekonateľný močiar (A): 0\\
308 Močiar (A): 0.80\\
310 Nevýrazný močiar (A): 0.90

\textit{Vegetácia}\\
401 Otvorená krajina (A): 0.97\\
402 Otvorená krajina s rozptýlenými stromami (A): 0.93\\
403 Drsná otvorená krajina (A): 0.77\\
404 Drsná otvorená krajina s rozptýlenými stromami (A): 0.70\\
405 Les (A): 0.90\\
406 Vegetácia: pomalý beh (A): 0.70\\
407 Vegetácia (podrast): pomalý beh, dobrá viditeľnosť (A): 0.80\\
408 Vegetácia: chôdza (A): 0.45\\
409 Vegetácia (podrast): chôdza, dobrá viditeľnosť (A): 0.60\\
410 Vegetácia: boj (A): 0.20\\
412 Obrábaná pôda (A): 0.80\\
413 Sad (A): 0.93\\
413 Drsný Sad (A): 0.80\\
414 Vinohrady alebo podobné (kultúry) (A): 0.80\\
414 Drsné vinohrady alebo podobné (kultúry) (A): 0.70

\textit{Cesty}\\
501 Spevnená plocha (A): 1\\
502 Široká cesta (L): 1\\
503 Cesta (L): 1\\
504 Vozová cesta (L): 1\\
505 Chodník (L): 0.99\\
506 Malý chodník (L): 0.97\\
507 Menej výrazný malý chodník (L): 1.07\\
508 Priesek alebo líniová trasa terénom (L): 1.07

\textit{Umelé objekty}\\
520 Zakázaný priestor (A): 0\\
521 Budova (A): 0\\
532 Schodisko (L): 0.93\\
709 Neprístupná oblasť (A): 0

\bigskip

\textbf{Líniové prekážky}

\begin{itemize}
    \item približný čas prekonania v sekundách
    \item ak je línia neprekonateľná, čas je nastavený na milión sekúnd
\end{itemize}

\textit{Terénne tvary}\\
104 Zemný zráz (L): 2\\
105 Zemný násyp (L): 2\\
107 Erózna ryha (L): 4

\textit{Skaly a balvany}\\
201 Neprekonateľný zráz (L): 1 000 000\\
202 Zráz (L): 5\\
215 Zákop (L): 1

\textit{Voda a močiare}\\
304 Prekonateľný vodný tok (L): 3\\
305 Malý prekonateľný vodný tok (L): 2

\textit{Umelé objekty}\\
513 Múr (L): 2\\
515 Neprekonateľný múr (L): 1 000 000\\
516 Plot (L): 2\\
518 Neprekonateľný plot (L): 1 000 000\\
528 Výrazný líniový objekt (L): 1.5\\
529 Výrazný neprekonateľný líniový objekt (L): 1 000 000

\end{document}
