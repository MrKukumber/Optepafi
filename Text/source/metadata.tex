%%% Vyplňte prosím základní údaje o závěrečné práci (odstraňte \xxx{...}).
%%% Automaticky se pak vloží na všechna místa, kde jsou potřeba.

% Druh práce:
%	"bc" pro bakalářskou
%	"mgr" pro diplomovou
%	"phd" pro disertační
%	"rig" pro rigorozní
\def\ThesisType{bc}

% Název práce v jazyce práce (přesně podle zadání)
\def\ThesisTitle{Trasy v mapách pro orientační běh}

% Název práce v angličtině
\def\ThesisTitleEN{Navigation in orienteering maps}

% Jméno autora (vy)
\def\ThesisAuthor{Matej Kukurugya}

% Rok odevzdání
\def\YearSubmitted{2024}

% Název katedry nebo ústavu, kde byla práce oficiálně zadána
% (dle Organizační struktury MFF UK:
% https://www.mff.cuni.cz/cs/fakulta/organizacni-struktura,
% případně plný název pracoviště mimo MFF)
\def\Department{Informatický ústav Univerzity Karlovy}
\def\DepartmentEN{Computer science institute of Charles University}

% Jedná se o katedru (department) nebo o ústav (institute)?
\def\DeptType{ústav}
\def\DeptTypeEN{institute}

% Vedoucí práce: Jméno a příjmení s~tituly
\def\Supervisor{RNDr. Ondřej Pangrác, Ph.D.}

% Pracoviště vedoucího (opět dle Organizační struktury MFF)
\def\SupervisorsDepartment{Informatický ústav Univerzity Karlovy}
\def\SupervisorsDepartmentEN{Computer science institute of Charles University}

% Studijní program (kromě rigorozních prací)
\def\StudyProgramme{Informatika (B0613A140006)}

% Nepovinné poděkování (vedoucímu práce, konzultantovi, tomu, kdo
% vám nosil pizzu a vařil čaj apod.)
\def\Dedication{
    Rád by som poďakoval RNDr. Ondřejovi Pangrácovi, Ph.D. za ochotu a pomoc pri vedení bakalárskej práce. Taktiež by som rád poďakoval svojím rodičom za ich podporu a vytvorenie podmienok, vďaka ktorým bolo možné tvorbu baklárskej práce započať a dokončiť. Zároveň by som rád poďakoval svojej priateľke za pomoc pri editácii textu bakalárky a cenné konzultácie v oblasti orientačného behu. Špeciálne poďakovanie patrí klubom Šk Sandberg a KOBRA Bratislava za zapožičanie zdrojových súborov máp, na ktorých bola aplikácia testovaná.  
}

% Abstrakt (doporučený rozsah cca 80-200 slov; nejedná se o zadání práce)
\def\Abstract{
Záměrem této bakalářské práce bylo vytvořit aplikaci, která bude na základě mapových dat hledat vhodnou trasu v prostředí orientačního běhu. Tato tvorba byla rozdělena do dvou částí. První čast obsahovala vytvoření návrhu a naprogramování samotné aplikace. Objektový návrh je tím pádem detailně promyšlen a propracován. Druhá část se skládala z vytvoření implementací jednotlivých konceptů aplikace, které se společne podílejí na spracovávání mapových souborů z prostředí orientačního běhu do grafových podob a následném vyhledávání nejrychlejších tras na takto vytvořených mapových reprezentacích. Tato práce tedy primárně pojednáva o myšlenkách aplikovaných v průběhu tvorby aplikace a implementacích konceptů využívaných v aplikaci.
}

% Anglická verze abstraktu
\def\AbstractEN{
The aim of~this bachelor's thesis was to create an~application that would find a~suitable route in~the~environment of~orienteering based on map data. This development was divided itno two parts. The first part involved designing and progamming the application itself. The objcet-oriented design was therefore thoroughly thought out and elaborated. The second part cosisted of creating the implementations of the individual concepts of the application, which together contribute to processing of map files from the orienteering environment into graph representations and subsequently searching for the fastest routes on these created map representations. This thesis primarily discusses the ideas applied during the development of the application and the implementations of the concepts used within the application.
}

% 3 až 5 klíčových slov (doporučeno) oddělených \sep
% Hodí se pro nalezení práce podle tématu.
\def\ThesisKeywords{
návrh aplikace\sep MVVM\sep hledáni tras v otevřeném terénu\sep orientační běh
}

\def\ThesisKeywordsEN{
application design\sep MVVM\sep path finding in open terrain\sep orienteering
}

% Pokud některá z položek metadat obsahuje TeXové řídící sekvence, je potřeba
% dodat i verzi v obyčejném textu, která se objeví v metadatech formátu XMP
% zabudovaných do výstupního souboru PDF. Pokud si nejste jistí, prohlédněte si
% vygenerovaný soubor thesis.xmpdata.
\def\ThesisAuthorXMP{\ThesisAuthor}
\def\ThesisTitleXMP{\ThesisTitle}
\def\ThesisKeywordsXMP{\ThesisKeywords}
\def\AbstractXMP{\Abstract}

% Máte-li dlouhý abstrakt a nechceme se mu vejít na informační stranu,
% můžete použít toto nastavení ke zmenšení písma informační strany.
% (Uvažte nicméně zkrácení abstraktu, to je často lepší.)
\def\InfoPageFont{}
%\def\InfoPageFont{\small}  % odkomentujte pro zmenšení písma
