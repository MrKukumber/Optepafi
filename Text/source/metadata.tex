%%% Vyplňte prosím základní údaje o závěrečné práci (odstraňte \xxx{...}).
%%% Automaticky se pak vloží na všechna místa, kde jsou potřeba.

% Druh práce:
%	"bc" pro bakalářskou
%	"mgr" pro diplomovou
%	"phd" pro disertační
%	"rig" pro rigorozní
\def\ThesisType{bc}

% Název práce v jazyce práce (přesně podle zadání)
\def\ThesisTitle{Trasy v mapách pro orientační běh}

% Název práce v angličtině
\def\ThesisTitleEN{Navigation in orienteering maps}

% Jméno autora (vy)
\def\ThesisAuthor{Matej Kukurugya}

% Rok odevzdání
\def\YearSubmitted{2024}

% Název katedry nebo ústavu, kde byla práce oficiálně zadána
% (dle Organizační struktury MFF UK:
% https://www.mff.cuni.cz/cs/fakulta/organizacni-struktura,
% případně plný název pracoviště mimo MFF)
\def\Department{Informatický ústav Univerzity Karlovy}
\def\DepartmentEN{Computer science institute of Charles University}

% Jedná se o katedru (department) nebo o ústav (institute)?
\def\DeptType{ústav}
\def\DeptTypeEN{institute}

% Vedoucí práce: Jméno a příjmení s~tituly
\def\Supervisor{RNDr. Ondřej Pangrác, Ph.D.}

% Pracoviště vedoucího (opět dle Organizační struktury MFF)
\def\SupervisorsDepartment{Informatický ústav Univerzity Karlovy}
\def\SupervisorsDepartmentEN{Computer science institute of Charles University}

% Studijní program (kromě rigorozních prací)
\def\StudyProgramme{Informatika (B0613A140006)}

% Nepovinné poděkování (vedoucímu práce, konzultantovi, tomu, kdo
% vám nosil pizzu a vařil čaj apod.)
\def\Dedication{
    Rád by som poďakoval RNDr. Ondřejovi Pangrácovi, Ph.D. za ochotu a pomoc pri vedení bakalárskej práce. Taktiež by som rád poďakoval svojím rodičom za ich podporu a vytvorenie podmienok, vďaka ktorým bolo možné tvorbu baklárskej práce započať a dokončiť. 
}

% Abstrakt (doporučený rozsah cca 80-200 slov; nejedná se o zadání práce)
\def\Abstract{
Původním záměrem této bakalářské práce bylo vytvořit aplikaci, která bude na základě mapových dat hledat vhodnou trasu v prostředí orientačního běhu. Tento záměr však nebyl plně naplněn, neboť až příliš času zabralo vytvoření návrhu a naprogramování samotné aplikace. Z toho důvodu nezbyl čas na vytvoření vyhledávacích alogirtmů ani mechanizmů na zpracováni map, na kterých by se cesty hledaly. Z tohoto důvodu vznikla jenom kostra, která v této chvíli sama o sobě neobsahuje žádnou konrkétní funkcionalitu. Oproti tomu objektový návrh je detailně promyšlen a propracován. 
Tato bakalářská práce tedy v konečném důsledku pojednává právě o objektovém návrhu vytovřené aplikace a myšlenkách aplikovaných v průběhu jeho tvorby. 
}

% Anglická verze abstraktu
\def\AbstractEN{
The prior intention of this bachelor thesis was creating and application which based on map data will look for suitable route in orienteering environment. This intention was not completly fulfilled though. Creating and programming of application itsel took to much time so there was left none for creating of usable searching algorithms nor mechanisms for processing of maps on which routes would be looked for. For this reason the skelet of application was created which currently does contain none specific functionality. On the other side design of application is thought out and elaborated in detail.
This bachelor thesis talks in the end specifically about design of created application and ideas applied during its creation process. 
}

% 3 až 5 klíčových slov (doporučeno) oddělených \sep
% Hodí se pro nalezení práce podle tématu.
\def\ThesisKeywords{
návrh aplikace\sep MVVM\sep hledáni tras v otevřeném terénu
}

\def\ThesisKeywordsEN{
application desing\sep MVVM\sep path finding in open terrain
}

% Pokud některá z položek metadat obsahuje TeXové řídící sekvence, je potřeba
% dodat i verzi v obyčejném textu, která se objeví v metadatech formátu XMP
% zabudovaných do výstupního souboru PDF. Pokud si nejste jistí, prohlédněte si
% vygenerovaný soubor thesis.xmpdata.
\def\ThesisAuthorXMP{\ThesisAuthor}
\def\ThesisTitleXMP{\ThesisTitle}
\def\ThesisKeywordsXMP{\ThesisKeywords}
\def\AbstractXMP{\Abstract}

% Máte-li dlouhý abstrakt a nechceme se mu vejít na informační stranu,
% můžete použít toto nastavení ke zmenšení písma informační strany.
% (Uvažte nicméně zkrácení abstraktu, to je často lepší.)
\def\InfoPageFont{}
%\def\InfoPageFont{\small}  % odkomentujte pro zmenšení písma
